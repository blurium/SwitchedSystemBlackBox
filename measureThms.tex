% !TEX root =main.tex

For a given sampling $\omega \in X^{N*}$, let $V:=\{x \in X: f(p^*(\omega),x) > 0\}$, i.e., the set of points for which at least one constraint is violated, and $V_\sphere, V_M$ be its projections on $\sphere$ and $M$, respectively. 

\begin{lemma}$\sigma^{n-1}(V_\sphere) \leq \frac{\mu(V)}{m_1}$, where $m_1 = \min \{\mu_M(\{k\}), k \in M\}$.\end{lemma}
\begin{proof}

Let $\calA \subset \Sigma$, $\calA_{\sphere} = \pi_{\sphere} (\calA)$ and $\calA_M = \pi_M (\calA)$. We notice that $\Sigma_M$ is the disjoint union of its $2^m$ elements $\{B_i, i \in \{1,2, \ldots 2^m\} \}$. Then $\calA$ is the disjoint union $\calA = \sqcup_{1 \leq i \leq 2^m} (\calA_i, B_i)$ where $\calA_i = \pi_M^{-1} (B_i) \in \sphere$. We notice that 
$\calA_{\sphere} = \sqcup_{1 \leq i \leq 2^m} \calA_i$, 
and
\begin{equation*}
\sigma^{n-1} (\calA_{\sphere}) = \sum_{1 \leq i \leq 2^m} \sigma^{n-1} (\calA_i).
\end{equation*}
We have 
\begin{eqnarray*}
\mu(\calA) = \mu(\sqcup_{1 \leq i \leq 2^m} (\calA_i, B_i)) &=& \sum_{1 \leq i \leq 2^m} \mu( (\calA_i, B_i)) \\
 &=& \sum_{1 \leq i \leq 2^m} \sigma^{n-1} \otimes \mu_M ((\calA_i, B_i)) \\
 & = &\sum_{1 \leq i \leq 2^m} \sigma^{n-1}(\calA_i) \mu_M (B_i).
\end{eqnarray*}
Let $m_1$ be the minimum value  of $\mu_M$ on its atoms: $m_1 = \min \{\mu_M(\{k\}), k \in M\}$ (recall that $m_1 > 0$). Then since $ \forall \ i$, $\mu_M(B_i) \geq m_1$, we have
\begin{equation}
\sigma^{n-1}(\calA_{\sphere}) \leq \frac{\mu(\calA)}{m_1}.
\end{equation}
This proves our statement by taking $\calA = V_{\sphere}$.
%Remark: In our case, we consider the special case where measures $\mu_1$ and $\mu_2$ are uniform and of total measure $1$ on $\sphere$ and $M$. $\mu_1$ is the classical spherical measure $\sigma^n$ and $\mu_2$ the classical uniform probability measure on $M$. We have then $m_2 = \frac{1}{m}$ and the inequality becomes:
\end{proof}

\begin{corollary} \label{cor:measure}When the modes are sampled from the set $M$ uniformly random, 
\begin{equation*}\sigma^{n-1}(V_\sphere) \leq m \mu(V). \end{equation*}
\end{corollary}

We consider the linear transformation mapping $\sphere$ to $E_P$, denoted by $L \in \mathcal{L}(\mathbb{R}^n)$. Note that since $P \in \calS^n(\R)$, it can be written in its Choleski form $P = U D U^{-1}$, with $D$ diagonal matrix of its eigenvalues, and $U \in O_n(\mathbb{R})$. We define $D^{1/2}$ the positive square root of $D$ as the matrix diag($\sqrt{d_1},\dots, \sqrt{d_n}$). Then, the positive square root of $P$ is $U D^{1/2} U^{-1}$. This means that, $L = P^{1/2}$. For the rest of the write-up, we denote $$V':=\proj_{\sphere}(L^{-1}(V_\sphere)),$$ and show how to upper bound $\sigma^{n-1}(V')$ in terms of $\mu(V)$.


Let us denote by $B_{V_{\sphere}}$ the sector of $\ball$ defined by $V_{\sphere}$. We recall the following result:

\begin{lemma} 
Let $\psi$ a smooth change of coordinates in $\mathbb{R}^n$ and $\cal{D} \subset \mathbb{R}^n$, whose image under $\psi$ is $\cal{D}' \subset \mathbb{R}^n$. We provide $\R^n$ with the classical Borel $\sigma$-algebra and the Lebesgue measure $\lambda$. Then
\begin{equation}
\lambda(\calD') = \int_{x \in \calD} 1_{x \in \calD} |\det J(\psi(x))| d\lambda(x),
\end{equation}
which becomes when $\psi \in \mathcal{L}(\R^n)$ (and thus $\forall\ x \in \R^n, \ \det J(\psi(x)) = \det(\psi)$ )
\begin{equation}
\lambda(\calD') = |\det(\psi)| \lambda(\calD).
\end{equation}
\end{lemma}


\begin{theorem} \label{lemma:lip}
\begin{equation}
\sigma^{n-1}(\proj_\sphere(L^{-1}(V_\sphere))) \leq \det(L^{-1}) \left(\frac{1}{\lambda_{\min}(L^{-1})}\right)^n \sigma^{n-1}(V_\sphere).
\end{equation}   
\end{theorem}

\begin{proof}
Let denote $C= L^{-1}(B_{V_{\sphere}})$. We have $\proj_{\sphere}(C) = V'$ and $B_{V'} \subset \calH_{\lambda_{\min}(L^{-1})}(C)$. 

We have then $\sigma^{n-1}(V') = \lambda(B_{V'}) \leq \lambda (\calH_{\lambda_{\min}(L^{-1})}(C))$.

Hence $\sigma^{n-1}(V') \leq \frac{1}{\lambda_{\min}(L^{-1})^n} \lambda(C)$.

By the former lemma, we have that $ \sigma^{n-1}(\proj_\sphere(L^{-1}(V_\sphere))) = \leq \det(L^{-1}) \left(\frac{1}{\lambda_{\min}(L^{-1})}\right)^n \sigma^{n-1}(V_\sphere)$.

%Note that the mapping $\proj_\sphere$ can be seen as the composition of the $\proj_{\sphere_r}$ for some $r > 0$, and $\calH_{\frac{1}{r}}$. Let $E':=L^{-1}(\sphere)$, then when $r < \min_{x \in E'}\|x\| = \lambda_{\min}(L^{-1})$ we have
%\begin{equation*}\proj_{\sphere_{\lambda_{\min}}} (x)= \proj_{\ball_{\lambda_{\min}}}(x) \qquad \forall\, x \in E'.
%\end{equation*}
%This shows that the restriction of $\proj_{\sphere_{\lambda_{\min}}}$ to $E'$ is a convex projector.

%We consider then $B_{\lambda_{\min}}$, and $P_{B_{\lambda_{\min}}}$ projector on $B_{\lambda_{\min}}$ defined in $\mathbb{R}^n$.

%Then by Corollary \ref{cor:proj} 
%\begin{equation}
%\lVert \proj_{\sphere_{\lambda_{\min}}}(x) - \proj_{\sphere_{\lambda_{\min}}}(y) \rVert \leq \lVert x - y \rVert, \qquad \forall \ x,y \in E'.
%\end{equation} 
%This shows that $1$ is a Lipschitz constant of the function $\proj_{\sphere_{\lambda_{\min}}}$ on $E'$. 

%By composing $\proj_{\sphere_{\lambda_{\min}}}$ with $\calH_{\frac{1}{\lambda_{\min}}}$, we obtain $\proj_\sphere$. Since the %Lipschitz constant of composition of two functions can be bounded by the multiplication of Lipschitz constants of each %function, the Lipschitz constant of $\Pi_S$ on $E'$ is $\frac{1}{\lambda_{\min}}$, which means that:
%\begin{equation}\label{eqn:lip}
%\lVert \proj_\sphere(x) - \proj_\sphere(y) \rVert \leq \frac{1}{\lambda_{\min}} \lVert x-y \rVert, \qquad \forall\,x,y \in E'.
%\end{equation}
%Note that, the inequality in \eqref{eqn:lip} is an equality when $x$ is in the eigenspace of $\lambda_{\min}$ and $y=-x$.

%Recall that for any smooth Lipschitz function $\phi$ with Lipschitz constant, Lip$(\phi)$, we have for all $x$, $|\det(J(\phi(x))| \leq \text{Lip}(\phi)^n$. Combining this with  \eqref{eqn:lip} and Lemma \ref{lemma:lip}, we get the statement %of the lemma.
\end{proof}

\begin{corollary}$\sigma^{n-1}(V') \leq m\epsilon \sqrt{\frac{\lambda_{\max}(P)^n}{\det(P)}}$,
where $\mu(V) = \epsilon$.
\end{corollary}


