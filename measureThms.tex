% !TEX root =main.tex

Given a set $V$ of points violating the constraints of the optimization problem \eqref{eqn:campiOpt02}, we look first for an upper bound on the measure of points on $\sphere$ violating at least one constraint.

\begin{lemma}$$\sigma^{n-1}(V_\sphere) \leq \frac{\mu(V)}{m_1},$$ where $m_1 = \min \{\mu_M(\{j\}), j \in M\}$.\end{lemma}
\begin{proof}

Let $\calZ \subset \Sigma$, $\calZ_{\sphere} = \pi_{\sphere} (\calZ)$ and $\calZ_M = \pi_M (\calZ)$. We notice that $\Sigma_M$ is the disjoint union of its $2^m$ elements $\{\calM_i, i \in \{1,2, \ldots 2^m\} \}$. Then $\calZ$ can be written as the disjoint union $\calZ = \sqcup_{1 \leq i \leq 2^m} (\calS_i, \calM_i)$ where $\calS_i \in \Sigma(\sphere)$. We notice that 
$\calZ_{\sphere} = \sqcup_{1 \leq i \leq 2^m} \calS_i$, 
and
\begin{equation*}
\sigma^{n-1} (\calZ_{\sphere}) = \sum_{1 \leq i \leq 2^m} \sigma^{n-1} (\calS_i).
\end{equation*}
We have 
\begin{eqnarray*}
\mu(\calZ) &=& \mu(\sqcup_{1 \leq i \leq 2^m} (\calS_i, \calM_i)) \\
&=& \sum_{1 \leq i \leq 2^m} \mu(\calS_i, \calM_i) \\
 &=& \sum_{1 \leq i \leq 2^m} \sigma^{n-1} \otimes \mu_M (\calS_i, \calM_i) \\
 & = &\sum_{1 \leq i \leq 2^m} \sigma^{n-1}(\calS_i) \mu_M (\calM_i).
\end{eqnarray*}
Let $m_1$ be the minimum value  of $\mu_M$ on its atoms (recall that $m_1 > 0$):
$$m_1 = \min \{\mu_M(\{j\}), j \in M\}.$$ Then since $ \forall \ i$, $\mu_M(\calM_i) \geq m_1$, we have
\begin{equation}
\sigma^{n-1}(\calZ_{\sphere}) \leq \frac{\mu(\calZ)}{m_1}.
\end{equation}
This proves our statement by taking $\calZ = V$.
%Remark: In our case, we consider the special case where measures $\mu_1$ and $\mu_2$ are uniform and of total measure $1$ on $\sphere$ and $M$. $\mu_1$ is the classical spherical measure $\sigma^n$ and $\mu_2$ the classical uniform probability measure on $M$. We have then $m_2 = \frac{1}{m}$ and the inequality becomes:
\end{proof}

We recall that we consider the specific case of a uniform sampling both on $\sphere$ and on $M$. 

\begin{corollary} \label{cor:measure}When the modes are sampled from the set $M$ uniformly random, 
\begin{equation*}\sigma^{n-1}(V_\sphere) \leq m \mu(V). \end{equation*}
\end{corollary}

We now have the measure of points on $\sphere$ that may violate at least one constraint of $\Opt(\omega_N)$. These points correspond to points on $E_P$ where the system may not be $\gamma$-contracting. However, the computation of $\delta(\beta, \omega_N)$ on $E_P$ is hard. To circumvent this, we perform a change of coordinates so that the ellipsoid $E_P$ becomes a sphere in this new coordinate system. We now formalize how the measure of violating points in $E_P$ can be mapped to this new coordinate system.

We consider the linear transformation mapping $\sphere$ to $E_P$, denoted by $L \in \mathcal{L}(\R^n)$. Note that since $P \in \calS^n$, it can be written in its Choleski form $P = U D U^{-1}$, with $D$ diagonal matrix of its eigenvalues, and $U \in O_n$. We define $D^{1/2}$ the positive square root of $D$ as the matrix diag($\sqrt{d_1},\dots, \sqrt{d_n}$). Then, the positive square root of $P$ is $U D^{1/2} U^{-1}$. This means that $L = P^{1/2}$. 
We now examine $L^{-1}$, which maps points from $\sphere$ to the ellipsoid that is the image of $\sphere$ after the linear transformation. We denote $$V':=\proj_{\sphere}(L^{-1}(V_\sphere)).$$
Note that, $V'$ is the projection of the violating points from the ellipsoid corresponding to the image of the unit sphere in the original coordinates to the unit sphere in the new coordinate system.  We now show how to over approximate $\sigma^{n-1}(V')$ in terms of $\mu(V)$.

\begin{remark} \label{rem:mappingMeasures}
If $\psi$ is a smooth change of coordinates in $\R^n$ and $X$ $\subset \mathbb{R}^n$, whose image under $\psi$ is $X \subset \R^n$, then
\begin{equation}
\lambda(X') = \int_{x \in X} 1_{x \in X} |\det J(\psi(x))| d\lambda(x),
\end{equation}
which becomes when $\psi \in \mathcal{L}(\R^n)$: (and thus $\forall\ x \in \R^n, \ \det J(\psi(x)) = \det(\psi)$)
\begin{equation}
\lambda(X') = |\det(\psi)| \lambda(X).
\end{equation}
\end{remark}

\begin{theorem} \label{lemma:lip}
\begin{equation}
\sigma^{n-1}(\proj_\sphere(L^{-1}(V_\sphere))) \leq \frac{\det(L^{-1})}{\left(\lambda_{\min}(L^{-1})\right)^n} \sigma^{n-1}(V_\sphere).
\end{equation}   
\end{theorem}

\begin{proof}
Let us consider $S^{V_{\sphere}}$ the sector of $\ball$ defined by $V_{\sphere}$. We denote $C:= L^{-1}(S^{V_{\sphere}})$. We have $\proj_{\sphere}(C) = V'$ and $S^{V'} \subset \calH_{1/ \lambda_{\min}(L^{-1})}(C)$.  We have then 
$$\sigma^{n-1}(V') = \lambda(S^{V'}) \leq \lambda (\calH_{1/ \lambda_{\min}(L^{-1})}(C)),$$ which means: $$\sigma^{n-1}(V') \leq \frac{1}{\lambda_{\min}(L^{-1})^n} \lambda(C).$$ Using Remark \ref{rem:mappingMeasures}, we have the result of the theorem.

%Note that the mapping $\proj_\sphere$ can be seen as the composition of the $\proj_{\sphere_r}$ for some $r > 0$, and $\calH_{\frac{1}{r}}$. Let $E':=L^{-1}(\sphere)$, then when $r < \min_{x \in E'}\|x\| = \lambda_{\min}(L^{-1})$ we have
%\begin{equation*}\proj_{\sphere_{\lambda_{\min}}} (x)= \proj_{\ball_{\lambda_{\min}}}(x) \qquad \forall\, x \in E'.
%\end{equation*}
%This shows that the restriction of $\proj_{\sphere_{\lambda_{\min}}}$ to $E'$ is a convex projector.

%We consider then $B_{\lambda_{\min}}$, and $P_{B_{\lambda_{\min}}}$ projector on $B_{\lambda_{\min}}$ defined in $\mathbb{R}^n$.

%Then by Corollary \ref{cor:proj} 
%\begin{equation}
%\lVert \proj_{\sphere_{\lambda_{\min}}}(x) - \proj_{\sphere_{\lambda_{\min}}}(y) \rVert \leq \lVert x - y \rVert, \qquad \forall \ x,y \in E'.
%\end{equation} 
%This shows that $1$ is a Lipschitz constant of the function $\proj_{\sphere_{\lambda_{\min}}}$ on $E'$. 

%By composing $\proj_{\sphere_{\lambda_{\min}}}$ with $\calH_{\frac{1}{\lambda_{\min}}}$, we obtain $\proj_\sphere$. Since the %Lipschitz constant of composition of two functions can be bounded by the multiplication of Lipschitz constants of each %function, the Lipschitz constant of $\Pi_S$ on $E'$ is $\frac{1}{\lambda_{\min}}$, which means that:
%\begin{equation}\label{eqn:lip}
%\lVert \proj_\sphere(x) - \proj_\sphere(y) \rVert \leq \frac{1}{\lambda_{\min}} \lVert x-y \rVert, \qquad \forall\,x,y \in E'.
%\end{equation}
%Note that, the inequality in \eqref{eqn:lip} is an equality when $x$ is in the eigenspace of $\lambda_{\min}$ and $y=-x$.

%Recall that for any smooth Lipschitz function $\phi$ with Lipschitz constant, Lip$(\phi)$, we have for all $x$, $|\det(J(\phi(x))| \leq \text{Lip}(\phi)^n$. Combining this with  \eqref{eqn:lip} and Lemma \ref{lemma:lip}, we get the statement %of the lemma.
\end{proof}

\begin{corollary}$\sigma^{n-1}(V') \leq m\epsilon \sqrt{\frac{\lambda_{\max}(P)^n}{\det(P)}}$,
where $\mu(V) = \epsilon$.
\end{corollary}


