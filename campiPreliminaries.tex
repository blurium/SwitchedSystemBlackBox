% !TEX root =main.tex

The proof consists of finding an upper-bound on the measure of points of $\sphere$ violating the constraints of $\Opt(\omega_N)$, and then of computing an upper-bound for $\delta$.

We assume from now that $\sphere$ is provided with $(\mathcal{B}_{\sphere},\sigma^{n-1})$ and that $M$ is provided with the classical $\sigma$-algebra associated to finite sets: $\Sigma_M = \wp(M)$, where $\wp(M)$ is the power set of $M$. We consider an unsigned finite measure\footnote{Recall that the support of a measure $\mu$ defined on a measurable space $(X, \Sigma)$ is $\text{supp}(\mu) = \overline{\{A \in \Sigma | \mu(A) > 0 \}}$} $\mu_M$ on $(M, \Sigma_M)$ with $\text{supp} (\mu_M) = M$. In other words, $\forall\,j \in M$, $\mu_M(\{j\})  > 0$. We denote the product $\sigma$-algebra $\mathcal{B}_{\sphere} \bigotimes \Sigma_M$ engendered by $\mathcal{B}_{\sphere}$ and $\Sigma_M$: $\Sigma = \sigma( \pi_{\sphere}^{-1}(\mathcal{B}_{\sphere}),  \pi_{M}^{-1}(\Sigma_M))$. On this set, we define the product measure $\mu = \sigma^{n-1} \otimes \mu_M$ which is an unsigned finite measure on $Z$. We also define $V_\sphere:=\pi_S(V), V_M:=\pi_M(V)$.