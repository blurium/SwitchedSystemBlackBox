% !TEX root = main.tex
We consider the usual Hilbert finite normed vector space $(\R^n,\ell_2)$, $n \in \mathbb{N}_{> 0}$, $\ell_2$ the classical euclidean norm. We denote a unit ball in $\R^n$ with $\ball$ and unit sphere in $\R^n$ of radius $r$ as $\sphere$. We only denote the radius $r$ explicitly as in $\ball_r$ and $\sphere_r$, when $r$ is different than $1$. We denote by $\proj_{\sphere}$ the (convex) projector on $\sphere$. We denote the set of real symmetric matrices of size $n$ by $\calS^n(\R)$, and the set of linear functions in $\R^n$ by $\mathcal{L}(\mathbb{R}^n)$. We denote the ellipsoid described by the matrix $P \in \mathbb{S}^n$ as $E_P$. We denote the homothety of ratio $\lambda$ by $\calH_{\lambda}$.

We denote by $M$ the set $\{1,2,\dots,m\}$, ($m \in \N_{>0}$) and consider a switched linear system of modes $\calM = \{A_1, A_2,\dots, A_m\} \subset \R^{n \times n}$ indexed by $M$. We observe the system: 
\begin{equation}\label{eq:dynamicalsystem}x_{k+1} = A_{\sigma(k)}x_k,
\end{equation}
where, $x_k \in \R^n$, k is index of time and $\sigma: \N \to M$ is the switching sequence. Let $y_k := x_{k+1}$. 
We assume that we only know the number of modes $m$, the input $x_k$ and the output $y_k$ at each time $k$. We ignore what are the matrices and the index of the matrix which is applied at every time. We consider the following problem.

In this case the Lyapunov exponent is known as the Joint Spectral Radius of the set of matrices, which can be alternatively defined as follows:
\begin{definition}  \cite{jungers_lncis} Given a set of matrices $\cM \subset \re^{n\times n},$ its \emph{joint spectral radius} (JSR) is given by
$$\rho(\cM) =\lim_{t\rightarrow \infty} \max_{i_1,\dots, i_t}\{||A_{i_1} \dots A_{i_t}||^{1/t}: A_i\in\cM\}. $$
\end{definition}

\begin{remark}
 Note that one can scale the problem because the JSR is homogeneous:
$$\rho(\cM/\lambda)=\rho(\cM)/\lambda\, \forall \lambda>0, $$ and $\cM/\lambda$ can be studied by studying the scaled inputs $$(x_t, y_t/\lambda,\sigma(t)).$$
\end{remark}

Under certain conditions \textcolor{red}{Where is this?} deciding stability amounts to decide whether $\rho<1.$  In order to understand the quality of our techniques, we will actually try to prove lower and upper bounds on $\rho.$ 
