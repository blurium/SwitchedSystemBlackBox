% !TEX root = main.tex
We consider the usual finite normed vector space $(\R^n,\ell_2)$, $n \in \mathbb{N}_{> 0}$, with $\ell_2$ the classical Euclidean norm. We denote the set of linear functions in $\R^n$ by $\mathcal{L}(\mathbb{R}^n)$, and the set of real symmetric matrices of size $n$ by $\calS^n$. In particular, the set of positive definite matrices, which are matrices $P \in \calS^n$ such that $\forall\ x \in \R^n \setminus \{0\}$, $x^TPx > 0$, is denoted by $\calS^n_{++}$. We write $P \succ 0$ to state that $P$ is positive definite. Given a set $X \subset \R^n$, and $r \in \R_{> 0}$ we write \mbox{$rX := \{x \in X : rx\}$} to denote the scaling of this set. We denote by $\ball$ (respectively $\sphere$) the ball (respectively sphere) of unit radius centered at the origin.  We denote the ellipsoid described by the matrix $P \in \calS^n_{++}$ as $E_P$, i.e., $E_P:= \{x \in \R^n: x^TPx = 1\}$, and we denote by $\tilde{E}_P$ the volume in $\R^n$ defined by $E_P$: $\tilde{E}_P = \{x \in \R^n: x^TPx \leq 1\}$. We denote the spherical projector on $\sphere$ by $\proj_{\sphere}$. %We denote the homothety of ratio $r$ by $\calH_{r}$. 
%We denote by $\wp(M)$ the power set of $M$.

%We denote by $M$ the set $\{1,2,\dots,m\}$, ($m \in \N_{>0}$) and consider a switched linear system of modes $\calM = \{A_1, A_2,\dots, A_m\} \subset \R^{n \times n}$ indexed by $M$. We observe the system: 
%\begin{equation}\label{eq:dynamicalsystem}x_{k+1} = A_{\sigma(k)}x_k,
%\end{equation}
%where, $x_k \in \R^n$, k is index of time and $\sigma: \N \to M$ is the switching sequence. Let $y_k := x_{k+1}$. 
%We assume that we only know the number of modes $m$, the input $x_k$ and the output $y_k$ at each time $k$. We ignore what are the matrices and the index of the matrix which is applied at every time. We consider the following problem.

For an ellipsoid centered at the origin, and for any of its subsets $\calA$, the \emph{sector} defined by $\calA$ is the subset $$\{t \calA, \ t \in [0,1]\} \subset\ \R^n.$$ We denote a sector induced by $\calA \subset E_P$ by $E_P^{\calA}$. In the particular case of the unit sphere, we instead write $S^{\calA}$.

We consider in this work the classical unsigned and finite uniform spherical measure on $\sphere$, denoted by $\sigma^{n-1}$. It is associated to $\mathcal{B}_{\sphere}$, the spherical Borelian $\sigma$-algebra, and is derived from the Lebesgue measure $\lambda$. We have $\mathcal{B}_{\sphere}$ defined by $\calA \in \mathcal{B}_{\sphere}$ if and only if $S^{\calA} \in \mathcal{B}_{\mathbb{R}^n}$. The spherical measure $\sigma^{n-1}$ is defined by
$$\forall\ \calA \in \mathcal{B}_{\sphere},\, \sigma(\calA) = \frac{\lambda(S^{\calA})}{\lambda(B)}. $$
In other words, the spherical measure of a subset of the sphere is related to the Lebesgue measure of the sector of the unit ball it induces. Notice that $\sigma^{n-1}(\sphere) = 1$.
Since $P \in \calS_{++}^n$, it can be written in its Choleski form $P = UDU^{-1}$, where $D$ is the diagonal matrix of its eigenvalues and $U$ is an orthogonal matrix (see e.g. \cite{boyd}). Let \begin{equation}\label{choleski}L:=UD^{1/2}U^{-1}.\end{equation} Note that, $L^{-1}$ maps the elements of $\sphere$ to $E_P$. Then, we define the measure on the ellipsoid $\sigma_P$ on the $\sigma$-algebra $\mathcal{B}_{E_P}:=L^{-1}\mathcal{B}_\sphere$, where $\forall\, \calA \in \mathcal{B}_{E_P},\, \sigma_P({\calA}) = \sigma^{n-1}(L\calA)$. 

%
%
%
%We now relate $\sigma^{n-1}(V_{\sphere})$ to $\sigma_P(E')$. Consider the transformation linear transformation $L \in \calL({\R^n})$ mapping $\sphere$ to $E_P$. Note that since $P \in \calS^n$, it can be written in its Choleski form $P = U D U^{-1}$, with $D$ diagonal matrix of its eigenvalues, and $U \in O_n$. Then, $L = P^{1/2}=U D^{1/2} U^{-1}$ where $D^{1/2}$ is the diagonal matrix comprised of square root of elements of $D$.


For $m \in \N_{>0}$, we denote by $M$ the set $M=\{1,2, \ldots,m \}$. Set $M$ is provided with the classical $\sigma$-algebra associated to the finite sets: $\Sigma_M = \wp(M)$, where $\wp(M)$ is the set of subsets of $M$. We consider the uniform measure $\mu_M$ on $(M, \Sigma_M)$. 

We define $Z = \sphere \times M$ as the Cartesian product of the unit sphere and $M$. We denote the product $\sigma$-algebra $\mathcal{B}_{\sphere} \bigotimes \Sigma_M$ generated by $\mathcal{B}_{\sphere}$ and $\Sigma_M$: $\Sigma = \sigma( \pi_{\sphere}^{-1}(\mathcal{B}_{\sphere}),  \pi_{M}^{-1}(\Sigma_M))$, where $\proj_{\sphere}:Z \to \sphere$ and $\proj_M: Z \to M$ are the standard projections. On this set, we define the product measure $\mu = \sigma^{n-1} \otimes \mu_M$. Note that, $\mu$ is a uniform measure on $Z$ and $\mu(Z)=1$.  