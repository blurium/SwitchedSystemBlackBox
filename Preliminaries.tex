% !TEX root = main.tex
We consider the usual finite normed vector space $(\R^n,\ell_2)$, $n \in \mathbb{N}_{> 0}$, with $\ell_2$ the classical Euclidean norm. We denote the set of linear functions in $\R^n$ by $\mathcal{L}(\mathbb{R}^n)$, the set of orthogonal matrices of size $n$ by $\mathcal{O}_n$, and the set of real symmetric matrices of size $n$ by $\calS^n$. A matrix $P \in \calS^n$ is said to be positive if and only if $\forall\ x \in \R^n$, $x^TPx \geq 0$, which we denote by $P \succeq 0$. The matrix $P$ is positive definite if $P \succeq 0$ and $x^T P x = 0 \implies x=0_{\R^n}$, which we denote by $P \succ 0$. We denote the ball (respectively sphere) of radius $r$ centered at the origin in $\R^n$ with $r\ball$ (respectively $r\sphere$). We $r=1$, we use the simplified notations $\ball$ and $\sphere$. We denote the ellipsoid described by the matrix $P \in \calS^n$ as $E_P$, i.e., $E_P:= \{x \in \R^n: x^TPx = 1\}$, and we denote by $\tilde{E}_P$ the volume in $\R^n$ defined by $E_P$: $\tilde{E}_P = \{x \in \R^n: x^TPx \leq 1\}$. We denote the spherical projector on $\sphere$ by $\proj_{\sphere}$[no ambiguity on uniqueness now!]. We denote the homothety of ratio $r$ by $\calH_{r}$. 
%We denote by $\wp(M)$ the power set of $M$.

%We denote by $M$ the set $\{1,2,\dots,m\}$, ($m \in \N_{>0}$) and consider a switched linear system of modes $\calM = \{A_1, A_2,\dots, A_m\} \subset \R^{n \times n}$ indexed by $M$. We observe the system: 
%\begin{equation}\label{eq:dynamicalsystem}x_{k+1} = A_{\sigma(k)}x_k,
%\end{equation}
%where, $x_k \in \R^n$, k is index of time and $\sigma: \N \to M$ is the switching sequence. Let $y_k := x_{k+1}$. 
%We assume that we only know the number of modes $m$, the input $x_k$ and the output $y_k$ at each time $k$. We ignore what are the matrices and the index of the matrix which is applied at every time. We consider the following problem.

For an ellipsoid centered at the origin, and for any of its subsets $\calA$, the sector defined by $\calA$ is the subset $$\{t \calA, \ t \in [0,1]\} \subset\ \R^n.$$ A sector induced by $\calA \subset E_P$ will be denoted by $E_P^{\calA}$. In the particular case of the unit sphere, we instead write $S^{\calA}$.

We consider in this work the classical unsigned and finite uniform spherical measure on $\sphere$, denoted by $\sigma^{n-1}$. It is associated to $\mathcal{B}_{\sphere}$, the spherical Borelian $\sigma$-algebra, and is derived from the Lebesgue measure $\lambda$. We have $\mathcal{B}_{\sphere}$ defined by $\calA \in \mathcal{B}_{\sphere}$ if and only if $S^{\calA} \in \mathcal{B}_{\mathbb{R}^n}$. The spherical measure $\sigma^{n-1}$ is defined by
$$\forall\ \calA \in \mathcal{B}_{\sphere},\, \sigma(\calA) = \frac{\lambda(S^{\calA})}{\lambda(B)}. $$
In other words, the spherical measure of a subset of the sphere is related to the Lebesgue measure of the sector of the unit ball it induces. Notice that $\sigma^{n-1}(\sphere) = 1$.
Similarly, we have a measure on the ellipsoid $\sigma_P$ defined on the $\sigma$-algebra $\mathcal{B}_{E_P}$ by: $\forall\ \calA \in \mathcal{B}_{E_P}$, $\sigma_P(\calA) = \frac{\lambda(E_P^{\calA})}{\lambda(\tilde{E}_P)}$.

For $m \in \N_{>0}$, we denote by $M$ the set $M=\{1,2, \ldots,m \}$. Set $M$ is provided with the classical $\sigma$-algebra associated to finite sets: $\Sigma_M = \wp(M)$, where $\wp(M)$ is the power set of $M$. We consider the uniform measure $\mu_M$ on $(M, \Sigma_M)$. 

We define $Z = \sphere \times M$ as the Cartesian product of the unit sphere and $M$. We denote the product $\sigma$-algebra $\mathcal{B}_{\sphere} \bigotimes \Sigma_M$ generated by $\mathcal{B}_{\sphere}$ and $\Sigma_M$: $\Sigma = \sigma( \pi_{\sphere}^{-1}(\mathcal{B}_{\sphere}),  \pi_{M}^{-1}(\Sigma_M))$. On this set, we define the product measure $\mu = \sigma^{n-1} \otimes \mu_M$. We have $\mu$ uniform and $\mu(Z)=1$.  