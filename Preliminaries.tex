% !TEX root = main.tex
We consider the usual Hilbert finite normed vector space $(\R^n,\ell_2)$, $n \in \mathbb{N}_{> 0}$, $\ell_2$ the classical euclidean norm. We denote a unit ball in $\R^n$ with $\ball$ and unit sphere in $\R^n$ of radius $r$ as $\sphere$. We only denote the radius $r$ explicitly as in $\ball_r$ and $\sphere_r$, when $r$ is different than $1$. We denote by $\proj_{\sphere}$ the (convex) projector on $\sphere$. We denote the set of real symmetric matrices of size $n$ by $\calS^n(\R)$, and the set of linear functions in $\R^n$ by $\mathcal{L}(\mathbb{R}^n)$. We denote the ellipsoid described by the matrix $P \in \mathbb{S}^n$ as $E_P$, i.e., $E_P:= \{x \in \R^n: x^TPx = 1\}$. We denote the homothety of ratio $r$ by $\calH_{r}$. 
%We denote by $\wp(M)$ the power set of $M$.

%We denote by $M$ the set $\{1,2,\dots,m\}$, ($m \in \N_{>0}$) and consider a switched linear system of modes $\calM = \{A_1, A_2,\dots, A_m\} \subset \R^{n \times n}$ indexed by $M$. We observe the system: 
%\begin{equation}\label{eq:dynamicalsystem}x_{k+1} = A_{\sigma(k)}x_k,
%\end{equation}
%where, $x_k \in \R^n$, k is index of time and $\sigma: \N \to M$ is the switching sequence. Let $y_k := x_{k+1}$. 
%We assume that we only know the number of modes $m$, the input $x_k$ and the output $y_k$ at each time $k$. We ignore what are the matrices and the index of the matrix which is applied at every time. We consider the following problem.

We consider the classical unsigned and finite uniform spherical measure on $\sphere$, commonly denoted by $\sigma^{n-1}$. It is associated to the spherical Borelian $\sigma$-algebra and is derived from the Lebesgue measure $\lambda$. We have $\mathcal{B}_{\sphere}$ with $\calA \in \mathcal{B}_{\sphere}$ if and only if the sector $t \calA$, $t \in [0,1]$ is in $\mathcal{B}_{\mathbb{R}^n}$. The spherical measure $\sigma^{n-1}$ is defined by
$$\forall\ \calA \in \mathcal{B}_{\sphere},\, \sigma(\calA) = \frac{\lambda(t\calA)}{\lambda(\mathcal{B})}. $$
In other words, the spherical measure of a subset of the sphere is related to the Lebesgue measure of the sector of the unit ball it induces. Notice that $\sigma^{n-1}(\sphere) = 1$.

