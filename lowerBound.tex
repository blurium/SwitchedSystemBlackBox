% !TEX root = main.tex
We start by proving the lower bound, which is based on the following theorem from the switched linear system literature.

\begin{theorem}\cite[Theorem 2.11]{jungers_lncis}\label{thm:john}
For any bounded set of matrices such that $\rho(\cM)<\frac{1}{\sqrt{n}},$ there exists a Common Quadratic Lyapunov Function (CQLF) for $\cM,$ that is, a $P\succ 0$ such that: $$\forall A\in \cM,\, A^TPA\preceq P. $$
\end{theorem}

The following theorem shows that the existence of a CQLF for \eqref{eq:switchedSystem} can be checked by collecting $N$ observation pairs.
%
%The existence of a CQLF for our (potentially scaled) blackbox system is certainly something we can check: after collecting $N$ observations, one can solve the following optimization problem efficiently. 
%

%Indeed, when $\lambda $ is fixed, the problem is a set of LMIs, and $\lambda $ can be optimized by bisection.

\begin{theorem}
For a given sampling: $$\omega_N = \{x_1, x_2, \ldots, x_N\},$$ let $\gamma^*(\omega_N)$ be the optimal solution of the following optimization problem:
\begin{eqnarray}
\nonumber \mbox{min}&&\gamma\\
 s.t.& & \label{eq:lowerbound}y_i^T P y_i \leq \gamma^2 x_i^TPx_i,\,  \forall \,i : \,1\leq i \leq N\\
\nonumber && P \succ 0.
\end{eqnarray}
If $\gamma^*(\omega_N)<\infty,$ we have:
$$\rho(\cM) \geq \frac{\gamma^*}{\sqrt{n}}.$$ Note that, \eqref{eq:lowerbound} can be solved by bisection on $\gamma$.

\end{theorem}
\begin{proof}
Using Remark \ref{rem:scaling}, for any $\epsilon > 0$, $\frac{\cM}{(\gamma^*-\epsilon),}$ has no CQLF. Then, applying Theorem \ref{thm:john} we get
\begin{equation*}\frac{\rho(\cM)}{\gamma^*} \geq \frac{1}{\sqrt{n}}.\end{equation*}
\end{proof}