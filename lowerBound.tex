% !TEX root = main.tex
We start by computing a lower bound for $\rho$ which is based on the following theorem from the switched linear systems literature.
\begin{theorem}\cite[Theorem 2.11]{jungers_lncis}\label{thm:john}
For any finite set of matrices such that $\rho(\cM)<\frac{1}{\sqrt{n}},$ there exists a Common Quadratic Lyapunov Function (CQLF) for $\cM,$ that is, a $P\succ 0$ such that: $$\forall\ A\in \cM,\, A^TPA\preceq P. $$
\end{theorem}

CQLFs are useful because they can be computed (if they exist) with semidefinite programming (see \cite{boyd}), and they constitute a stability guarantee for switched systems as we formalize next.\begin{theorem}\cite[Prop. 2.8]{jungers_lncis}\label{thm:cqlf} Consider a finite set of matrices $\cM$. If there exist a $\gamma \geq 0$ and $P \succ 0$ such that $$\forall\ A \in \calM, A^TPA \preceq \gamma^2P,$$ then $\rho(\cM) \leq \gamma$.
\end{theorem}
Note that the smaller $\gamma$ is, the tighter is the upper bound we get on $\rho(\cM)$. Therefore, we can consider, in particular, the optimal solution $\gamma^*$ of the following optimization problem:
\begin{equation}\label{eqn:campiOpt0}
\begin{aligned}
& \text{min}_{\gamma, P} & & \gamma \\
& \text{s.t.} 
&  & (Ax)^TP(Ax) \leq \gamma^2 x^TPx,\,\forall\ A \in \calM, \,\forall\, x \in \R^n,\\
& && P \succ 0. \\
\end{aligned}
\end{equation}
%where we were able to replace the condition $\forall\,x \in \R^n$ with $\,\forall\, x \in \sphere$, where $\sphere$ is the unit sphere. This is thanks to the Property \ref{property:homogeneity} which implies that it is sufficient to show that a CQLF is decreasing on a set enclosing the origin, e.g. the unit sphere. 

Even though this upper bound is more difficult to obtain in a black-box setting where only a finite number of observations are available, in this section we leverage Theorem \ref{thm:john} in order to derive a straightforward lower bound.

The following theorem shows that the existence of a CQLF for \eqref{eq:switchedSystem} can be checked by considering $N$ pairs $(x_i, j_i) \in \R^n \times M$, where $i \in \{1, \ldots N\}.$ Recall that in our setting, we assume that we observe pairs of the form $(x_k,x_{k+1}),$ but we do not observe the mode applied to the system during this time step.
%The existence of a CQLF for our (potentially scaled) blackbox system is certainly something we can check: after collecting $N$ observations, one can solve the following optimization problem efficiently. 
%

%Indeed, when $\lambda $ is fixed, the problem is a set of LMIs, and $\lambda $ can be optimized by bisection.
\begin{theorem}\label{thm:lowerbound}
For a given uniform sampling: $$\omega_N := \{(x_1, j_1), (x_2, j_2), \ldots, (x_N, j_N)\} \subset \R^n \times M,$$ let $W_{\omega_N}=\{(x_1,y_1), \ldots,(x_N,y_N)\}$ be the corresponding available observations, which satisfy $$y_i= A_{j_i}x_i\,\, \forall (x_i, y_i) \in W_{\omega_N}.$$
Also let $\gamma^*(\omega_N)$ be the optimal solution of the following optimization problem:
\begin{equation}\label{eq:lowerbound}
\begin{aligned}
& \text{min}_P & & \gamma \\
& \text{s.t.} 
&  & (y_i)^T P (y_i) \leq \gamma^2 x_i^TPx_i,\,  \forall (x_i, y_i) \in W_{\omega_N}\\
& && P \succ 0. \\
\end{aligned}
\end{equation}
Then, we have:
$$\rho(\cM) \geq \frac{\gamma^*(\omega_N)}{\sqrt{n}}.$$ Note that, \eqref{eq:lowerbound} can be efficiently solved by semidefinite programming and bisection on the variable $\gamma$ (see \cite{boyd}).
\end{theorem}
\begin{proof}
Let $\epsilon >0$. By definition of $\gamma^*$,  there exists no matrix $P \in \calS^n_{++}$ such that:
\begin{equation*}(Ax)^T P (Ax) \leq (\gamma^*(\omega_N) -\epsilon)^2x^TPx,\quad \forall x \in \R^n, \forall\, A \in \calM.
\end{equation*}
By Property \ref{rem:scaling}, this means that there exists no CQLF for the scaled set of matrices $\frac{\cM}{(\gamma^*(\omega_N)-\epsilon)}$. Then, using Theorem~\ref{thm:john}, we conclude:
\begin{equation*}\frac{\rho(\cM)}{\gamma^*(\omega_N)} \geq \frac{1}{\sqrt{n}}.\end{equation*}
\end{proof}
