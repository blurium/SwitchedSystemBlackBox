% !TEX root = main.tex
For the lower bound, we will leverage the following theorem from the switching system literature.

\begin{theorem}\cite[Theorem 2.11]{jungers_lncis}\label{thm:john}
For any bounded set of matrices such that $\rho(\cM)<1/\sqrt(n),$ there exists a Common Quadratic Lyapunov Function (CQLF) for $\cM,$ that is, a $P\succeq 0$ such that $$\forall A\in \cM,\, A^TPA\preceq P. $$
\end{theorem}

%\comrj{We can probably generalize this result.  What is needed is a Converse Convex Lyapunov Theorem: stability implies the existence of a convex (or, quasiconvex, it should suffice) Lyapunov function.  Maybe we should use Mircea's result \cite{geiselhart2014alternative}.  This would be nice per se, I think.}
%
%\begin{theorem}{jungers_lncis}\label{thm:john}
%If a For any bounded set of matrices such that $\rho(\cM)<1/\sqrt(n),$ there exists a Common Quadratic Lyapunov Function (CQLF) for $\cM,$ that is, a $P\succeq 0$ such that $$\forall A\in \cM,\, A^TPA\preceq P. $$
%\end{theorem}

The existence of a CQLF for our (potentially scaled) blackbox system is certainly something we can check: after collecting $N$ observations, one can solve the following optimization problem efficiently. 

\begin{eqnarray}
\nonumber \mbox{min}&&\lambda\\
 s.t.& & \label{eq:lowerbound} \forall 1\leq i \leq N,\ (y_i^T P y_i)/\lambda^2 \leq x_i^TPx_i\\
\nonumber && P \succeq 0.
\end{eqnarray}
Indeed, when $\lambda $ is fixed, the problem is a set of LMIs, and $\lambda $ can be optimized by bisection.

\begin{theorem}
Let $\lambda^*$ be the minimum $\lambda$ such that \eqref{eq:lowerbound} above has a solution.  If $\lambda^*<\infty,$ one has $$\rho(\cM) \geq \lambda^*/\sqrt(n) .$$
\end{theorem}
\begin{proof}
Just apply Theorem \ref{thm:john} to $\cM /(\lambda^*-\epsilon),$ for any $\epsilon>0.$ Since this latter set has no CQLF, we obtain that $\rho(\cM)/\lambda^*\geq 1/\sqrt{n}.$
\end{proof}
This result could be improved in several ways: first, it is classical in the JSR literature to replace the CQLF with a SOS polynomial of degree $2d,$ $d>1.$ this narrows the $1/\sqrt(n)$ accuracy factor, up to one when $d\rightarrow \infty.$  \comrj{this is not 100 \% clear but interesting: if we take a too large degree for a fixed number of points, we will always find a very small lower bound! Nevertheless I hope that it will be easy to understand how increasing the degree can improve the accuracy, and anyway there are interesting things there.}
