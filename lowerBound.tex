% !TEX root = main.tex
CQLFs are useful because they can be computed (if they exist) with Semidefinite Programming (see \cite{boyd}), and they constitute a stability guarantee for the switched system as the following theorem formalizes:
\begin{theorem}\cite[Prop. 2.8]{jungers_lncis}\label{thm:cqlf} Consider a bounded set of matrices $\cM$. If there exists a $\gamma \geq 0$ and $P \succ 0$ such that $$\forall\, A \in \calM, A^TPA \preceq \gamma^2P,$$ then $\rho(\cM) \leq \gamma$. 
\end{theorem}

Even though an upper bound is more difficult to obtain in the black box model where only a finite number of observations are available, in this section we leverage Theorem 3.1 in order to derive an easy lower bound in our setting.

\begin{theorem}\cite[Theorem 2.11]{jungers_lncis}\label{thm:john}
For any bounded set of matrices such that $\rho(\cM)<\frac{1}{\sqrt{n}},$ there exists a Common Quadratic Lyapunov Function (CQLF) for $\cM,$ that is, a $P\succ 0$ such that: $$\forall A\in \cM,\, A^TPA\preceq P. $$
\end{theorem}

The following theorem shows that the existence of a CQLF for \eqref{eq:switchedSystem} can be checked by considering $N$ pairs $(x_i, j_i) \in \R^n \times M$, where $i \in \{1, \ldots N\}.$ Recall that in our setting, we assume that we observe pairs of the form $(x_t,x_{t+1}),$ but we do not observe the mode applied to the system during this time step.
%The existence of a CQLF for our (potentially scaled) blackbox system is certainly something we can check: after collecting $N$ observations, one can solve the following optimization problem efficiently. 
%

%Indeed, when $\lambda $ is fixed, the problem is a set of LMIs, and $\lambda $ can be optimized by bisection.
\begin{theorem}
For a given homogenous sampling: $$\omega_N := \{(x_1, j_1), (x_2, j_2), \ldots, (x_N, j_N)\} \subset \R^n \times M,$$ and let $S_{\omega_N}=\{(x_1,y_1),...,(x_N,y_N)\}$ be the corresponding observations available, satisfying $$y_i= A_{j_i}x_i .$$
Defining $\gamma^*(\omega_N)$ be the optimal solution of the following optimization problem:
\begin{equation}\label{eq:lowerbound}
\begin{aligned}
& \text{min} & & \gamma \\
& \text{s.t.} 
&  & (y_i)^T P (y_i) \leq \gamma^2 x_i^TPx_i,\,  \forall \,i :1\leq i \leq N.\\
& && P \succ 0 \\
\end{aligned}
\end{equation}
Then, we have:
$$\rho(\cM) \geq \frac{\gamma^*(\omega_N)}{\sqrt{n}}.$$ Note that, \eqref{eq:lowerbound} can be efficiently solved by semidefinite programming and bisection on gamma \cite{boyd}.

\end{theorem}
\begin{proof}
By definition of $\gamma^*$, for any $\epsilon > 0$, there exists no $P \succ 0$ such that \eqref{eq:lowerbound} is satisfied. By using Remark \ref{rem:scaling} this means that, there exists no CQLF for the scaled set of matrices $\frac{\cM}{(\gamma^*(\omega_N)-\epsilon)}$. Then, by Theorem \ref{thm:john} we get:
\begin{equation*}\frac{\rho(\cM)}{\gamma^*(\omega_N)} \geq \frac{1}{\sqrt{n}}.\end{equation*}
\end{proof}