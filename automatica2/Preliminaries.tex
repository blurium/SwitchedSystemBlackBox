% !TEX root = main.tex
We consider the usual finite normed vector space $(\mathbb{R}^n,\ell_2)$, $n \in \mathbb{N}_{> 0}$, with $\ell_2$ the classical Euclidean norm. We denote by $\lVert x \rVert$ the $\ell_2$-norm of $x \in \mathbb{R}^n$. We also denote the set of linear functions in $\mathbb{R}^n$ by $\mathcal{L}(\mathbb{R}^n)$, and the set of real symmetric matrices of size $n$ by $\mathcal{S}^n$. In particular, the set of positive definite matrices is denoted by $\mathcal{S}^n_{++}$. We write $P \succ 0$ to state that $P$ is positive definite, and $P \succeq 0$ to state that $P$ is positive semi-definite. Given a set $X \subset \mathbb{R}^n$, and $r \in \mathbb{R}_{> 0}$ we write \mbox{$rX := \{x \in X : rx\}$} to denote the scaling of ratio $r$ of this set. We denote by $\ball$ (respectively $\sphere$) the ball (respectively sphere) of unit radius centered at the origin.  We denote the ellipsoid described by the matrix $P \in \mathcal{S}^n_{++}$ as $E_P$, i.e., $E_P:= \{x \in \mathbb{R}^n: x^T P x = 1\}$. Finally, we denote the spherical projector on $\sphere$ by $\proj_{\sphere} := x/\Vert x\Vert$. 
%For a distance $\dist$ on $\mathbb{R}^n$, the distance between a set $X \subset \mathbb{R}^n$ and a point $p \in \mathbb{R}^n$ is given by $\dist(X, p):=\inf_{x \in X}\dist(x,p)$. Note that the map $p \mapsto \dist(X,p)$ is continuous on $\mathbb{R}^n$. Given a set $X \subset \mathbb{R}^n$, $\partial X$ denotes the boundary of set $X$.
 
We consider in this work the classical uniform spherical measure on $\sphere$, denoted by $\sigma^{n-1}$, and derived from the Lebesgue measure $\lambda$ (see the appendix \textcolor{red}{[technical report]} for precise definitions). For $m \in \mathbb{N}_{>0}$, we denote by $M$ the set $M=\{1,2, \dots,m \}$ and we provide it with the uniform measure $\mu_M$. For any $l \in \mathbb{N}_{>0}$, we denote by $M^l$ the $l$-Cartesian product of $M$: $M^l = \Pi_1^l M$, and similarly we provide it with the uniform measure $\mu_{M^l} = \otimes^l \mu_M$. We can then define $Z_l = \sphere \times M^l$ as the Cartesian product of the unit sphere $\sphere$ and $M^l$. On the set $Z_l$, we define the product measure $\mu_l = \sigma^{n-1} \otimes \mu_{M^l}$. Note that, $\mu_l$ is a uniform measure on $Z_l$ and has total mass $1$. Finally, we denote by $\pi_\sphere$ the classical projection from $Z_l$ to $\sphere$.
