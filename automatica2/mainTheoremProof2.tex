% !TEX root = main.tex
\begin{subsection}{Main Theorem}
We are now ready to prove our main theorem by putting together all the above pieces. For a given level of confidence $\beta$, we prove that the upper bound $\gamma^{*}(\omega_N)$, which is valid solely on finitely many observations, is in fact a true upper bound, at the price of increasing it by the factor $\frac{1}{\sqrt[l]{\delta(\beta, \omega_N)}}$. Moreover, as expected, this factor gets smaller as we increase $N$ and decrease $\beta$.


\begin{thm}\label{thm:mainTheorem}
Consider an $n$-dimensional switched linear system as in \eqref{eq:switchedSystem} and a uniform random sampling $\omega_N \subset Z_l$, where $N \geq \frac{n(n+1)}{2}+1$. Let $\gamma^{*}(\omega_N) $ be the optimal solution to \eqref{eqn:campiOpt03}. Then, for any given $\beta \in [0,1)$ and $\eta > 0$, we can compute $\delta(\beta, \omega_N)$, such that with probability at least $\beta$ we have: $$\rho \leq \frac{\gamma^{*}(\omega_N) (1+ \eta)}{\sqrt[l]{\delta(\beta, \omega_N)}},$$ where $\lim_{N \to \infty}\delta(\beta, \omega_N) = 1$ with probability $1$.
\end{thm}
\begin{pf*}{Proof.}
By definition of $\gamma^{*}(\omega_N)$ we have
\begin{equation*}
(\mathbf{A}_l x)^T P \mathbf{A}_l x \leq (\gamma^{*}(1+\eta))^{2l} x^T P x,\, \forall (x, \mathbf{j}) \in \omega_N.
\end{equation*}
for some $P \succ 0$. Then, by taking $\beta = 1- I(1-\varepsilon; N-d, d+1)$ in Theorem \ref{mainTheorem0}, with probability at least $\beta$ the following holds:
\begin{equation*} 
\begin{aligned}
& (\mathbf{A}_l x)^T P \mathbf{A}_l x \leq  \left((\gamma^{*}(1+\eta) \right)^{2l} x^T P x, \\
& \qquad \qquad \qquad \qquad \qquad \qquad \forall\ (x,\mathbf{j}) \in Z_l \setminus V(\omega_N).
\end{aligned}
\end{equation*}
with $\mu_l(V(\omega_N)) \leq \varepsilon$ and $\varepsilon(\beta, N)=1- I^{-1}(1-\beta; N-d, d+1)$. Thanks to Corollary~\ref{cor:gettingRidOfm}, we even have
\begin{equation*} 
(\mathbf{A}_l x)^T P \mathbf{A}_l x \leq  \left( (\gamma^{*}(1+\eta) \right)^{2l} x^T P x, \forall x \in \sphere \setminus \tilde{\sphere}, \forall \mathbf{j} \in M^l
\end{equation*}
with $\tilde{\sphere} = \pi_{\sphere}(V)$ and $\sigma^{n-1}(\tilde{\sphere}) \leq \varepsilon m^l$. Then by Theorem \ref{thm:mainTheorem01}, we can compute $\delta(\beta, \omega_N) =\delta(\varepsilon'(\beta,N))$, where
\begin{equation}\label{eqn:eps2}
\varepsilon'(\beta, N) = \frac{1}{2} \varepsilon(\beta,N) m^l \kappa(P) 
\end{equation} 
such that with probability at least $\beta$ we have: $$\rho \leq \frac{\gamma^{*}(\omega_N) (1 + \eta)}{\sqrt[l]{\delta(\beta, \omega_N)}},$$ which completes the proof of the first part of the theorem. Note that, the ratio $\frac{1}{2}$ introduced in the expression of $\varepsilon'$ is, as we have already mentioned, due to the homogeneity of the system.


%this implies that with probability at least $\beta$ the following also holds:
%\begin{equation}
%(\mathbf{A}_l x)^T \mathbf{A}_l x \leq ( \gamma^{*}(1+\eta) )^{2l} x^T x, \, \forall x \in \sphere \setminus \tilde{\sphere}, \forall \mathbf{j} \in M^l,
%\end{equation}
%for some $\tilde{\sphere}$ where $\sigma^{n-1}(\tilde{\sphere}) \leq \varepsilon m^l \kappa(P)$. Then, applying Lemma~\ref{lemma:epsilon1}, we can compute $\delta(\beta, \omega_N) =\delta(\varepsilon'(\beta,N))$, where
%\begin{equation}\label{eqn:eps2}
%\varepsilon'(\beta, N) = \frac{1}{2} \varepsilon(\beta,N) m^l \kappa(P) 
%\end{equation} 
%such that with probability at least $\beta$ we have:
%\begin{equation*}
%\mathbf{A}_l (\ball) \subset \frac{(\gamma^{*}(\omega_N)(1+\eta))^l}{\delta(\beta, \omega_N)}\ball, \, \forall \mathbf{j} \in M^l
%\end{equation*}
%
%By Property \ref{rem:scaling}, this means that with probability at least $\beta$:
%$$\rho \leq \frac{\gamma^{*}(\omega_N) (1 + \eta)}{\sqrt[l]{\delta(\beta, \omega_N)}},$$
%which completes the proof of the first part of the theorem. Note that, the ratio $\frac{1}{2}$ introduced in the expression of $\varepsilon'$ is, as we have already mentioned Section~\ref{sec:pbsphere}, due to the homogeneity of the system.
%
%%
%%$$\frac{\calM}{\gamma^*} \conv(E_P \setminus E') \subset \convhull(E_P \setminus E'),$$
%%for some $E' \subset E_P$, where $\sigma_P(E') \leq \epsilon$.
%%Then, due to Theorem \ref{thm:mainTheorem0} we also have \eqref{eqn:contraction}, meaning:
%%$$\left(\frac{\cM}{\delta \gamma^*}\right) \convhull (E_P \setminus X_P) \subset \convhull (E_P\setminus X_P).$$
%%Then, $\delta\gamma^*$ is an upper bound on $\rho,$ with probability at least~$\beta$. 
%

Let us prove now that $\lim_{N \to \infty} \delta(\beta, \omega_N) = 1$ with probability $1$. We recall that, $\delta(\beta, \omega_N) = \delta \left( \varepsilon(\beta, \omega_N) m^l \kappa(P(\omega_N)) \right)$. We start by showing that $\kappa \left( P(\omega_N) \right)$  is uniformly bounded in $N$. The optimization problem $\Opt(\omega_N)$ given in \eqref{eqn:campiOpt03}, with $\gamma^{*}(\omega_N)$ replaced by $\gamma^{*}(Z_l)(1+\frac{\eta}{2})$ is strictly feasible, and thus admits a finite optimal value $K$ for some solution $P_{\eta/2}$. Note that, $\lim_{N \to \infty} \gamma^{*}(\omega_N)= \gamma^{*}(Z_l)$ with probability $1$. Thus, for large enough $N$, \mbox{$\gamma^{*}(\omega_N)(1+\eta) > \gamma^{*}(Z_l)(1+\frac{\eta}{2})$.} This also means that, for large enough $N$, $\Opt(\omega_N)$ admits $P_{\eta/2}$ as a feasible solution and thus the optimal value of $\Opt(\omega_N)$ is bounded by $K.$ In other words, \mbox{$\lambda_{\max}(P({\omega_N})) \leq K$.} Moreover, since  
$\lambda_{\max}(P(\omega_N))\geq 1$, we also have \mbox{$\det(P(\omega_N)) \geq 1$,} which means that
\begin{equation}\label{kappa}
\kappa \left( P(\omega_N) \right) = \sqrt{\frac{\lambda_{\max}(P(\omega_N))^n}{\det(P(\omega_N))}} \leq \sqrt{K^n}
\end{equation}
We next show that for a fixed $\beta \in [0,1)$, $\lim_{N \to \infty} \varepsilon(\beta, N) = 0$. Note that, $\varepsilon(\beta, N)$ is intrinsically defined by \\
$1-\beta = \sum_{j=0}^d {{N}\choose{j}} \varepsilon^j (1-\varepsilon)^{N-j}$. We can then upper bound the term $1-\beta$ as in:
\begin{equation}\label{eqn:beta}
1-\beta \leq  (d+1)N^d (1-\varepsilon)^{N-d}.
\end{equation}
We prove $\lim_{N \to \infty} \varepsilon(\beta, N) = 0$ by contradiction. Assume that $\lim_{N \to \infty} \varepsilon(\beta, N) \not= 0$. This means that, there exists some $c > 0$ such that $\varepsilon(\beta, N) > c$ infinitely often. Then, consider the subsequence $N_k$ such that $\forall k$, $\varepsilon(\beta, N_k) > c$. Then, by \eqref{eqn:beta} we have for any $k \in \mathbb{N}$:
\begin{equation*}
1-\beta \leq  (d+1)N_k^d (1-\varepsilon)^{N_k-d}\hspace{-0.4mm} \leq \hspace{-0.7mm}(d+1)N_k^d (1-c)^{N_k-d}. 
\end{equation*}
Note that $\lim_{k \to +\infty}(d+1)N_k^d (1-c)^{N_k-d} = 0$, which implies that there exists a $k'$ such that:
$$(d+1)N_{k'}^d (1-c)^{N_k'-d} < 1 - \beta,$$ which is a contradiction. Therefore, we must have  $\lim_{N \to \infty} \varepsilon (\beta, N) = 0$. Putting this together with \eqref{kappa}, we get: $\lim_{N \to \infty} m^l \kappa(P(\omega_N)) \varepsilon(\beta, \omega_N) = 0$. By the continuity of the function $I^{-1}$ this also implies: $\lim_{N \to \infty} \delta \left( \varepsilon(\beta, \omega_N) m^l \kappa(P(\omega_N)) \right) = 1.$
\end{pf*}
\end{subsection}