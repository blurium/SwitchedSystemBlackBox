 % !TEX root = main.tex
A switched linear system is of the form:
\begin{equation}\label{eq:switchedSystem}x_{k+1} = A_{\sigma(k)}x_k,\end{equation}
with switching sequence $\sigma: \N \to M$, where  $M := \{1,2, \dots,m\}$ and $A_{\sigma(k)} \in \calM$, for all $\sigma$ and $k \in M$. There are two important properties of linear switched systems that we exploit in this paper.
\begin{property}\label{property:homogeneity}
Let $\xi(x, k)$ denote the state of the system \eqref{eq:switchedSystem} at time $k$ starting from the initial condition $x$. The dynamical system \eqref{eq:switchedSystem} is homogeneous:
$$\xi(\gamma x, k)= \gamma \xi(x, k). $$
\end{property}
\begin{property}\label{property:convpres}
The dynamics given in \eqref{eq:switchedSystem} is convexity-preserving, meaning that for any set of points $X \subset \R^n$ we have:
$$ f(\conv{X})\subset \conv\{f(X)\}. $$
\end{property}

We now introduce the \emph{Lyapunov exponent} of the system, which is a numerical quantity describing its stability.
\begin{definition}Given a dynamical system as in \eqref{eq:dynamicalsystemGeneral}, its \emph{Lyapunov exponent} is given by
$$\rho =\inf{\{r:\,\forall\, x_0, \exists\, C\in \re^+: \xi(x_0, k)\leq Cr^k\}}. $$
\end{definition}
%Under certain conditions, deciding stability amounts to decide whether $\rho<1.$  In order to understand the quality of our techniques, we will actually try to prove lower and upper bounds on $\rho.$ 
\textcolor{red}{@Raphael: We need  a theorem about the stability and $\rho<1$ here.}
%Under certain conditions deciding stability amounts to decide whether $\rho<1.$ 

In the case of switched linear systems, the Lyapunov exponent is known as the joint spectral radius of the set of matrices, which can be alternatively defined as follows:
\begin{definition} \cite{jungers_lncis} Given a set of matrices $\cM \subset \re^{n\times n},$ its \emph{joint spectral radius} (JSR) is given by
$$\rho(\cM) =\lim_{k\rightarrow \infty} \max_{i_1,\dots, i_k}\{||A_{i_1} \dots A_{i_k}||^{1/k}: A_{i_j}\in\cM\}. $$
\end{definition}

\begin{remark}\label{rem:scaling}
Note that, JSR is a homogeneous function:
$$\rho\left(\frac{\cM}{\gamma}\right)=\frac{\rho(\cM)}{\gamma}, \forall \gamma > 0.$$
\end{remark}

