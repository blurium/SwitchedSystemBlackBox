% !TEX root = main.tex

%\textcolor{red}{Here, we need to talk about proving the Lyapunov decrement in finitely, many points, talk about campi and then make a transition}

%In this section we suppose that we do not observe the modes, but only the pairs $(x_k,y_k).$
%The idea is that if we have an ellipsoid that is contracted for all our observed pairs $(x,y),$  it is not enough to imply stability, because we observed only some possible behaviours of the dynamics.  However, suppose that we observed that the $\lambda-$contraction of the polytope $\cP$ is satisfied for many values of $x,$ then, it is tempting to extrapolate our observation, and claim that (noting $y=f(x) $ as always),
Before proceeding to prove Theorem \ref{thm:mainTheorem0} we introduce some further notation that will help us along the way. Let us consider $Z = \sphere \times M$ the Cartesian product of the unit sphere $\sphere$ with $M$ and. Every element of $Z$ can be written as $z = (x_z, k_z)$ with $x_z \in \sphere$ and $k_z \in M$. For notational simplicity, we drop the subscript $z$ whenever it is clear from the context. We define the classical projections of $Z$ on the sphere and $M$ by $\pi_{\sphere}: Z \to \sphere$ and $\pi_M: Z \to M$.

%The first statement in the Theorem \ref{thm:mainTheorem0} is equation \eqref{eqn:exceptEps}, which states that $\gamma^*$ and $P$ satisfying:
%\begin{equation*} (A_ix_i)^TP(A_ix_i)\leq \gamma^*x_i^TPx_i, \forall (x_i, A_i) \in \omega_N\end{equation*}
%also satisfies \eqref{eqn:exceptEps}. In other words, $\gamma^*$ and $P$ that are found from finitely many observations generalizes to the rest of the observations except a set of measure $\epsilon$. We now formalize.
To obtain, an upper bound on $\rho$ the optimization problem we would like to solve is:
\begin{equation}\label{eqn:campiOpt0}
\begin{aligned}
& \text{min} & & \gamma \\
& \text{s.t.} 
&  & (Ax)^TP(Ax) \leq \gamma x^TPx,\,\forall A \in \calM, \,\forall\, x \in \sphere,\\
& && P \succ 0. \\
\end{aligned}
\end{equation}
Note that if $P$ is a solution of \eqref{eqn:campiOpt0}, due to Property \ref{property:homogeneity}, then so is $\alpha P$ for any $\alpha \in \R_{> 0}.$ Therefore, the constraint $P \succ 0$ can be replaced with the constraint $P \succeq I$. Also note that, if the optimal solution $\gamma^*$ satisfies $\rho \leq \gamma^*$. However, solving \eqref{eqn:campiOpt0} is hard since it involves infinitely many constraints. Therefore, for a given sampling $\omega_N$, we instead consider the optimization problem \eqref{eqn:campiOpt01} with finitely many constraints sampled from $\sphere \times M$: 

Let $\gamma^*(\omega_N)$ be the optimal solution of \eqref{eqn:campiOpt01}. We now analyze the relationship between the solutions of the optimization problem \eqref{eqn:campiOpt0} and:
\begin{equation}\label{eqn:campiOpt02}
\begin{aligned}
& \text{find} & & P \\
& \text{s.t.} 
&  & (A_k x)^TP(A_k x) \leq \gamma^* x^TPx,\,\forall (x,k) \in \omega_N \\
& && P \succ 0. \\
\end{aligned}
\end{equation}
We can rewrite \eqref{eqn:campiOpt02} in the following form:
\begin{equation}
\label{eqn:campiOpt2}
\begin{aligned}
& \text{find} & & P \\
& \text{s.t.} 
& & f_{\gamma^*}(P, z) \leq 0,\,\forall\ z \in \sphere \times M\end{aligned}
\end{equation}
where $f_{\gamma^*}(P,z) = \max(f_1(P, z), f_2(P))$, and 
\begin{eqnarray*}
f_1(P, z) &:=& (A_k x)^TP(A_k x) - {\gamma^*}^2 x^TPx \\
f_2(P) &:=& \lambda_{\max}(-P) +1,
\end{eqnarray*}
We denote by  $\Opt(\omega_N)$ the optimization problem \eqref{eqn:campiOpt01} for the rest of the paper. Let $P(\omega_N)$ be the solution of $\Opt(\omega_N)$. We are interested in the probability of $P(\omega_N)$ violating at least one constraint in the original problem \eqref{eqn:campiOpt0}. Therefore, we define constraint violation property next.

\begin{definition}(from \cite{campi}) The \emph{constraint violation probability} is defined as:
\begin{equation}\label{eqn:violation}
\calV^*(\omega_N)=
      \mathbb{P}\{x \in X: f(P(\omega_N), x) > 0\},\, \forall\, \omega_N \in X^{N*}\}.
\end{equation}
where $$X^{N*}:=\{\omega_N \in X^N: \text{the solution of $\Opt(\omega_N)$ exists}\}.$$ 
Due to the definition of $\gamma^*$, we know that a solution to $\Opt(\omega_N)$ exists and therefore $X^{N*} = X^N$. Also note that, since we have $\mathbb{P}(A) = \frac{\mu(A)}{\mu(X)}$, we can rewrite \eqref{eqn:violation} as:
\begin{equation*}
\calV^*(\omega_N)=
      \frac{\mu\{V\}}{\mu(X)},\, \forall\, \omega_N \in X^N,
\end{equation*}
where $V:=\{x \in X: f(P(\omega_N),x) > 0\}$, i.e., the set of points for which at least one constraint is violated.
\end{definition}

The following theorem from \cite{campi} explicitly gives a relationship between $\calV^*(\omega_N)$, $N$, and $n$.
\begin{theorem}[from \cite{campi}]\label{thm:campi}Consider the optimization problem $\Opt(\omega_N)$ given in \eqref{eqn:campiOpt2}. \textcolor{red}{How to talk about assumptions of this theorem here?} Then, for all $\epsilon \in (0,1)$ the following holds:
\begin{equation*}\mathbb{P}^N\{\{\calV^*(\omega) \leq \epsilon\} \} \geq 1- \sum_{j=0}^{d} \binom{N}{j}\epsilon^j (1-\epsilon)^{N-j}.\end{equation*}
\end{theorem}
Note that $\epsilon=1-I^{-1}(\beta, N-d, d+1)$ and can be interpreted as the ratio of the measure of points in $X$ that might violate at least one of the constraints in \eqref{eqn:campiOpt01} to the measure of all points in $X$.

%We now state our main theorem which gives a probabilistic upper bound on JSR, and devote the next section to proving it step by step.
%
%\begin{theorem}[Main Theorem] \label{thm:mainTheorem} For any $\eta > 0$, given $N \geq n+1$ and $\beta \in [0,1)$, we can compute $\delta(N) < \infty$ such that with probability at least $\beta$, $\rho \leq \frac{\gamma^*}{\delta}$. Moreover, $\lim_{N \to \infty} \delta(N) = 1$.
%\end{theorem}






%
%Note that, if such $P$ exists the optimal solution$\gamma^*$ satisfies $\rho \leq \gamma^*$. Also note that if $P$ is a solution to \eqref{eqn:campiOpt0}, due to Property \ref{property:homogeneity}, then so is $\alpha P$ for any $\alpha \in \R_{> 0}.$ Therefore, we the constraint $P \succ 0$ can be replaced with the constraint $P \succeq I$ and we can rewrite \eqref{eqn:campiOpt0} as in:
%\begin{equation}
%\label{eqn:campiOpt2}
%\begin{aligned}
%& \text{find} & & P \\
%& \text{s.t.} 
%& & f_\gamma(P, x) \leq 0,\,\forall A_i \in \calM, \,\forall\, s \in \sphere, \end{aligned}
%\end{equation}
%where $f_\gamma(P,x) = \max(f_1(P, x), f_2(P))$, and 
%\begin{eqnarray*}
%f_1(P, x) &:=& (A_ks)^TP(A_ks) - {\gamma^*}^2 s^TPs \\
%f_2(P) &:=& \lambda_{\max}(-P) +1.
%\end{eqnarray*}
%

%Note that if $P$ is a solution to \eqref{eq:lowerbound}, due to Property \ref{property:homogeneity}, then so is $\alpha P$ for any $\alpha \in \R_{> 0}.$ Therefore, we can rewrite the constraint $P \prec 0$ in \eqref{eq:lowerbound} as the following optimization problem:
%\begin{equation}
%\label{eqn:campiOpt1}
%\begin{aligned}
%& \min & & \gamma  \\
%& \text{s.t.} 
%&  & (A_ks)^TP(A_ks) \leq \gamma^2 s^TPs,\, \forall A_i \in \calM, \,\forall\, s \in \sphere,\\
%& & & P \succeq I.
%\end{aligned}
%\end{equation}


%
%
%
%We are interested in solving the following optimization problem for a given $\gamma \in (0,1)$:
%\begin{equation}
%\label{eqn:campiOpt0}
%\begin{aligned}
%& \text{find} & & P \\
%& \text{subject to} 
%&  & (A_is)^TP(A_is) \leq \gamma^2 s^TPs,\,\forall A_i \in \calM, \,\forall\, s \in \sphere,\\
%& && P \succ 0. \\
%\end{aligned}
%\end{equation}
%Note that, if such a $P$ exists, $\rho \leq \lambda$.


%\begin{equation}
%\label{eqn:campiOpt1}
%\begin{aligned}
%& \text{find} & & P \\
%& \text{subject to} 
%&  & (A_ks)^TP(A_ks) \leq \gamma^2 s^TPs,\, \forall A_i \in \calM, \,\forall\, s \in \sphere,\\
%& & & P \succeq I.
%\end{aligned}
%\end{equation}

%We define the linear isomorphism $\Phi$ as the natural mapping \mbox{$\Phi: \mathbb{R}^{\frac{n(n+1)}{2}} \to \mathbb{S}^n$.} Using this mapping, for a fixed $\gamma \in (0, 1]$ we can rewrite \eqref{eqn:campiOpt01} as the following convex optimization problem:
%
%\begin{equation}
%\label{eqn:campiOpt1}
%\begin{aligned}
%& \text{find} & & p \\
%& \text{subject to} 
%& & f(p,x) \leq 0, \forall x \in X.
%\end{aligned}
%\end{equation}
%where $f(p,x) = \max(f_1(p, x), f_2(p))$, and 
%\begin{eqnarray*}
%f_1(p,x) &:=& (A_ks)^T\Phi(p)(A_ks) - \gamma^2 s^T\Phi(p)s \\
%f_2(p) &:=& \lambda_{\max}(\Phi(-p)) +1.
%\end{eqnarray*}

%\textcolor{red}{We have to change the objective function!}
%$f(p, \xi)$ is defined as in:
%\begin{equation}\label{eqn:objFunction}f_\gamma(p, x) = \max\left((A_is)^T\Phi(p)(A_is) - \gamma x^T\Phi(p)x,\lambda_{\max}(\Phi(p)) - t, \lambda_{\max}(\Phi(-p)) +1\right).\end{equation}

%The optimization problem \eqref{eqn:campiOpt1} is convex in $p$.

%\begin{proof}The function $f_1(p,x)$ is clearly convex in $p$ for a fixed $x \in X$. The function 
%$\lambda_{max}: \mathbb{S}^n \to \R$ maps a symmetric positive matrix to its maximum eigenvalue. It is well-known that the function $\lambda_{\max}$ is a convex function of $P$. \cite{boyd}. This means that, $p \mapsto \Phi(\lambda_{\max}(p))$ is convex in $p$. Moreover, maximum of convex functions is also convex, which shows that $f(p, x)$ is convex in $p$.
%\end{proof}
%The optimization problem \eqref{eqn:campiOpt0} has infinitely many constraints. We next consider the following optimization problem where we sample $N$ constraints of \eqref{eqn:campiOpt0} independently and identically with the probability measure
%$\mathbb{P}(A) = \frac{\mu(A)}{\mu(X)}, \forall\, A \in \Sigma$, where $N \geq \frac{n(n+1)}{2}$, and $d := \frac{n(n+1)}{2}$. We denote this sampling by $\omega_N=\{x_1, x_2, \ldots, x_N\} \subset X$, and obtain the following convex optimization problem for a given $\gamma \in R$ $\Opt(\omega_N)$: 
%
%\begin{equation}
%\label{eqn:campiOpt2}
%\begin{aligned}
%& \text{find} & & P \\
%& \text{s.t.} 
%& & f_\gamma(P, x) \leq 0, \forall x \in \omega_N, \end{aligned}
%\end{equation}
%where $f_\gamma(P,x) = \max(f_1(P, x), f_2(P))$, and 
%\begin{eqnarray*}
%f_1(P, x) &:=& (A_ks)^TP(A_ks) - \gamma^2 s^TPs \\
%f_2(P) &:=& \lambda_{\max}(-P) +1.
%\end{eqnarray*}

%Let $P^*(\omega_N)$ be the solution of $\Opt(\omega_N)$. We are interested in the probability of $P^*(\omega_N)$ violating at least one constraint in the original problem \eqref{eqn:campiOpt0}. Therefore, we define constraint violation property next.



%We make the following assumptions on the problem $\Opt(\omega)$:
%\begin{enumerate}
%\item \label{assumption1} Uniqueness of solution: Note that this can be enforced by adding a tie-break rule of at most $\frac{n(n-1)}{2}$ convex conditions discriminating our solutions. 
%\item \label{assumption2} Nondegenaracy: with probability $1$, there is no redundancy in the constraint obtained from the sampling. 
%We can notice that a constraint obtained for a point $(s, k)$ gives us a constraint also on the point $(-s, k)$, and that a set of constraints containing some redundancy in them will be a set of constraints obtained from samples for a same mode. Now, let assume we consider for a fixed mode of index $k$ a nondegenerate sample of points $\{x_1, \dots, x_i\}$. Then if $x_{i+1} \notin \{(\pm s_{x_j},k), \ j \in \{1, \dots, i\} \}$ (which is a set of measure zero), the constraint obtained from $x_{i+1}$ is not redundant. Indeed, the mode of index $k$ might have all its eigenvalues but one with norm lower than $1$, and $x_{i+1}$ might be in the eigenspace of the latter eigenvalue (thus the constraint given by $x_{i+1}$ will be violated). Then, by considering the shortest distance between $x_{i+1}$ and the other points of the sample, we can choose for this eigenvalue a value close enough to $1$ (while being strictly greater than $1$) such that the constraints relative to the other points will still be satisfied.
%\end{enumerate}
