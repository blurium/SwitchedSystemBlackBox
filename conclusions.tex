% !TEX root =main.tex
In this paper, we investigated the question of how one can conclude stability of a dynamical system when a model is not available and, instead, we have state measurements. Our goal, motivated by both practical and theoretical considerations, was to avoid combining identification of the system with classical stability analysis techniques. Indeed, in real-world applications, often the true equations describing the dynamics might be extremely complex or nonstandard, or might even be not well-defined. Moreover, identification of such complex systems is typically hard, and it is not clear how observation or computation errors would propagate in such a two-step strategy. 

For these reasons, we aim at understanding how the observation of well-behaved trajectories \emph{intrinsically} implies stability of a system. It is not surprising that we need some standing assumptions on the system, in order to allow for any sort of nontrivial certificate, solely from a finite number of observations. The novelty of our contribution is twofold: 
First, we use as such a standing assumption that the system is a switching linear system. This assumption covers a wide range of systems of interest nowadays, and to our knowledge no ``black-box'' result was available so far on switched systems.  
Second, we apply powerful techniques from chance constrained optimization. The application is not obvious, and relies on geometric properties of linear switching systems. We believe that this guarantee is quite powerful, in view of the hardness of the general problem, and in the future, we plan to investigate how to generalize it to more complex or realistic systems. We are also improving the numerical properties of our technique by incorporating Sum-Of-Squares optimization, and relaxing the sampling assumptions on the observations.
