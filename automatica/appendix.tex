% !TEX root = main.tex
\begin{section}{Precisions on Measures Definitions}\label{app:measures}

For an ellipsoid centered at the origin, and for any of its subsets $\calA$, the \emph{sector} defined by $\calA$ is the subset $$\{t \calA, \ t \in [0,1]\} \subset\ \R^n.$$ We denote by $E_P^{\calA}$ the sector induced by $\calA \subset E_P$. In the particular case of the unit sphere, we instead write $\sphere^{\calA}$. We can notice that $E_P^{E_P}$ is the volume in $\R^n$ defined by $E_P$: $E_P^{E_P} = \{x \in \R^n: x^TPx \leq 1\}$.

The spherical Borelian $\sigma$-algebra, denoted by $\mathcal{B}_{\sphere}$, is defined by $$\calA \in \mathcal{B}_{\sphere} \iff \sphere^{\calA} \in \mathcal{B}_{\mathbb{R}^n}.$$ We provide $(\sphere,\mathcal{B}_{\sphere})$ with the classical, unsigned and finite uniform spherical measure $\sigma^{n-1}$ defined by
$$\forall\ \calA \in \mathcal{B}_{\sphere},\, \sigma(\calA) = \frac{\lambda(\sphere^{\calA})}{\lambda(\ball)}. $$
In other words, the spherical measure of a subset of the sphere is related to the Lebesgue measure of the sector of the unit ball it induces. Notice that $\sigma^{n-1}(\sphere) = 1$.

Since $P \in \calS_{++}^n$, we recall that it can be written in its Cholesky form \eqref{cholesky}. Note that, $L^{-1}$ maps the elements of $\sphere$ to $E_P$. Then, we define the measure on the ellipsoid $\sigma_P$ on the $\sigma$-algebra $\mathcal{B}_{E_P}:=L^{-1}\mathcal{B}_\sphere$, where $\forall\, \calA \in \mathcal{B}_{E_P},\, \sigma_P({\calA}) = \sigma^{n-1}(L\calA)$. 

Set $M$ is provided with the classical $\sigma$-algebra associated to the finite sets: $\Sigma_M = \wp(M)$, where $\wp(M)$ is the set of subsets of $M$. We provide $(M, \Sigma_M)$ with the uniform measure $\mu_M$. Similarly, we define $\Sigma_{M^l}$ as the product $\bigotimes^l \Sigma_M$ (which is here equal to $\wp(M)^l$), and we provide $(M^l, \Sigma_{M^l})$ with the uniform product measure $\mu_{M^l} = \otimes^l \mu_M$.

We can now denote the product $\sigma$-algebra $\mathcal{B}_{\sphere} \bigotimes (\Sigma_{M^l})$ generated by $\mathcal{B}_{\sphere}$ and $\Sigma_{M^l}$: $\Sigma = \sigma( \pi_{\sphere}^{-1}(\mathcal{B}_{\sphere}),  \pi_{M^l}^{-1}(\Sigma_{M^l}))$, where $\pi_{\sphere}: Z_l \to \sphere$ and $\pi_{M^l}: Z_l \to M^l$ are the standard projections. On $(Z_l,\mathcal{B}_{\sphere} \bigotimes (\Sigma_{M^l}) )$, we define the product measure $\mu_l = \sigma^{n-1} \otimes \mu_{M^l}$. Note that, $\mu_l$ is a uniform measure on $Z_l$ and $\mu_l(Z_l)=1$.  

\end{section}


\begin{section}{Proof of Corollary~\ref{cor:gettingRidOfm}}\label{proof:cor_m}
We prove here Corollary~\ref{cor:gettingRidOfm} stated in section~\ref{sec:IntroMainThm}.

Let $\sphere' = \pi_{\sphere}(V)$. We know that $\Sigma_{M^l}$ is the disjoint union of its $2^{m^l}$ elements $\{\calM^l_i, i \in \{1,2, \dots, 2^{m^l} \} \}$. Then $V$ can be written as the disjoint union $V = \sqcup_{1 \leq i \leq 2^{m^l}} (\calS_i, \calM^l_i)$ where $\calS_i \in \Sigma_{\sphere}$. We notice that 
$\sphere' = \sqcup_{1 \leq i \leq 2^{m^l}} \calS_i$, 
and
\begin{equation*}
\sigma^{n-1} (\sphere') = \sum_{1 \leq i \leq 2^{m^l}} \sigma^{n-1} (\calS_i).
\end{equation*}
We have 
\begin{eqnarray*}
\mu_l(V) &=& \mu_l \left( \sqcup_{1 \leq i \leq 2^{m^l}} (\calS_i, \calM^l_i) \right) = \sum_{1 \leq i \leq 2^{m^l}} \mu_l(\calS_i, \calM^l_i) \\
 &=& \sum_{1 \leq i \leq 2^{m^l}} \sigma^{n-1} \otimes \mu_{M^l} (\calS_i, \calM^l_i) \\
 & = &\sum_{1 \leq i \leq 2^{m^l}} \sigma^{n-1}(\calS_i) \mu_{M^l} (\calM^l_i).
\end{eqnarray*}
Note that we have $\min_{(j_1,\dots,j_l) \in M^l} \mu_{M^l}(\{j_1,\dots,j_l\}) = \frac{1}{m^l}.$ Then since $ \forall \ i$, $\mu_{M^l}(\calM^l_i) \geq \min_{(j_1,\dots,j_l) \in M^l} \mu_{M^l}(\{j_1,\dots,j_l\}) = \frac{1}{m^l}$, we get:
\begin{equation}
\sigma^{n-1}(sphere') \leq \frac{\mu_l(V)}{\frac{1}{m^l}} \leq m^l \varepsilon.
\end{equation}
This means that 
\begin{equation}\label{eqn:P2}
(A_{j_l} A_{j_{l-1}} \dots A_{j_1} x)^T P (A_{j_l} A_{j_{l-1}} \dots A_{j_1} x) \leq \gamma^{2l} x^T P x,\,\, \forall\, x \in \sphere \setminus \sphere', \,\forall\, (j_1,\dots,j_l) \in M^l,
\end{equation}
where $\sigma^{n-1}(\sphere') \leq m^l \varepsilon.$

\end{section}


\begin{section}{Preliminary Results on Spherical Caps}\label{sec:app_prelim_caps}

Before proceeding to the proof of Proposition~\ref{lemma:propositionSphericalCaps}, we first introduce some necessary definitions and related background on spherical caps. We recall that a \emph{spherical cap} on $\sphere$ for a given hyperplane $c^Tx = k$ is defined by $\calC_{c,k} := \{x \in \sphere : c^Tx >k\}$.


\begin{remark}\label{lemma:muMonotone}
Consider the spherical caps $\calC_{c, k_1}$ and $\calC_{c, k_2}$ such that $k_1 > k_2$, then we have:
$$\sigma^{n-1}(\calC_{c,k_1}) < \sigma^{n-1}(\calC_{c,k_2}).$$
\end{remark}


\begin{remark}\label{prop:distance}
The distance between the point $x=0$ and the hyperplane $c^Tx = k$ is $\frac{|k|}{\|c\|}$.
\end{remark}

We also recall that we defined in section~\ref{sec:pbsphere} the function $\Delta: \wp(\sphere) \to [0,1]$ as $\Delta(X) := \sup \{r: r\ball \subset \conv(\sphere \setminus X)\}$.


\begin{lemma} \label{lemma:delta2} 
Consider the spherical cap $\calC_{c,k}$. We have:
$$\Delta(\calC_{c,k}) = \min\left(1, \frac{|k|}{\|c\|}\right).$$
\end{lemma}

\begin{proof}
Note that: $$\conv(\sphere \setminus X)= \left\{x \in \ball : c^Tx \leq k \right\}.$$
Then the following equalities hold:
\begin{eqnarray}
\nonumber \Delta(X) &=& \dist(\partial \conv(\sphere \setminus X), 0) \\
\nonumber &=& \min(\dist(\partial \ball, 0), \dist(\partial\{x : c^Tx \leq k\}, 0)) \\
\nonumber &=& \min(\dist(\sphere, 0), \dist(\{x : c^Tx = k\}, 0)) \\
\nonumber &=& \min\left(1, \frac{|k|}{\|c\|}\right).
\end{eqnarray}
\end{proof}

\begin{corollary}\label{lemma:deltaMonotone} 
Consider the spherical caps $\calC_{c, k_1}$ and $\calC_{c, k_2}$ such that $k_1 \leq k_2$. Then we have: $$\Delta(\calC_{c,k_1}) \leq \Delta(\calC_{c,k_2}).$$
\end{corollary}

\begin{lemma}\label{lemma:constructSC}
For any set $X \subset \sphere$, there exist $c$ and $k$ such that $\calC_{c,k}$ satisfies: $\calC_{c,k} \subset X,$ and $\Delta(\calC_{c,k}) = \Delta(X).$
\end{lemma}

\begin{proof} Let $\tilde{X} := \conv(S \setminus X)$.
Since $\dist$ is continuous and the set $\partial \tilde{X}$ is compact, there exists a point $x^* \in \partial \tilde{X}$, such that:
\begin{eqnarray}\nonumber \Delta(X) = \dist(\partial\tilde{X}, 0) = 
\label{deltaSupporting} \min_{x \in \partial \tilde{X}}\dist(x, 0) = \dist(x^*, 0).\end{eqnarray} 
Next, consider the hyperplane which is tangent to the ball $\Delta(X)\ball$ at $x^*$, which we denote by $\left\{x : c^Tx = k\right\}$.
%\begin{equation*} \partial (\Delta(X)\ball) \subset \tilde{X} \subset \left\{x : c^Tx \leq k\right\}.\end{equation*}  By Remark \ref{suppHyperplaneRemark}, this implies that $\left\{x : c^Tx = k\right\}$ is in fact the unique supporting hyperplane at $x^*$.
Then we have:
\begin{equation*}\Delta(X) =  \dist(x^*, 0) = \dist(\{x : c^Tx = k\}, 0) = \min\left(1, \frac{|k|}{\|c\|}\right).
\end{equation*}
Now, consider the spherical cap defined by this tangent plane i.e., $\calC_{c, k}$. Then, by Lemma \ref{lemma:delta2} we have
$\Delta(\calC_{c,k}) =  \min\left(1, \frac{|k|}{\|c\|}\right)$. Therefore, $\Delta(X) = \Delta(\calC_{c,k})$.


We next show $\calC_{c, k} \subset X$. We prove this by contradiction. Assume $x \in \calC_{c,k}$ and $x \notin X$. Note that, if $x \notin X$, then $x \in \sphere \setminus X \subset \conv(\sphere \setminus X).$ Since $x \in \calC_{c,k}$, we have $c^Tx>k$. But due to the fact that $x \in \conv(\sphere \setminus X)$, we also have $c^Tx \leq k$, which leads to a contradiction. Therefore, $\calC_{c, k} \subset X$. 
\end{proof}

We are now able to prove Proposition~\ref{thm:mainSphericalCap} given in section~\ref{sec:pbsphere}, which states that, for any $\epsilon \in (0,1)$, the function $\Delta(X)$ attains its minimum over $\calX_{\epsilon}$ for some $X$ which is a spherical cap.


\begin{proof}
We prove this via contradiction. Assume that there exists no spherical cap in $\calX_{\epsilon}$ such that $\Delta(X)$ attains its minimum. This means there exists an $X^* \in \calX_{\epsilon}$, where $X^*$ is not a spherical cap and $\argmin_{X \in \calX_{\epsilon}}(\Delta(X))=X^*$. By Lemma \ref{lemma:constructSC}, we can construct a spherical cap $\calC_{c,k}$ such that $\calC_{c,k} \subset X^*$ and $\calC_{c,k} = \Delta(X^*)$. Note that, we further have $\calC_{c,k} \subsetneq X^*$, since $X^*$ is assumed not to be a spherical cap. This means that, there exists a spherical cap $\sigma^{n-1}(\calC_{c,k})$ such that $\sigma^{n-1}(\calC_{c,k}) < \epsilon$. 

Then, the spherical cap $\calC_{c, \tilde{k}}$ with $\sigma^{n-1}(\calC_{c, \tilde{k}}) = \epsilon$, satisfies $\tilde{k} < k$ by Remark \ref{lemma:muMonotone}. This implies $$\Delta(\calC_{c, \tilde{k}}) < \Delta(\calC_{c, k}) = \Delta(X^*)$$ by Corollary \ref{lemma:deltaMonotone}. Therefore, $\Delta(\calC_{c, \tilde{k}}) < \Delta(X^*)$. This is a contradiction since we initially assumed that $\Delta(X)$ attains its minimum over $\calX_{\epsilon}$ at $X^*$.
\end{proof}
\end{section}

\begin{section}{Proof of Lemma~\ref{lemma:propositionSphericalCaps}}\label{sec:ProofSphericalCaps}

%\label{app:eps}
%\begin{lemma}\label{lemma:eps}Let $X_\epsilon \subset \sphere$ such that $\sigma^{n-1}(X_\epsilon) \leq \epsilon$ for some $\epsilon \in (0, 1)$, then the following holds:
%\begin{equation}\label{eqn:ballinc}\Delta(X_\epsilon) \geq \alpha(\epsilon)^2.\end{equation}
%where  \begin{equation}\label{defn:alpha}\alpha(\epsilon) = 1- I^{-1}\left(\frac{\epsilon\Gamma(\frac{n}{2})}{\pi^{n/2}}; \frac{n-1}{2}, \frac{1}{2}\right),\end{equation} and $\Gamma(x)=\int_{0}^{\infty} t^{x-1} e^{-t} \text{d}t.$ Here $I^{-1}$ is the inverse incomplete beta function, i.e.,  $I^{-1}(y; a,b)= x$ where $I(x;a,b)=y$.
%\end{lemma}



%\begin{lemma}\label{propositionSphericalCaps}
%Let $\epsilon \in (0, \frac{1}{2})$ and $\alpha: (0,1) \to \R_{\geq 0}$ be defined by:
%\begin{equation}
%\alpha(\epsilon) := \inf_{X \in \calX_\epsilon} \sup\{r: r\ball \subset \convhull(\sphere \setminus X)\},
%\end{equation}
%where $\calX_{\epsilon} = \{X \subset \sphere: \sigma^{n-1}(X) \leq \epsilon\}$. Then, $\alpha(\epsilon)$ is given by the formula:
%\begin{equation}
%\alpha(\epsilon) = \sqrt{1- I^{-1}\left(2\epsilon; \frac{n-1}{2}, \frac{1}{2}\right)}, \end{equation}
%where $I$ is the regularized incomplete beta function.
%\end{lemma}

We present now a proof of Lemma~\ref{lemma:propositionSphericalCaps}. By Proposition~\ref{thm:mainSphericalCap} we know that:
\begin{equation}\label{eqn:sc}\alpha(\epsilon) = \Delta(\calC_{c, k}),
\end{equation}
for some spherical cap $\calC_{c,k} \subset \sphere$, where  $\sigma^{n-1}(\calC_{c, k}) = \epsilon$. It is known (see e.g. \cite{sphericalCapRef}) that the area of such $\calC_{c, k}$, is given by the equation:
\begin{equation}\sigma^{n-1}(\calC_{c, k}) = \frac{I\left(1-\Delta(\calC_{c,k})^2; \frac{n-1}{2}, \frac{1}{2}\right)}{2}.
\end{equation}
Since, \mbox{$\sigma^{n-1}(\calC_{c, k})= \epsilon$,} we get the following set of equations:
\begin{eqnarray}\nonumber \epsilon &=& \frac{I\left(1- \Delta(\calC_{c,k})^2;\frac{n-1}{2}, \frac{1}{2}\right)}{2} \\
\nonumber 1- \Delta(\calC_{c, k})^2 &=&  I^{-1}\left(2\epsilon; \frac{n-1}{2}, \frac{1}{2}\right) \\
\label{eqn:last}\Delta(\calC_{c, k})^2 &=&  1- I^{-1}\left(2\epsilon; \frac{n-1}{2}, \frac{1}{2}\right).
\end{eqnarray}
Then, the inequalities \eqref{eqn:sc} and \eqref{eqn:last} imply the inclusion given in \eqref{eqn:ballinc}, which concludes the proof.

%\begin{equation}\label{eqn:alphaEpsilon}\alpha(\epsilon) = \sqrt{1- I^{-1}\left(\frac{\epsilon\Gamma(\frac{d}{2})}{\pi^{d/2}}; \frac{d-1}{2}, \frac{1}{2}\right)}.\end{equation}

\end{section}


%
% 
%
%\begin{lemma}\label{lemma:epsilon1}Let $\epsilon \in (0, 1)$. Consider a set of matrices $A \in \calM$, and an ellipsoid $E_P$ satisfying
%\begin{equation*}(A_jx)^TP(A_jx) \leq \gamma^* x^TPx, \quad \forall\, x \in E_P \setminus E', \forall\,j \in, M,\end{equation*}
%for some $E' \subset E_P$ where $\sigma^{n-1}(E') \leq \epsilon$, then we have:
%\begin{equation*}(\bar{A}_jLx)^T(\bar{A}_jLx) \leq \gamma^* x^TPx, \quad \forall\, x \in \sphere \setminus \sphere', \forall\,j \in, M,\end{equation*}
%where $\bar{A}:=L^{-1}AL$, for some $\sphere'  \subset \sphere$ where \mbox{$\sigma^{n-1}(\sphere') \leq \frac{1}{\lambda_{\min}(L)^n} \epsilon$}
%\end{lemma}
%
%
%Given the bound $\sigma^{n-1}(\sphere')\leq \frac{1}{\lambda_{\min}(L)^n} \epsilon$, we now proceed to the second part where we compute $a$
%\begin{equation*}\bar{A} \sphere \subset \alpha(\epsilon) \sphere, \forall\, j \in M,
%\end{equation*}
%where $\bar{A}:=L^{-1}AL$.
%
%
%
%
%
%%. We denote $\epsilon:=\frac{m\epsilon}{2} \sqrt{\frac{\lambda_{\max}(P)^n}{\det(P)}}$, where the additional factor $\frac{1}{2}$ follows from the homogeneity of the dynamics which implies a symmetry of $V_\sphere$, i.e., \begin{equation*}x \in V_{\sphere} \implies -x \in V_{\sphere}.\end{equation*}
%
%\begin{corollary}\label{cor:main} Let $\beta \in (0,1)$. We have: $$\delta(\beta, \omega_N) =  \Delta(\calC_{c, k}),$$ where
%$\sigma^{n-1}(\calC_{c,k}) = \epsilon$.
%\end{corollary}
%
%
%\begin{lemma}\textcolor{blue}{Consider the sequential sampling $\omega_N$.} Let $d=\frac{n(n+1)}{2}$ and $P(\omega_N)$ be the optimal solution to the optimization problem $\Opt(\omega_N)$ and let $\lambda_{\max}(P(\omega_N))$ be the optimal objective function value for this $P(\omega_N)$. Then, $\lambda_{\max}(P(\omega_N))$ is uniformly bounded in $N$.
%\end{lemma}
%
%\begin{proof}We first define the following optimization problem:
%\begin{equation}\label{eqn:appendix}
%\begin{aligned}
%& \text{min}_{P} & & \lambda_{\max}(P) \\
%& \text{s.t.} 
%&  & (A_j x)^TP(A_j x) \leq {(1 +\eta)\gamma}^2 x^TPx,\,\forall (x, j) \in \omega_N, \\
%& && P \succeq I, \\
%\end{aligned}
%\end{equation}
%where we denote its optimal solution by $\lambda_{\max}(\gamma , \omega_N)$.
%
%Note that, for all $d \in \Z$ such that $0< d \leq N$ we have $\gamma^*(\omega_d) \leq \gamma^*(\omega_N)$. Also note that,
%\begin{equation*}\lambda_{\max}(\gamma^*(\omega_N), \omega_N) \leq \lambda_{\max}(\gamma^*(\omega_d), \omega_N).
%\end{equation*}
%But note that, there exists a $c > 0$ such that $\lambda_{\max}(\gamma^*(\omega_d), \omega_N) <  c$ since the problem \eqref{eqn:appendix} is strictly feasible for any $\gamma$ such that $\gamma \leq \gamma^*$. This implies: $\lambda_{\max}(\gamma^*(\omega_d), \omega_N) \leq c$, which completes the proof of this lemma.
%\end{proof}
%
%
%%\begin{corollary}\label{cor:final}Let $\beta \in (0,1)$. $$\lim_{N \to \infty} \delta(\beta, \omega_N) = 1.$$
%%\end{corollary}
%
%\begin{proof} 
%We first prove that $\lim_{N \to \infty} \epsilon(\beta, N)= 0$. Note that, we can upper bound $1-\beta$ as follows:
%\begin{equation}\label{eqn:beta}1-\beta = \sum_{j=0}^d {{N}\choose{j}} \epsilon^j (1-\epsilon)^{N-j} \leq  (d+1)N^d (1-\epsilon)^{N-d}.
%\end{equation}
%We prove $\lim_{N \to \infty} \epsilon(\beta, N) = 0$ by contradiction. Assume that $\lim_{N \to \infty} \epsilon(\beta, N) \not= 0$. This means that, there exists some $\delta > 0$ such that $\epsilon(\beta, N) > \delta$ infinitely often. Then, consider the subsequence $N_k$ such that $\epsilon(\beta, N_k) > \delta$, $\forall\, k.$ Then, by \eqref{eqn:beta} we have:
%\begin{equation*}1-\beta\leq (d+1)N_k^d (1-\epsilon)^{N_k-d} \leq (d+1)N_k^d (1-\delta)^{N_k-d}\, \forall\,k \in \N. 
%\end{equation*}
%Note that $\lim_{k \to \infty}(d+1)N_k^d (1-\delta)^{N_k-d} = 0.$ Therefore, there exists a $k'$ such that, we have $$(d+1)N_{k'}^d (1-\delta)^{N_k'-d} < 1 - \beta,$$ which is a contradiction. Therefore, we must have  $\lim_{N \to \infty} \epsilon (\beta, N) = 0$.
%
%\textcolor{red}{Still to show: If $\frac{\lambda{\max}(P)}{\lambda{\min}(P)}$ is uniformly bounded in $N$, then
%since $I^{-1}$ is monotonic in its first parameter, $\delta = \sqrt{1-\alpha}$ tends to $1$ as $\epsilon \to 0$.}
%\end{proof}