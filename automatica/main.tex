\documentclass[twocolumn]{autart} 

\usepackage[T1]{fontenc}
\usepackage{amssymb}
\usepackage{mathtools}
\usepackage{color, comment}
\usepackage{graphicx}
\usepackage{textcomp}
\usepackage{gensymb,bbm}
\usepackage{stmaryrd}
\usepackage{cite}
\usepackage{tikz}
\usepackage{algorithmic}
\usepackage{algorithm}
\usetikzlibrary{calc}
\usepackage{float}


%\include{references.bib}
%   
%\renewcommand\paragraph{\@startsection{paragraph}{4}{\z@}%
%            {-2.5ex\@plus -1ex \@minus -.25ex}%
%            {1.25ex \@plus .25ex}%
%            {\normalfont\normalsize\bfseries}}
%\makeatother
%\setcounter{secnumdepth}{4} % how many sectioning levels to assign numbers to
%\setcounter{tocdepth}{4}    % how many sectioning levels to show in ToC
%
%\DeclarePairedDelimiter{\ceil}{\lceil}{\rceil}
%
\newcommand{\convhull}{\mbox{convhull } }
\newcommand{\Al}{\mathbf{A}_l }
\newcommand{\barAl}{\mathbf{\bar{A}}_l }
\newcommand{\conv}{\convhull }
\newcommand{\proj}{\Pi }
\newcommand{\koverc}{\frac{|k_i|}{\|c_i\|} }
\newcommand{\ball}{\mathbb{B}}
\newcommand{\dist}{d}
\providecommand{\rmj[1]}{{\color{red}#1}}
\providecommand{\ayca[1]}{{\color{blue}#1}}
%\providecommand{\com[2]}{\begin{tt}[#1: #2]\end{tt}}
%\providecommand{\comrj[1]}{\com{RJ}{\rmj{#1}}}
%\providecommand{\comff[1]}{{\small \color{blue} [FF: {#1}]}}
%
\DeclareMathOperator*{\argmin}{arg\,min}

\newcommand{\sphere}{\mathbb{S}}
\newcommand{\supp}{\text{supp}}
\newcommand{\Opt}{\text{Opt} }
%
%\renewcommand\labelitemi{{\boldmath$\cdot$}}
%
\floatname{algorithm}{Procedure:}
\renewcommand{\algorithmicrequire}{\textbf{Input:}}
\newcommand{\algorithmicblabla}{\textbf{Input 2:}}
\renewcommand{\algorithmicensure}{\textbf{Output:}}
\renewcommand{\algorithmicreturn}{\textbf{Compute:}}
\renewcommand{\thealgorithm}{} 

\newtheorem{property}{Property}[section]


%\thanks{\textcolor{red}{*This work was supported by.}}
%\author{Ayca Balkan$^{1}$ and Raphael Jungers$^{2}$ and Joris Kenanian$^{1}$ and Paulo Tabuada$^{1}$ 
%\thanks{$^{1}$A. Balkan, J.Kenanian, and P. Tabuada are with Dept. of Electrical Engineering at University of California, Los Angeles (UCLA). Their work is supported by NSF Grants 1645824, 1521617, and NSF Expeditions in Computer Augmented Program Engineering.}
%\thanks{$^{2}$R. Jungers is a Fulbright Fellow and a F.N.R.S. fellow currently visiting the Dept. of
%Electrical Engineering at UCLA. His work is supported by the French Community of Belgium and the IAP network Dysco.}}


\begin{document}
\begin{frontmatter}
%\runtitle{Insert a suggested running title} 
\title{Data Driven Stability Analysis of Black-box Switched Linear Systems}

\thanks[footnoteinfo]{A. Balkan, J. Kenanian, and P. Tabuada had their work supported by NSF Grants 1645824, 1521617, and NSF Expeditions in Computer Augmented Program Engineering. R. Jungers is a Fulbright Fellow and a F.N.R.S. Fellow currently visiting the Dept. of Electrical Engineering at UCLA. His work is supported by the French Community of Belgium and the IAP network Dysco.}

\author[UCLA]{Marcus Tullius Cicero}\ead{cicero@senate.ir},    % Add the 
\author[UCLA]{Julius Caesar}\ead{julius@caesar.ir},               % e-mail address 
\author[Louvain]{Publius Maro Vergilius}\ead{vergilius@culture.ir}  % (ead) as shown

\address[UCLA]{Department of Electrical Engineering at University of California, Los Angeles (UCLA)}  % Please supply                                              
\address[Louvain]{UC Louvain/Raphael's affiliation}             % full addresses



\begin{keyword}                           % Five to ten keywords,  
\textcolor{red}{5 to 10 keywords to pick from the list: http://www.autsubmit.com/documents/keywords.html}   % chosen from the IFAC 
\end{keyword}                             % keyword list or with the 
                                          % help of the Automatica 
                                          % keyword wizard

\begin{abstract}
We address the problem of deciding stability of a ``black-box'' dynamical system (i.e., a system whose model is not known) from a set of observations. The only assumption we make on the black-box system is that it can be described by a switched linear system. We show that, for a given (randomly generated) set of observations, one can give a stability guarantee, for some level of confidence, with a trade-off between the quality of the guarantee and the level of confidence. We provide an explicit way of computing the best stability guarantee, as a function of both the number of observations and the required level of confidence. Our results rely on geometrical analysis and combining chance-constrained optimization theory with stability analysis techniques for switched systems.
\end{abstract}
\end{frontmatter}

\section{Introduction}
% !TEX root = writeUpMainTheorems.tex
Stability analysis of dynamical systems is a challenging task. The existing methods to construct stability certificates such as Lyapunov functions require the knowledge of a model of the dynamical system of interest. \cite{lyapunov} However, in industrial applications, typically there is no explicit model of the system in the form of difference or differential equations. On the other hand for a wide variety of scenarios, simulating dynamical systems with different model parameters and initial conditions is usually possible, and computationally efficient. Therefore, it is valuable to be able to analyze the stability of a given dynamical system with an unknown model, directly from data. More formally, given a dynamical system as in:
\begin{equation}\label{eq:dynamicalsystemGeneral}x_{k+1} = f(k, x_k),
\end{equation}
where, $x_k \in \R^n$, k is index of time. Let $y_k := x_{k+1}$
We ask the following question: given N input-output pairs, $(x_1, y_1)$, $(x_2, y_2)$, $\ldots$, $(x_N, y_N)$ such that $y_{k} = f(k, x_k)$, what can we say about the stability of the system \eqref{eq:dynamicalsystemGeneral}? The answer is immediate when \eqref{eq:dynamicalsystemGeneral} is a linear time-invariant system, since we can simply identify the system by $n$ linearly independent output traces. In this paper, we seek the answer to this question for switched linear system for which immediately the problem immediately becomes nontrivial. Switched linear systems are in the form:
\begin{equation}\label{eq:switchedSystem}x_{k+1} = A_{\sigma(k)}x_k,\end{equation}
where, $\sigma: \N \to \{1,2, \ldots, m\}$ is the switching sequence and $A_{\sigma(k)} \in \calM$, for all $\sigma$ and $k$. Aside from out theoretical interest, switched systems model the behavior of dynamical systems in the presence of a known or unknown varying parameter. This parameter can model internal parameters such as model uncertainties, look-up tables as well as exogenous parameters such as inputs provided by a controller in a closed-loop control system. 

%To make our reasoning clearer, we introduce the \emph{Lyapunov exponent} of the system \eqref{eq:switchedSystem}, which is a numerical quantity describing its stability.
%\begin{definition} Given a dynamical system as in \eqref{eq:dynamicalsystem} its \emph{Lyapunov exponent} is given by
%$$\rho =\inf{\{r:\,\forall x_0, \exists C\in \re^+: \quad x(0)=x_0 \Rightarrow x(t)\leq Cr^t\}}. $$
%\end{definition}
%Under certain conditions, deciding stability amounts to decide whether $\rho<1.$  In order to understand the quality of our techniques, we will actually try to prove lower and upper bounds on $\rho.$ 
Assessing the stability of nonlinear systems by leveraging simulations has been an active area of research in the recent years. Topcu et.al. \cite{topcu} and Kapinski et.al. \cite{kapinski} construct Lyapunov function candidates using the simulation traces, however to be able to formally verify the constructed Lyapunov function, they require the knowledge of the full dynamics. In \cite{lazar} and \cite{lazar2} Bobiti and Lazar address this and provide sampling based probabilistic and deterministic guarantees of a given Lyapunov function candidate. The presented method requires the knowledge of how fast the output of the system can change as the initial condition changes, and moreover the number of samples needed increases exponentially in the dimension of the state, $n$.

The stability of switched systems closely relates closely to the \emph{joint spectral radius} (JSR) of the matrices appearing in \eqref{eq:switchedSystem}. Under certain conditions deciding stability amounts to deciding whether JSR less than one or not. There has been a lot of work on developing algorithms to approximate this quantity, when the matrices appearing in \eqref{eq:switchedSystem} are known. Therefore, our work also is connected to the identification of switched systems, since once the system \eqref{eq:switchedSystem} is identified one can then apply these well-established results. However, there are two main reasons behind our quest to directly to from data to stability and bypassing the identification phase: (1) Even when $\calM$ is known, approximating the JSR is NP-hard \cite{jungers}. (2) Identifying the set $\calM$ is also NP-hard. Therefore, the existing identification techniques can identify $\calM$ up to an approximation error. As a result, how to relate this identification error to an error in deciding the stability of \eqref{eq:switchedSystem} is still nontrivial.

In this paper, we present an algorithm to approximate the JSR of a switched linear system from $N$ input-output pairs. This algorithm provides an upper bound on the JSR with a user-defined confidence level. As the number of samples increases, this bound gets tight. Moreover, we characterize with a closed form expression what the exact trade-off between the tightness of this bound and the number of samples. In order to understand the quality of our technique, the algorithm also provides a deterministic lower-bound.
%
%In this paper, we consider discrete-time switched linear systems of the form:
%\begin{equation}\label{eq:dynamicalsystem}x_{k+1} = A_{\sigma(k)}x_k,
%\end{equation}
%where, $x_k \in \R^n$, k is index of time and $\sigma: \N \to \{1,2, \ldots, m\}$ is the switching sequence. Let $y_k := x_{k+1}$. 
%We ask the following question:
%\begin{centering}
%\emph{Given N input-output-matrix pairs, $(x_1, y_1)$, $(x_2, y_2)$, $\ldots$, $(x_N, y_N)$ such that
%\begin{equation}\label{eq:triples}y_{k} = A_{\sigma(k)}x_k,
%\end{equation} for some $\sigma(k),$ what can we say about the stability of the system \eqref{eq:dynamicalsystem}?
%\end{centering}
%The only instance where answer to this question has been clearly shown is for linear time-invariant systems.
%
%Note that if \eqref{eq:dynamicalsystem} is a linear time-invariant system with only one mode, this question is easily answered by observing $n$ linearly independent data points. Switched systems can be used to model the behavior of a system of interest for different values of a parameter that varies. This parameter can represent internal parameters such as model uncertainties, as well as exogenous parameters such as inputs provided by a controller in a closed-loop control system. 


%- Stability analysis of dynamical systems is a challenging task. \\
%- Most methods in literature rely on knowing the model. Cite \\
%- How to make the case about the assumption of switched linear yet, we don't construct a model? Here cite
%something that says that this is a hard problem in general. (What is the relationship between switching linear regression)
%- Here we present stability analysis from data, this is easy for linear systems. We present something for switch linear.



%I can refer to the condition number paper?

\section{Preliminaries}\label{sec:preliminaries}
\subsection{Notations}
% !TEX root = main.tex
We consider the usual Hilbert finite normed vector space $(\R^n,\ell_2)$, $n \in \mathbb{N}_{> 0}$, $\ell_2$ the classical euclidean norm. We denote a unit ball in $\R^n$ with $\ball$ and unit sphere in $\R^n$ of radius $r$ as $\sphere$. We only denote the radius $r$ explicitly as in $\ball_r$ and $\sphere_r$, when $r$ is different than $1$. We denote by $\proj_{\sphere}$ the (convex) projector on $\sphere$. We denote the set of real symmetric matrices of size $n$ by $\calS^n(\R)$, and the set of linear functions in $\R^n$ by $\mathcal{L}(\mathbb{R}^n)$. We denote the ellipsoid described by the matrix $P \in \mathbb{S}^n$ as $E_P$, i.e., $E_P:= \{x \in \R^n: x^TPx = 1\}$. We denote the homothety of ratio $r$ by $\calH_{r}$. 
%We denote by $\wp(M)$ the power set of $M$.

%We denote by $M$ the set $\{1,2,\dots,m\}$, ($m \in \N_{>0}$) and consider a switched linear system of modes $\calM = \{A_1, A_2,\dots, A_m\} \subset \R^{n \times n}$ indexed by $M$. We observe the system: 
%\begin{equation}\label{eq:dynamicalsystem}x_{k+1} = A_{\sigma(k)}x_k,
%\end{equation}
%where, $x_k \in \R^n$, k is index of time and $\sigma: \N \to M$ is the switching sequence. Let $y_k := x_{k+1}$. 
%We assume that we only know the number of modes $m$, the input $x_k$ and the output $y_k$ at each time $k$. We ignore what are the matrices and the index of the matrix which is applied at every time. We consider the following problem.

We consider the classical unsigned and finite uniform spherical measure on $\sphere$, commonly denoted by $\sigma^{n-1}$. It is associated to the spherical Borelian $\sigma$-algebra and is derived from the Lebesgue measure $\lambda$. We have $\mathcal{B}_{\sphere}$ with $\calA \in \mathcal{B}_{\sphere}$ if and only if the sector $t \calA$, $t \in [0,1]$ is in $\mathcal{B}_{\mathbb{R}^n}$. The spherical measure $\sigma^{n-1}$ is defined by
$$\forall\ \calA \in \mathcal{B}_{\sphere},\, \sigma(\calA) = \frac{\lambda(t\calA)}{\lambda(\mathcal{B})}. $$
In other words, the spherical measure of a subset of the sphere is related to the Lebesgue measure of the sector of the unit ball it induces. Notice that $\sigma^{n-1}(\sphere) = 1$.

 
\subsection{Stability of Switched Linear Systems}\label{sec:stab}
 % !TEX root = main.tex
A \emph{switched linear system} with a set of modes \\ 
\mbox{$\mathcal{M}= \{A_i, i \in M \}$} is of the form:
\begin{equation}\label{eq:switchedSystem}
x_{k+1} = f(k,x_k),
\end{equation}
with $f(k,x_k) = A_{\tau(k)}x_k$ and switching sequence \mbox{$\tau: \mathbb{N} \to M$}.

In this paper, we are interested in the worst-case global stability of this system, that is, we want to guarantee the following property: $$\forall\ \tau \in M^{\mathbb{N}}, \forall\ x_0 \in \mathbb{R}^n, \lVert x_k \rVert \to_{k \to +\infty} 0.$$ 
It is well-known that the joint spectral radius of a set of matrices $\mathcal{M}$ closely relates to the stability of the underlying switched linear systems \eqref{eq:switchedSystem} defined by $\mathcal{M}$. This quantity is an extension to switched linear systems of the classic spectral radius for linear systems. It is the maximum asymptotic growth rate of the norm of the state under the dynamics \eqref{eq:switchedSystem}, over all possible initial conditions and sequences of matrices of $\mathcal{M}$.

\begin{defn}[from \cite{jungers_lncis}] 
Given a finite set of matrices \mbox{$\mathcal{M} \subset \mathbb{R}^{n\times n}$}, its \emph{joint spectral radius} (JSR) is given by $$\rho(\mathcal{M}) =\lim_{k \rightarrow +\infty} \max_{i_1,\dots, i_k} \left\{ ||A_{i_1} \dots A_{i_k}||^{1/k}: A_{i_j} \in \mathcal{M}\ \right\}. $$
\end{defn}

\begin{property}[Corollary 1.1, \cite{jungers_lncis}]
Given a finite set of matrices $\mathcal{M}$, the corresponding switched dynamical system is stable if and only if $\rho(\mathcal{M})<1$.
\end{property}

\begin{property}[Proposition 1.3, \cite{jungers_lncis}]\label{rem:scaling}
Given a finite set of matrices $\mathcal{M}$, and any invertible matrix $T$, $$\rho(\mathcal{M})=\rho(T \mathcal{M} T^{-1}),$$
i.e., the JSR is invariant under similarity transformations (and is a fortiori a homogeneous function: $\forall\  \gamma > 0$, $\rho \left( \mathcal{M}/\gamma \right) = \mathcal{M}/\gamma$).
\end{property}

The JSR also relates to a tool classically used in control theory to study stability of systems: Lyapunov functions. We will consider here a family of such functions that is particularly adapted to the case of switched linear systems.

\begin{defn}
Consider a finite set of matrices $\mathcal{M} \subset \mathbb{R}^{n \times n}$. A \emph{common quadratic Lyapunov function (CQLF)} for a system \eqref{eq:switchedSystem} with set of matrices $\mathcal{M}$, is a positive definite matrix $P \in \mathcal{S}_{++}^n$ such that for some $\gamma \geq 0$, $$\forall\ A \in \mathcal{M}, A^T P A \preceq \gamma^2P.$$
\end{defn}
CQLFs are useful because they can be computed (if they exist) with semidefinite programming (see \cite{boyd}), and they constitute a stability guarantee for switched systems as we formalize next.

\begin{thm}\cite[Prop. 2.8]{jungers_lncis}\label{thm:cqlf} 
Consider a finite set of matrices $\mathcal{M}$. If there exist some $\gamma \geq 0$ and $P \succ 0$ such that $$\forall\ A \in \mathcal{M}, A^T P A \preceq \gamma^2 P,$$ then $\rho(\mathcal{M}) \leq \gamma$.
\end{thm}

It turns out that one can guarantee the accuracy of this Lyapunov technique thanks to the following converse CQLF theorem.

\begin{thm}\cite[Theorem 2.11]{jungers_lncis}\label{thm:john}
For any finite set of matrices such that $\rho(\mathcal{M}) < \frac{1}{\sqrt{n}},$ there exists a CQLF for $\mathcal{M}$, that is, a $P \succ 0$ such that: $$\forall\ A\in \mathcal{M},\, A^T P A \preceq P. $$
\end{thm}

Note that, the smaller $\gamma$ is in \ref{thm:cqlf}, the tighter is the upper bound we get on $\rho(\mathcal{M})$. Therefore, we could consider, in particular, the optimal solution $\gamma^*$ of the following optimization problem:
\begin{equation}\label{eqn:campiOpt0}
\begin{aligned}
& \text{min}_{\gamma, P} & & \gamma\\
& \text{s.t.} 
&  & (Ax)^T P(Ax) \leq \gamma^2 x^T P x,\,\forall\ A \in \mathcal{M}, \,\forall\, x \in \mathbb{R}^n\\
& && P \succ 0. \\
\end{aligned}
\end{equation}
%where we were able to replace the condition $\forall\,x \in \R^n$ with $\,\forall\, x \in \sphere$, where $\sphere$ is the unit sphere. This is thanks to the Property \ref{property:homogeneity} which implies that it is sufficient to show that a CQLF is decreasing on a set enclosing the origin, e.g. the unit sphere. 

%Even though this upper bound is more difficult to obtain in a black-box setting where only a finite number of observations are available, in this section we leverage Theorem \ref{thm:john} in order to derive a straightforward lower bound.

\begin{property}\label{property:homogeneity}
Let $\xi(x, k, \tau)$ denote the state of the system \eqref{eq:switchedSystem} at time $k$ starting from the initial condition $x$ and with switching sequence $\tau$. The dynamical system \eqref{eq:switchedSystem} is homogeneous: $\xi(\gamma x, k, \tau)= \gamma \xi(x, k, \tau).$
\end{property}

Property \ref{property:homogeneity} enables us to restrict the set of constraints $x$ to the unit sphere $\sphere$, instead of considering it as being all $\mathbb{R}^n$. Indeed, the homogeneity implies that it is sufficient to show the decrease of a CQLF on an arbitrary set enclosing the origin, e.g., $\sphere$. Hence, we consider from now the following optimization problem, with its optimal solution that will also be $\gamma^{*}$:
\begin{equation}\label{eqn:campiOpt1}
\begin{aligned}
& \text{min}_{\gamma, P} & & \gamma \\
& \text{s.t.} 
&  & (Ax)^TP(Ax) \leq \gamma^2 x^TPx,\,\forall\ A \in \mathcal{M}, \,\forall\, x \in \sphere\\
& && P \succ 0. \\
\end{aligned}
\end{equation}


The two theorems above provide us with a \emph{converse Lyapunov result}: if there exists a CQLF, then our system is stable. If, on the contrary, there is no such stability guarantee, one may conclude a lower bound on the JSR. By combining these two results, one obtains an approximation algorithm for the JSR: the upper bound $\gamma^*$ obtained above is within an error factor of $\frac{1}{\sqrt{n}}$ of the true value. It turns out that one can still refine this technique, in order to improve the error factor, and asymptotically get rid of it. This is a well-known technique for the ``white-box'' computation of the JSR, which we summarize in the following corollary.

\begin{cor}\label{cor:approx-products}
For any finite set of matrices such that $\rho(\mathcal{M}) < \frac{1}{\sqrt[2l]{n}},$ there exists a CQLF for $$\mathcal{M}^l:=\{A_{i_1},\dots, A_{i_{l}}: A_i \in \mathcal{M}\},$$ that is, a $P\succ 0$ such that: $$\forall\ A \in \mathcal{M}^l,\, A^T P A\preceq P. $$
\end{cor}

\begin{pf}
It is easy to see from the definition of the JSR that $$\rho(\mathcal{M}^l)=\rho(\mathcal{M})^l.$$ Thus, applying Theorem \ref{thm:john} to $\mathcal{M}^l,$ one directly obtains the corollary.
\end{pf}
We can then consider from now, for any $l \in \mathbb{N}_{>0}$, the following optimization problem, whose solution $\gamma^{*}$ will be an upper bound on $\rho(\mathcal{M})$:
\begin{equation}\label{eqn:campiOpt2}
\begin{aligned}
& \text{min}_{\gamma, P} & & \gamma \\
& \text{s.t.} 
&  & (Ax)^T P(Ax) \leq \gamma^{2l} x^T P x,\,\forall\ A \in \mathcal{M}^l, \,\forall\, x \in \sphere\\
& && P \succ 0. \\
\end{aligned}
\end{equation}
\subsection{Problem Formulation}
 % !TEX root = main.tex
We now formally present the problem we will be considering from now. We recall that our observations are traces of the form $(x_k,x_{k+1},\dots,x_{k+l})$ for some arbitrary $l \in \mathbb{N}_{>0}$, and that we do not have access to the mode applied to the system at each time step. To generate these traces, we assume that we can randomly draw a finite number of initial conditions $x_0^i \in \sphere$, and that a random sequence of $l$ modes is applied to each of these points. Hence, the probability event corresponding to a given observed trace $(x_k,x_{k+1},\dots,x_{k+l})$ is another $(l+1)$-tuple $(x_k,j_1,\dots,j_l)$. More precisely, we assume that we can uniformly sample $N$ such $(l+1)$-tuples in $Z_l = \sphere \times M^l$, giving us a sample denoted by 
\begin{equation*}
\omega_N := \{(x_0^i, j_{i,1}, \dots, j_{i,l}), 1 \leq i \leq N \}  \subset Z_l
\end{equation*}
and a corresponding set of $N$ available observations $\{(x_0^i,x_1^i, \dots, x_l^i), 1\leq i\leq N \}$ which satisfy for all $1 \leq i \leq N$ and $1 \leq k \leq l$, $x_k^i= A_{j_{k,i}} \dots A_{j_{1,i}} x_0^i$.
\begin{rem}\label{rem:probGeneralize}
Let us motivate the choice of considering a uniform sampling for the modes. We assumed that we only have access to random observations of the state of the system, and ignore the process that generates them. In particular, we ignore the process that picks the modes at each time step, and model it as a random process. We suppose that with nonzero probability, each mode is active: the problem would indeed not make a lot of sense otherwise, since in a such case, the system would be unidentifiable with probability $1$ and would prevent to ever observe some of its behaviors. We take this distribution uniform here since we cannot say that some modes are privileged a priori. We can still take any other distribution satisfying our assumption: if we have a nonzero lower bound on the probability of each mode, our guarantees naturally extend to them.
\end{rem}
In this work, we aim at understanding what type of guarantees one can obtain on the stability of System \eqref{eq:switchedSystem} (that is, on the JSR of $\mathcal{M}$) from such data. More precisely, we answer the following problem:
\begin{prob}\label{problem} 
Consider a finite set of matrices, $\mathcal{M},$ describing a switched system \eqref{eq:switchedSystem}, and suppose that one has a set of $N$ observations relative to $\omega_N$, finite sample of $Z_l$ drawn according to the uniform measure $\mu_l$.
\begin{itemize}
\item For a given number $\beta \in [0,1)$, provide an upper bound on $\rho(\mathcal{M})$, together with a level of confidence $\beta$, that is, a number $\overline{\rho(\omega_N)}$ such that $$\mu_l \left( \{\omega_N: \ \rho(\mathcal{M}) \leq \overline{\rho(\omega_N)} \} \right) \geq \beta.$$
\item For the same given level of confidence $\beta$, provide a lower bound $\underline{\rho(\omega_N)}$ on $\rho(\mathcal{M})$.
\end{itemize}
\end{prob}
\begin{rem}
We will see in Section~\ref{sec:lowerBound} that we can even derive a deterministic lower bound for a given (sufficiently high) number of observations. 
\end{rem}
The idea from now will be to leverage the fact that conditions for the existence of a CQLF for \eqref{eq:switchedSystem} can be obtained by considering a finite number of traces in $\mathbb{R}^n$ of the form $(x_k,x_{k+1}, \dots, x_{k+l})$. It will lead us to the following algorithm, that is the main result of our work and that answers Problem~\ref{problem}:
\begin{alg}[Probabilistic upper bound]
 \ 
\newline
\textbf{Input:} observations produced by a uniform random sample $\omega_N \subset Z$ of size $N \geq \frac{n(n+1)}{2}+1$;

\vspace{-0.2cm}
\textbf{Input:} $\beta$ desired level of certainty;

\vspace{-0.2cm}
\textbf{Compute:} a candidate for the upper bound, $\gamma^{*}(\omega_N)$, solution of a convex optimization problem;

\vspace{-0.2cm}
\textbf{Compute:} $\varepsilon(\beta,\omega_N)$ the proportion of points where our inference on the upper bound may be invalid;

\vspace{-0.2cm}
\textbf{Compute:} $\delta(\varepsilon)$ a correcting factor;

\vspace{-0.2cm}
\textbf{Output:} $\frac{\gamma^{*}(\omega_N)}{\sqrt[2l]{n}} \leq \rho \leq \frac{\gamma^{*}(\omega_N)}{\sqrt[l]{\delta(\varepsilon)}}$, (right-hand inequality valid with probability at least $\beta$ and $\delta(\beta, \omega_N)   \xrightarrow[N \to \infty]{} 1$).
\end{alg}

\section{A Deterministic Lower Bound}\label{sec:lowerBound}
% !TEX root =main.tex
In Section~\ref{sec:stab}, we gave an optimization problem, \eqref{eqn:campiOpt2}, that provides a stability guarantee. Nevertheless, solving this problem as stated solely from observation of traces (that gives a finite number of constraints) is not possible since \eqref{eqn:campiOpt2} involves infinitely many constraints. We consider then the following optimization problem:
\begin{equation}\label{eq:lowerbound}
\begin{aligned}
& \text{min}_P & & \gamma \\
& \text{s.t.} 
&  & (\mathbf{A}_l x)^T P \mathbf{A}_l x \leq \gamma^{2l} x^T P x, \,  \forall (x, \mathbf{j}) \in \omega_N\\
& && P \succ 0,\ \gamma \geq 0. \\
\end{aligned}
\end{equation}
where $\mathbf{A}_l :=  A_{j_l} A_{j_{l-1}} \dots A_{j_1}$ and $\mathbf{j}:=\{j_1,\dots, j_l\}$. Note that, \eqref{eq:lowerbound} can be efficiently solved by semidefinite programming and bisection on the variable $\gamma$ (see \cite{boyd}). Let us denote from now by $\gamma^*(\omega_N)$ the optimal solution of this problem, which we will use to compute a deterministic lower bound and a probabilistic upper bound on the JSR. In this section, we provide a theorem for a deterministic lower bound based on the observations given by $\omega_N$, whose accuracy depends on the ``horizon'' $l$.
\begin{thm}\label{thm:lowerbound}
For an arbitrary $l \in \mathbb{N}_{>0}$ and a given uniform sample $\omega_N \subset Z_l$, by considering $\gamma^*(\omega_N)$ the optimal solution of the optimization problem \eqref{eq:lowerbound}, we have $$\rho(\mathcal{M}) \geq \frac{\gamma^*(\omega_N)}{\sqrt[2l]{n}}.$$ 
\end{thm}
\vspace{-0.8cm}
\begin{pf*}{Proof.}
Let $\nu >0$. By definition of $\gamma^*(\omega_N)$, there exists no matrix $P \in \mathcal{S}^n_{++}$ such that, $\forall x \in \sphere, \, \forall \mathbf{A}_l \in \mathcal{M}^{l}$,
\begin{equation*}
(\mathbf{A}_l x)^T P \mathbf{A}_l x \leq (\gamma^*(\omega_N) -\nu)^{2l} x^T P x.
\end{equation*}
By Theorem~\ref{prop:scaling}, this means that there exists no CQLF for the scaled set of matrices $\frac{\mathcal{M}^l}{(\gamma^*(\omega_N)-\nu)^l}$. Since this is valid for every $\nu > 0$, using Theorem~\ref{thm:cqlf}, and the fact that $\rho(\mathcal{M}^l)=\rho(\mathcal{M})^l,$ we conclude:
\begin{equation*}
\frac{\rho(\mathcal{M})}{\gamma^*(\omega_N)} \geq \frac{1}{\sqrt[2l]{n}}.
\end{equation*}
\end{pf*}

\section{A Probabilistic Stability Guarantee}\label{sec:upperbound}
% !TEX root = main.tex
\begin{theorem}
Consider a black-box switching system and $N$ samples of its dynamics as in \eqref{eq:switchedSystem}. Consider the optimal solution $(\lambda^*,P)$ which minimizes $\lambda$ in \eqref{eq:lowerbound}. For any factor $1<\delta,$ one can compute the level of confidence $\beta$ such that $\rho<\delta\cdot\lambda^*.$ 
% denote $\gamma(P,\epsilon)$ the largest $\gamma$ such that $\gamma^2y_i^TPy_i\leq x_i^TPx_i$ 
\end{theorem}

\begin{proof}
Let us fix $\delta>1$ and denote $E_P$ the ellipsoid described by $P$ (i.e., $\{x:x^TPx= 1\}$), and denote $\epsilon$ such that for any subset $S_\epsilon$ of measure $\epsilon,$ 
$$ E_{\delta^2P} \subset  \convhull (E_P\setminus S_\epsilon) .$$ 
  	
Now, denoting $N$ the number of observations available, compute $0< \beta< 1$ such that $$N=N(\epsilon,\beta)$$ in Theorem \ref{thm:campi} above. 

Summarizing, the equation above means that with high probability, one has that \eqref{eq:lowerbound} is satisfied for all $x\in \re^n,$ except for a set of measure $\epsilon.$  Let us denote $S_\epsilon$ this set of violated constraints.  Thus, $$(\cM/\lambda^*) \convhull (E_P\setminus S_\epsilon) \subset \convhull (E_P\setminus S_\epsilon).$$  Now, by definition of $\epsilon,$ one has

$$ E_{\delta^2P} \subset  \convhull (E_P\setminus S_\epsilon) ,$$ and so
$$
(\cM/\delta \lambda^*) \convhull (E_P \setminus S_\epsilon) \subset \convhull (E_P\setminus S_\epsilon).$$

Then, $\delta\lambda$ is un upper bound on $\rho,$ with a confidence $\beta.$ 
\end{proof}
% !TEX root = main.tex

\begin{theorem}\label{thm:mainTheorem}Consider an $n$-dimensional switching system as in \eqref{eq:switchedSystem}. For any given $\beta \in (0,1]$, $\eta > 0$ and a uniform random sampling $\omega_N \subset Z$, with $N \geq \frac{n(n+1)}{2}+1$, and let $\gamma^*(\omega_N) $ be the optimal solution to \eqref{eqn:campiOpt01}. Then, we can compute $\delta(\beta, \omega_N)$, such that with probability at least $\beta$ we have:
$$\rho \leq \frac{\gamma^*(\omega_N) (1+ \eta)}{\delta(\beta, \omega_N)},$$
where $\lim_{N \to \infty}\delta(\beta, \omega_N) = 1$.
\end{theorem}

\begin{proof}
Note that, by definition of $\gamma^*(\omega_N)$ we have:
\begin{equation*} (A_j x)^TP(A_j x) \leq {(\gamma^*(1+\eta))}^2 x^TPx, \quad \forall\, (x, j)  \in \omega_N \end{equation*}
for some $P \succ 0$. 
Note that the inequality \eqref{eqn:violation} in Theorem \ref{mainTheorem0} can be also written as:
\begin{equation}\label{eqn:violation2}\mathbb{P}^N\left\{ \mu(V(\omega_N)) \leq \epsilon \right\} \geq 1- I(1-\epsilon; N-d, d+1),\end{equation}
where $I(\ell;a,b)$ is the regularized in complete beta function. Then, for all $\epsilon \in (0,1)$ satisfying:
\begin{equation}\label{eqn:eps}\epsilon \leq 1- I^{-1}(1-\beta; N-d, d+1),\end{equation} we have $\mathbb{P}^N\left\{ \mu(V(\omega_N)) \leq \epsilon \right\} \geq \beta$.
Then, by Theorem \ref{mainTheorem0} for all $\epsilon$ satisfying \eqref{eqn:eps}, with probability at least $\beta$ the following holds:
\begin{equation*} (A_jx)^TP(A_jx) \leq  {(\gamma^*(1+\eta))}^2 x^TPx, \quad \forall (x, j) \in Z \setminus V.\end{equation*}
By Theorem \ref{thm:mainTheorem01}, this implies that with probability at least $\beta$ the following also holds:
\begin{equation*}(\bar{A_j} x)^T(\bar{A_j} x) \leq {\gamma}^2x^Tx,\,\, \forall\, x \in \sphere \setminus \sphere', \forall\, j \in M,\end{equation*}
for some $S'$ where $\sigma^{n-1}(S') \leq m\epsilon \kappa(P).$ Then, applying Lemma \ref{lemma:epsilon1}, we can compute $$\delta(\beta, \omega_N) =\alpha(m\kappa(P)(1- I^{-1}(1-\beta; N-d, d+1)))$$ such that with probability at least $\beta$ we have:
\begin{equation*}
\bar{A_j} \ball \subset \frac{\gamma^*(\omega_N)(1+\eta)}{\delta(\beta, \omega_N)}\ball,\, \forall\, j \in M,
\end{equation*}
By Property \ref{rem:scaling}, this means that with probability at least $\beta$:
$$\rho \leq \frac{\gamma^*(\omega_N) (1 + \eta)}{\delta(\beta, \omega_N)},$$
which completes the proof of the first part of the theorem. We show that $\lim_{N \to \infty}\delta(\beta, \omega_N) = 1$ in the Appendix \ref{app:delta}.



%
%$$\frac{\calM}{\gamma^*} \conv(E_P \setminus E') \subset \convhull(E_P \setminus E'),$$
%for some $E' \subset E_P$, where $\sigma_P(E') \leq \epsilon$.
%Then, due to Theorem \ref{thm:mainTheorem0} we also have \eqref{eqn:contraction}, meaning:
%$$\left(\frac{\cM}{\delta \gamma^*}\right) \convhull (E_P \setminus X_P) \subset \convhull (E_P\setminus X_P).$$
%Then, $\delta\gamma^*$ is an upper bound on $\rho,$ with probability at least~$\beta$. 
\end{proof}



\section{Experimental Results}\label{sec:experiments}
% !TEX root =main.tex
We illustrate our technique on a two-dimensional switched system with $4$ modes. We fix the condifence level, $\beta = 0.92$ and compute both upper and lower bounds on JSR for $N:=15+50k,\, k \in\{0, \ldots, 10\}.$ We demonstrate the performance of the average performance of our algorithm over $10$ different runs in Figure \ref{fig:1} and Figure \ref{fig:2}. Figure \ref{fig:1} demonstrates the evolution of $\delta(\beta, N)$ as $N$ increases. We observe that $\delta$ converges to $1$ as expected. In Figure \ref{fig:2}, we plot the upper bound and lower bound for the JSR of the system computed by Theorem \ref{thm:mainTheorem} and Theorem \ref{thm:lowerbound}, respectively. As can be seen, the upper boud

\begin{figure}
\begin{center}
\includegraphics[scale=0.35]{delta1.jpg}
\label{fig:1}
\caption{Evolution of $\delta$ along $N$.}
\end{center}
\end{figure}


\begin{figure}
\begin{center}
\includegraphics[scale=0.35]{bounds1.jpg}
\label{fig:2}
\caption{Evolution of the bounds along $N$.}
\end{center}
\end{figure}

We observe that $\delta$ converges to $1$, the upper-bound converges to $\rho$, the value of the JSR, and the lower-bound converges to the value of the JSR divided by $\sqrt{n}$.

We then test our algorithm with a system in dimension $4$, with $5$ modes and JSR equal to $0.7929$. We observe a slower convergence of the values, even with more points in our sampling.
[to add plots here, but they are not super nice: we do not see the convergence to $1$ of $\delta$: even after $10 000$ points, $\delta$ is below $0.85$].

Finally, we generate $10000$ cases with systems of dimension between $2$ and $7$, number of modes $m$ between $2$ and $5$, and size of samples $N$ between $30$ and $800$. We take $\beta = 0.92$ and we test if the upper bound computed is greater than the actual JSR of the system. We get $xxx$ positive tests, giving us a probability of having an upper bound equal to $xxx$. 





\section{Conclusions}\label{sec:conclusions}
% !TEX root =main.tex
In this paper, we investigated the question of how one can conclude stability of a dynamical system when a model is not available and, instead, we have state measurements. Our goal is to understand how the observation of well-behaved trajectories \emph{intrinsically} implies stability of a system. It is not surprising that we need some standing assumptions on the system, in order to allow for any sort of nontrivial stability certificate solely from a finite number of observations. The novelty of our contribution is twofold: First, we use as standing assumption that the unknown system can be described by a switching linear system. This assumption covers a wide range of systems of interest, and to our knowledge no such ``black-box'' result has been available so far on switched systems.  
Second, we apply powerful techniques from chance constrained optimization. The application is not obvious, and relies on geometric properties of linear switched systems. We believe that this guarantee is quite powerful, in view of the hardness of the general problem. In the future, we plan to investigate how to generalize our results to more complex or realistic systems. We are also improving the numerical properties of our technique by incorporating sum-of-squares optimization, and relaxing the sampling assumptions on the observations.



\begin{ack}                               % Place acknowledgements
Acknowledgements, NSF etc.  % here.
\end{ack}

\bibliographystyle{plain} 
\bibliography{bibliography}

\appendix\label{appendix}
% !TEX root = main.tex
\begin{section}{Precisions on Measures Definitions}\label{app:measures}

For an ellipsoid centered at the origin, and for any of its subsets $\mathcal{A}$, the \emph{sector} defined by $\mathcal{A}$ is the subset $$\{t \mathcal{A}, \ t \in [0,1]\} \subset\ \mathbb{R}^n.$$ We denote by $E_P^{\mathcal{A}}$ the sector induced by $\mathcal{A} \subset E_P$. In the particular case of the unit sphere, we instead write $\sphere^{\mathcal{A}}$. We can notice that $E_P^{E_P}$ is the volume in $\mathbb{R}^n$ defined by $E_P$: $E_P^{E_P} = \{x \in \mathbb{R}^n: x^T P x \leq 1\}$.

The spherical Borelian $\sigma$-algebra, denoted by $\mathcal{B}_{\sphere}$, is defined by $$\mathcal{A} \in \mathcal{B}_{\sphere} \iff \sphere^{\mathcal{A}} \in \mathcal{B}_{\mathbb{R}^n}.$$ We provide $(\sphere,\mathcal{B}_{\sphere})$ with the classical, unsigned and finite uniform spherical measure $\sigma^{n-1}$ defined by
$$\forall\ \mathcal{A} \in \mathcal{B}_{\sphere},\, \sigma(\mathcal{A}) = \frac{\lambda(\sphere^{\mathcal{A}})}{\lambda(\ball)}. $$
In other words, the spherical measure of a subset of the sphere is related to the Lebesgue measure of the sector of the unit ball it induces. Notice that $\sigma^{n-1}(\sphere) = 1$.

Since $P \in \mathcal{S}_{++}^n$, we recall that it can be written in its Cholesky form \eqref{cholesky}. Note that, $L^{-1}$ maps the elements of $\sphere$ to $E_P$. Then, we define the measure on the ellipsoid $\sigma_P$ on the $\sigma$-algebra $\mathcal{B}_{E_P}:=L^{-1}(\mathcal{B}_\sphere)$, where $\forall\, \mathcal{A} \in \mathcal{B}_{E_P},\, \sigma_P({\mathcal{A}}) = \sigma^{n-1}(L\mathcal{A})$. 

Set $M$ is provided with the classical $\sigma$-algebra associated to the finite sets: $\Sigma_M = \wp(M)$, where $\wp(M)$ is the set of subsets of $M$. We provide $(M, \Sigma_M)$ with the uniform measure $\mu_M$. Similarly, we define $\Sigma_{M^l}$ as the product $\bigotimes^l \Sigma_M$ (which is here equal to $\wp(M)^l$), and we provide $(M^l, \Sigma_{M^l})$ with the uniform product measure $\mu_{M^l} = \otimes^l \mu_M$.

We can now denote the product $\sigma$-algebra $\mathcal{B}_{\sphere} \bigotimes (\Sigma_{M^l})$ generated by $\mathcal{B}_{\sphere}$ and $\Sigma_{M^l}$: $\Sigma = \sigma( \pi_{\sphere}^{-1}(\mathcal{B}_{\sphere}),  \pi_{M^l}^{-1}(\Sigma_{M^l}))$, where $\pi_{\sphere}: Z_l \to \sphere$ and $\pi_{M^l}: Z_l \to M^l$ are the standard projections. On $(Z_l,\mathcal{B}_{\sphere} \bigotimes (\Sigma_{M^l}) )$, we define the product measure $\mu_l = \sigma^{n-1} \otimes \mu_{M^l}$. Note that, $\mu_l$ is a uniform measure on $Z_l$ and $\mu_l(Z_l)=1$.  

\end{section}


\begin{section}{Proof of Corollary~\ref{cor:gettingRidOfm}}\label{proof:cor_m}
We prove here Corollary~\ref{cor:gettingRidOfm} stated in section~\ref{sec:IntroMainThm}.

Let $\sphere' = \pi_{\sphere}(V)$. We know that $\Sigma_{M^l}$ is the disjoint union of its $2^{m^l}$ elements $\{\mathcal{M}^l_i, i \in \{1,2, \dots, 2^{m^l} \} \}$. Then $V$ can be written as the disjoint union $V = \sqcup_{1 \leq i \leq 2^{m^l}} (\mathcal{S}_i, \mathcal{M}^l_i)$ where $\mathcal{S}_i \in \Sigma_{\sphere}$. We notice that 
$\sphere' = \sqcup_{1 \leq i \leq 2^{m^l}} \mathcal{S}_i$, 
and
\begin{equation*}
\sigma^{n-1} (\sphere') = \sum_{1 \leq i \leq 2^{m^l}} \sigma^{n-1} (\mathcal{S}_i).
\end{equation*}
We have 
\begin{eqnarray*}
\mu_l(V) &=& \mu_l \left( \sqcup_{1 \leq i \leq 2^{m^l}} (\mathcal{S}_i, \mathcal{M}^l_i) \right) = \sum_{1 \leq i \leq 2^{m^l}} \mu_l \left( \mathcal{S}_i, \mathcal{M}^l_i \right) \\
 &=& \sum_{1 \leq i \leq 2^{m^l}} \sigma^{n-1} \otimes \mu_{M^l} \left( \mathcal{S}_i, \mathcal{M}^l_i \right) \\
 &=& \sum_{1 \leq i \leq 2^{m^l}} \sigma^{n-1}(\mathcal{S}_i) \mu_{M^l} (\mathcal{M}^l_i).
\end{eqnarray*}

Note that we have $\min_{(j_1,\dots,j_l) \in M^l} \mu_{M^l}(\{j_1,\dots,j_l\}) = \frac{1}{m^l}.$ Then since $ \forall \ i$, $\mu_{M^l}(\mathcal{M}^l_i) \geq \min_{(j_1,\dots,j_l) \in M^l} \mu_{M^l}(\{j_1,\dots,j_l\}) = \frac{1}{m^l}$, we get:
\begin{equation}
\sigma^{n-1}(sphere') \leq \frac{\mu_l(V)}{\frac{1}{m^l}} \leq m^l \varepsilon.
\end{equation}
This means that 
\begin{equation}\label{eqn:P2}
\begin{aligned}
& (A_{j_l} A_{j_{l-1}} \dots A_{j_1} x)^T P (A_{j_l} A_{j_{l-1}} \dots A_{j_1} x) \leq \gamma^{2l} x^T P x,\\
& \qquad \qquad \qquad \qquad \forall\, x \in \sphere \setminus \sphere', \,\forall\, (j_1,\dots,j_l) \in M^l,
\end{aligned}
\end{equation}
where $\sigma^{n-1}(\sphere') \leq m^l \varepsilon.$

\end{section}


\begin{section}{Preliminary Results on Spherical Caps}\label{sec:app_prelim_caps}

Before proceeding to the proof of Proposition~\ref{lemma:propositionSphericalCaps}, we first introduce some necessary definitions and related background on spherical caps. We recall that a \emph{spherical cap} on $\sphere$ for a given hyperplane $c^Tx = k$ is defined by $\mathcal{C}_{c,k} := \{x \in \sphere : c^Tx >k\}$.


\begin{rem}\label{lemma:muMonotone}
Consider the spherical caps $\mathcal{C}_{c, k_1}$ and $\mathcal{C}_{c, k_2}$ such that $k_1 > k_2$, then we have:
$$\sigma^{n-1}(\mathcal{C}_{c,k_1}) < \sigma^{n-1}(\mathcal{C}_{c,k_2}).$$
\end{rem}


\begin{rem}\label{prop:distance}
The distance between the point $x=0$ and the hyperplane $c^Tx = k$ is $\frac{|k|}{\|c\|}$.
\end{rem}

We also recall that we defined in section~\ref{sec:pbsphere} the function $\Delta: \wp(\sphere) \to [0,1]$ as $\Delta(X) := \sup \{r: r\ball \subset \conv(\sphere \setminus X)\}$.


\begin{lem} \label{lemma:delta2} 
Consider the spherical cap $\mathcal{C}_{c,k}$. We have:
$$\Delta(\mathcal{C}_{c,k}) = \min\left(1, \frac{|k|}{\|c\|}\right).$$
\end{lem}

\begin{pf}
Note that: $$\conv(\sphere \setminus X)= \left\{x \in \ball : c^Tx \leq k \right\}.$$
Then the following equalities hold:
\begin{eqnarray}
\nonumber \Delta(X) &=& \dist(\partial \conv(\sphere \setminus X), 0) \\
\nonumber &=& \min(\dist(\partial \ball, 0), \dist(\partial\{x : c^Tx \leq k\}, 0)) \\
\nonumber &=& \min(\dist(\sphere, 0), \dist(\{x : c^Tx = k\}, 0)) \\
\nonumber &=& \min\left(1, \frac{|k|}{\|c\|}\right).
\end{eqnarray}
\end{pf}

\begin{cor}\label{lemma:deltaMonotone} 
Consider the spherical caps $\mathcal{C}_{c, k_1}$ and $\mathcal{C}_{c, k_2}$ such that $k_1 \leq k_2$. Then we have: $$\Delta(\mathcal{C}_{c,k_1}) \leq \Delta(\mathcal{C}_{c,k_2}).$$
\end{cor}

\begin{lem}\label{lemma:constructSC}
For any set $X \subset \sphere$, there exist $c$ and $k$ such that $\mathcal{C}_{c,k}$ satisfies: $\mathcal{C}_{c,k} \subset X,$ and $\Delta(\mathcal{C}_{c,k}) = \Delta(X).$
\end{lem}

\begin{pf} Let $\tilde{X} := \conv(S \setminus X)$.
Since $\dist$ is continuous and the set $\partial \tilde{X}$ is compact, there exists a point $x^* \in \partial \tilde{X}$, such that:
\begin{eqnarray}\nonumber \Delta(X) = \dist(\partial\tilde{X}, 0) = 
\label{deltaSupporting} \min_{x \in \partial \tilde{X}}\dist(x, 0) = \dist(x^*, 0).\end{eqnarray} 
Next, consider the hyperplane which is tangent to the ball $\Delta(X)\ball$ at $x^*$, which we denote by $\left\{x : c^Tx = k\right\}$.
%\begin{equation*} \partial (\Delta(X)\ball) \subset \tilde{X} \subset \left\{x : c^Tx \leq k\right\}.\end{equation*}  By Remark \ref{suppHyperplaneRemark}, this implies that $\left\{x : c^Tx = k\right\}$ is in fact the unique supporting hyperplane at $x^*$.
Then we have:
\begin{equation*}\Delta(X) =  \dist(x^*, 0) = \dist(\{x : c^Tx = k\}, 0) = \min\left(1, \frac{|k|}{\|c\|}\right).
\end{equation*}
Now, consider the spherical cap defined by this tangent plane i.e., $\mathcal{C}_{c, k}$. Then, by Lemma \ref{lemma:delta2} we have
$\Delta(\mathcal{C}_{c,k}) =  \min\left(1, \frac{|k|}{\|c\|}\right)$. Therefore, $\Delta(X) = \Delta(\mathcal{C}_{c,k})$.


We next show $\mathcal{C}_{c, k} \subset X$. We prove this by contradiction. Assume $x \in \mathcal{C}_{c,k}$ and $x \notin X$. Note that, if $x \notin X$, then $x \in \sphere \setminus X \subset \conv(\sphere \setminus X).$ Since $x \in \mathcal{C}_{c,k}$, we have $c^Tx>k$. But due to the fact that $x \in \conv(\sphere \setminus X)$, we also have $c^Tx \leq k$, which leads to a contradiction. Therefore, $\mathcal{C}_{c, k} \subset X$. 
\end{pf}

We are now able to prove Proposition~\ref{thm:mainSphericalCap} given in section~\ref{sec:pbsphere}, which states that, for any $\varepsilon \in (0,1)$, the function $\Delta(X)$ attains its minimum over $\mathcal{X}_{\varepsilon}$ for some $X$ which is a spherical cap.


\begin{pf}
We prove this via contradiction. Assume that there exists no spherical cap in $\mathcal{X}_{\varepsilon}$ such that $\Delta(X)$ attains its minimum. This means there exists an $X^* \in \mathcal{X}_{\varepsilon}$, where $X^*$ is not a spherical cap and $\argmin_{X \in \mathcal{X}_{\varepsilon}}(\Delta(X))=X^*$. By Lemma \ref{lemma:constructSC}, we can construct a spherical cap $\mathcal{C}_{c,k}$ such that $\mathcal{C}_{c,k} \subset X^*$ and $\mathcal{C}_{c,k} = \Delta(X^*)$. Note that, we further have $\mathcal{C}_{c,k} \subsetneq X^*$, since $X^*$ is assumed not to be a spherical cap. This means that, there exists a spherical cap $\sigma^{n-1}(\mathcal{C}_{c,k})$ such that $\sigma^{n-1}(\mathcal{C}_{c,k}) < \varepsilon$. 

Then, the spherical cap $\mathcal{C}_{c, \tilde{k}}$ with $\sigma^{n-1}(\mathcal{C}_{c, \tilde{k}}) = \varepsilon$, satisfies $\tilde{k} < k$ by Remark \ref{lemma:muMonotone}. This implies $$\Delta(\mathcal{C}_{c, \tilde{k}}) < \Delta(\mathcal{C}_{c, k}) = \Delta(X^*)$$ by Corollary \ref{lemma:deltaMonotone}. Therefore, $\Delta(\mathcal{C}_{c, \tilde{k}}) < \Delta(X^*)$. This is a contradiction since we initially assumed that $\Delta(X)$ attains its minimum over $\mathcal{X}_{\varepsilon}$ at $X^*$.
\end{pf}
\end{section}

\begin{section}{Proof of Lemma~\ref{lemma:propositionSphericalCaps}}\label{sec:ProofSphericalCaps}
We present now a proof of Lemma~\ref{lemma:propositionSphericalCaps}. By Proposition~\ref{thm:mainSphericalCap} we know that:
\begin{equation}\label{eqn:sc}
\alpha(\varepsilon) = \Delta(\mathcal{C}_{c, k}),
\end{equation}
for some spherical cap $\mathcal{C}_{c,k} \subset \sphere$, where  $\sigma^{n-1}(\mathcal{C}_{c, k}) = \varepsilon$. It is known (see e.g. \cite{sphericalCapRef}) that the area of such $\mathcal{C}_{c, k}$, is given by the equation:
\begin{equation}\sigma^{n-1}(\mathcal{C}_{c, k}) = \frac{I\left(1-\Delta(\mathcal{C}_{c,k})^2; \frac{n-1}{2}, \frac{1}{2}\right)}{2}.
\end{equation}
Since, \mbox{$\sigma^{n-1}(\mathcal{C}_{c, k})= \varepsilon$,} we get the following set of equations:
\begin{eqnarray}\nonumber \varepsilon &=& \frac{I\left(1- \Delta(\mathcal{C}_{c,k})^2;\frac{n-1}{2}, \frac{1}{2}\right)}{2} \\
\nonumber 1- \Delta(\mathcal{C}_{c, k})^2 &=&  I^{-1}\left(2\varepsilon; \frac{n-1}{2}, \frac{1}{2}\right) \\
\label{eqn:last}\Delta(\mathcal{C}_{c, k})^2 &=&  1- I^{-1}\left(2\varepsilon; \frac{n-1}{2}, \frac{1}{2}\right).
\end{eqnarray}
Then, the inequalities \eqref{eqn:sc} and \eqref{eqn:last} imply the inclusion given in \eqref{eqn:ballinc}, which concludes the proof.

%\begin{equation}\label{eqn:alphaEpsilon}\alpha(\varepsilon) = \sqrt{1- I^{-1}\left(\frac{\varepsilon\Gamma(\frac{d}{2})}{\pi^{d/2}}; \frac{d-1}{2}, \frac{1}{2}\right)}.\end{equation}

\end{section}


%
% 
%
%\begin{lemma}\label{lemma:epsilon1}Let $\varepsilon \in (0, 1)$. Consider a set of matrices $A \in \calM$, and an ellipsoid $E_P$ satisfying
%\begin{equation*}(A_jx)^TP(A_jx) \leq \gamma^* x^TPx, \quad \forall\, x \in E_P \setminus E', \forall\,j \in, M,\end{equation*}
%for some $E' \subset E_P$ where $\sigma^{n-1}(E') \leq \varepsilon$, then we have:
%\begin{equation*}(\bar{A}_jLx)^T(\bar{A}_jLx) \leq \gamma^* x^TPx, \quad \forall\, x \in \sphere \setminus \sphere', \forall\,j \in, M,\end{equation*}
%where $\bar{A}:=L^{-1}AL$, for some $\sphere'  \subset \sphere$ where \mbox{$\sigma^{n-1}(\sphere') \leq \frac{1}{\lambda_{\min}(L)^n} \varepsilon$}
%\end{lemma}
%
%
%Given the bound $\sigma^{n-1}(\sphere')\leq \frac{1}{\lambda_{\min}(L)^n} \varepsilon$, we now proceed to the second part where we compute $a$
%\begin{equation*}\bar{A} \sphere \subset \alpha(\varepsilon) \sphere, \forall\, j \in M,
%\end{equation*}
%where $\bar{A}:=L^{-1}AL$.
%
%
%
%
%
%%. We denote $\varepsilon:=\frac{m\varepsilon}{2} \sqrt{\frac{\lambda_{\max}(P)^n}{\det(P)}}$, where the additional factor $\frac{1}{2}$ follows from the homogeneity of the dynamics which implies a symmetry of $V_\sphere$, i.e., \begin{equation*}x \in V_{\sphere} \implies -x \in V_{\sphere}.\end{equation*}
%
%\begin{corollary}\label{cor:main} Let $\beta \in (0,1)$. We have: $$\delta(\beta, \omega_N) =  \Delta(\calC_{c, k}),$$ where
%$\sigma^{n-1}(\calC_{c,k}) = \varepsilon$.
%\end{corollary}
%
%
%\begin{lemma}\textcolor{blue}{Consider the sequential sampling $\omega_N$.} Let $d=\frac{n(n+1)}{2}$ and $P(\omega_N)$ be the optimal solution to the optimization problem $\Opt(\omega_N)$ and let $\lambda_{\max}(P(\omega_N))$ be the optimal objective function value for this $P(\omega_N)$. Then, $\lambda_{\max}(P(\omega_N))$ is uniformly bounded in $N$.
%\end{lemma}
%
%\begin{proof}We first define the following optimization problem:
%\begin{equation}\label{eqn:appendix}
%\begin{aligned}
%& \text{min}_{P} & & \lambda_{\max}(P) \\
%& \text{s.t.} 
%&  & (A_j x)^TP(A_j x) \leq {(1 +\eta)\gamma}^2 x^TPx,\,\forall (x, j) \in \omega_N, \\
%& && P \succeq I, \\
%\end{aligned}
%\end{equation}
%where we denote its optimal solution by $\lambda_{\max}(\gamma , \omega_N)$.
%
%Note that, for all $d \in \Z$ such that $0< d \leq N$ we have $\gamma^*(\omega_d) \leq \gamma^*(\omega_N)$. Also note that,
%\begin{equation*}\lambda_{\max}(\gamma^*(\omega_N), \omega_N) \leq \lambda_{\max}(\gamma^*(\omega_d), \omega_N).
%\end{equation*}
%But note that, there exists a $c > 0$ such that $\lambda_{\max}(\gamma^*(\omega_d), \omega_N) <  c$ since the problem \eqref{eqn:appendix} is strictly feasible for any $\gamma$ such that $\gamma \leq \gamma^*$. This implies: $\lambda_{\max}(\gamma^*(\omega_d), \omega_N) \leq c$, which completes the proof of this lemma.
%\end{proof}
%
%
%%\begin{corollary}\label{cor:final}Let $\beta \in (0,1)$. $$\lim_{N \to \infty} \delta(\beta, \omega_N) = 1.$$
%%\end{corollary}
%
%\begin{proof} 
%We first prove that $\lim_{N \to \infty} \varepsilon(\beta, N)= 0$. Note that, we can upper bound $1-\beta$ as follows:
%\begin{equation}\label{eqn:beta}1-\beta = \sum_{j=0}^d {{N}\choose{j}} \varepsilon^j (1-\varepsilon)^{N-j} \leq  (d+1)N^d (1-\varepsilon)^{N-d}.
%\end{equation}
%We prove $\lim_{N \to \infty} \varepsilon(\beta, N) = 0$ by contradiction. Assume that $\lim_{N \to \infty} \varepsilon(\beta, N) \not= 0$. This means that, there exists some $\delta > 0$ such that $\varepsilon(\beta, N) > \delta$ infinitely often. Then, consider the subsequence $N_k$ such that $\varepsilon(\beta, N_k) > \delta$, $\forall\, k.$ Then, by \eqref{eqn:beta} we have:
%\begin{equation*}1-\beta\leq (d+1)N_k^d (1-\varepsilon)^{N_k-d} \leq (d+1)N_k^d (1-\delta)^{N_k-d}\, \forall\,k \in \N. 
%\end{equation*}
%Note that $\lim_{k \to \infty}(d+1)N_k^d (1-\delta)^{N_k-d} = 0.$ Therefore, there exists a $k'$ such that, we have $$(d+1)N_{k'}^d (1-\delta)^{N_k'-d} < 1 - \beta,$$ which is a contradiction. Therefore, we must have  $\lim_{N \to \infty} \varepsilon (\beta, N) = 0$.
%
%\textcolor{red}{Still to show: If $\frac{\lambda{\max}(P)}{\lambda{\min}(P)}$ is uniformly bounded in $N$, then
%since $I^{-1}$ is monotonic in its first parameter, $\delta = \sqrt{1-\alpha}$ tends to $1$ as $\varepsilon \to 0$.}
%\end{proof}

\end{document}