% !TEX root = main.tex

\begin{theorem}\label{thm:mainTheorem}Consider an $n$-dimensional switching system as in \eqref{eq:switchedSystem}. For any given $\beta \in (0,1]$, $\eta > 0$ and a uniform random sampling $\omega_N \subset Z$, with $N \geq \frac{n(n+1)}{2}+1$, and let $\gamma^*(\omega_N) $ be the optimal solution to \eqref{eqn:campiOpt01}. Then, we can compute $\delta(\beta, \omega_N)$, such that with probability at least $\beta$ we have:
$$\rho \leq \frac{\gamma^*(\omega_N) (1+ \eta)}{\delta(\beta, \omega_N)},$$
where $\lim_{N \to \infty}\delta(\beta, \omega_N) = 1$.
\end{theorem}

\begin{proof}
Note that, by definition of $\gamma^*(\omega_N)$ we have:
\begin{equation*} (A_j x)^TP(A_j x) \leq {(\gamma^*(1+\eta))}^2 x^TPx, \quad \forall\, (x, j)  \in \omega_N \end{equation*}
for some $P \succ 0$. 
Note that the inequality \eqref{eqn:violation} in Theorem \ref{mainTheorem0} can be also written as:
\begin{equation}\label{eqn:violation2}\mathbb{P}^N\left\{ \mu(V(\omega_N)) \leq \epsilon \right\} \geq 1- I(1-\epsilon; N-d, d+1),\end{equation}
where $I(\ell;a,b)$ is the regularized in complete beta function. Then, for all $\epsilon \in (0,1)$ satisfying:
\begin{equation}\label{eqn:eps}\epsilon \leq 1- I^{-1}(\beta; N-d, d+1),\end{equation} we have $\mathbb{P}^N\left\{ \mu(V(\omega_N)) \leq \epsilon \right\} \geq \beta$.
Then, by Theorem \ref{thm:mainTheorem0} for all $\epsilon$ satisfying \eqref{eqn:eps}, with probability at least $\beta$ the following holds:
\begin{equation*} (A_jx)^TP(A_jx) \leq  {(\gamma^*(1+\eta))}^2 x^TPx, \quad \forall (x, j) \in Z \setminus V.\end{equation*}
By Theorem \ref{thm:mainTheorem01}, this implies that with probability at least $\beta$ the following also holds:
\begin{equation*} (A_jx)^TP(A_jx) \leq  {(\gamma^*(1+\eta))}^2x^TPx, \quad \forall x \in E_P \setminus E', \forall\, j \in M,\end{equation*}
for some $E'$ where $\sigma_P(E') \leq m \epsilon$.  Then, applying Theorem \ref{thm:mainTheorem1}, we can compute $$\delta(\beta, \omega_N) = \bar{\alpha}(m(1- I^{-1}(1-\beta; N-d, d+1)))$$ such that with probability at least $\beta$ we have:
\begin{equation*}
A_j E_P \subset \frac{\gamma^*(\omega_N)(1+\eta)}{\delta(\beta, \omega_N)}E_P, \forall\, j \in M,
\end{equation*}
where $\lim_{N \to \infty}\delta(\beta, \omega_N) = 1.$
By Property \ref{rem:scaling}, this means that $P$ defines a CQLF for the switched system \eqref{eq:switchedSystem} since we have $\frac{A_j}{\gamma^*(1+\eta)}^TP\frac{A_j}{\gamma^*(1+\eta)} \preceq P,\, \forall\, j$. Then, the followings holds with probability at least $\beta$:
$$\rho \leq \frac{\gamma^*(\omega_N) (1 + \eta)}{\delta(\beta, \omega_N)},$$
which completes the proof of the theorem.



%
%$$\frac{\calM}{\gamma^*} \conv(E_P \setminus E') \subset \convhull(E_P \setminus E'),$$
%for some $E' \subset E_P$, where $\sigma_P(E') \leq \epsilon$.
%Then, due to Theorem \ref{thm:mainTheorem0} we also have \eqref{eqn:contraction}, meaning:
%$$\left(\frac{\cM}{\delta \gamma^*}\right) \convhull (E_P \setminus X_P) \subset \convhull (E_P\setminus X_P).$$
%Then, $\delta\gamma^*$ is an upper bound on $\rho,$ with probability at least~$\beta$. 
\end{proof}

