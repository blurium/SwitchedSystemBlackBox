 % !TEX root = main.tex
A \emph{switched linear system} with a set of modes \mbox{$\mathcal{M}= \{A_i, i \in M \}$} is of the form:
\begin{equation}\label{eq:switchedSystem}x_{k+1} = f(k,x_k),\end{equation}
with $f(k,x_k) = A_{\tau(k)}x_k$ and switching sequence \mbox{$\tau: \N \to M$.} There are two important properties of switched linear systems that we exploit in this paper.
\begin{property}\label{property:homogeneity}
Let $\xi(x, k, \tau)$ denote the state of the system \eqref{eq:switchedSystem} at time $k$ starting from the initial condition $x$ and with switching sequence $\tau$. The dynamical system \eqref{eq:switchedSystem} is homogeneous: $\xi(\gamma x, k, \tau)= \gamma \xi(x, k, \tau).$
\end{property}

%\begin{property}\label{property:convpres}
%The dynamics given in \eqref{eq:switchedSystem} is convexity-preserving, meaning that for any set of points $X \subset \R^n$ we have:
%$$ f(\conv({X}))\subset \conv(f(X)). $$
%\end{property}

%Under certain conditions, deciding stability amounts to decide whether $\rho<1.$  In order to understand the quality of our techniques, we will actually try to prove lower and upper bounds on $\rho.$ 
%\textcolor{red}{@Raphael: We need  a theorem about the stability and $\rho<1$ here.}
%Under certain conditions deciding stability amounts to decide whether $\rho<1.$ 

The joint spectral radius of the set of matrices $\calM$ closely relates to the stability of the system \eqref{eq:switchedSystem} and is defined as follows:
\begin{definition}[from \cite{jungers_lncis}] Given a finite set of matrices \mbox{$\cM \subset \re^{n\times n},$} its \emph{joint spectral radius} (JSR) is given by
$$\rho(\cM) =\lim_{k\rightarrow \infty} \max_{i_1,\dots, i_k} \left\{||A_{i_1} \dots A_{i_k}||^{1/k}: A_{i_j}\in\cM\ \right\}. $$
\end{definition}

\begin{property}[Corollary 1.1, \cite{jungers_lncis}]
Given a finite set of matrices $\mathcal{M}$, the corresponding switched dynamical system is stable if and only if $\rho(\mathcal{M})<1$.
\end{property}

\begin{property}[Proposition 1.3, \cite{jungers_lncis}]\label{rem:scaling}
Given a finite set of matrices $\mathcal{M}$, and any invertible matrix $T$, 
$$\rho(\mathcal{M})=\rho(T \mathcal{M} T^{-1}),$$
i.e., the JSR is invariant under similarity transformations (and is a fortiori a homogeneous function: $\forall\  \gamma > 0$, $\rho \left(\cM/\gamma \right) = \cM/\gamma$).
\end{property}


%We start by computing a lower bound for $\rho$ which is based on the following theorem from the switched linear systems literature.

\begin{theorem}\cite[Theorem 2.11]{jungers_lncis}\label{thm:john}
For any finite set of matrices such that $\rho(\cM)<\frac{1}{\sqrt{n}},$ there exists a Common Quadratic Lyapunov Function (CQLF) for $\cM,$ that is, a $P\succ 0$ such that: $$\forall\ A\in \cM,\, A^TPA\preceq P. $$
\end{theorem}

CQLFs are useful because they can be computed (if they exist) with semidefinite programming (see \cite{boyd}), and they constitute a stability guarantee for switched systems as we formalize next.

\begin{theorem}\cite[Prop. 2.8]{jungers_lncis}\label{thm:cqlf} Consider a finite set of matrices $\cM$. If there exist some $\gamma \geq 0$ and $P \succ 0$ such that $$\forall\ A \in \calM, A^TPA \preceq \gamma^2P,$$ then $\rho(\cM) \leq \gamma$.
\end{theorem}

Note that the smaller $\alpha$ is, the tighter is the upper bound we get on $\rho(\cM)$. Therefore, we could consider, in particular, the optimal solution $\gamma^*$ of the following optimization problem:
\begin{equation}\label{eqn:campiOpt0}
\begin{aligned}
& \text{min}_{\gamma, P} & & \gamma\\
& \text{s.t.} 
&  & (Ax)^TP(Ax) \leq \gamma^2 x^TPx,\,\forall\ A \in \calM, \,\forall\, x \in \R^n,\\
& && P \succ 0. \\
\end{aligned}
\end{equation}
%where we were able to replace the condition $\forall\,x \in \R^n$ with $\,\forall\, x \in \sphere$, where $\sphere$ is the unit sphere. This is thanks to the Property \ref{property:homogeneity} which implies that it is sufficient to show that a CQLF is decreasing on a set enclosing the origin, e.g. the unit sphere. 

%Even though this upper bound is more difficult to obtain in a black-box setting where only a finite number of observations are available, in this section we leverage Theorem \ref{thm:john} in order to derive a straightforward lower bound.

Note that, instead of considering the set of constraints $x$ as being all $\R^n$, we could restrict it to the unit sphere $\sphere$. This is due to Property \ref{property:homogeneity}, since it implies that it is sufficient to show the decrease of a CQLF on an arbitrary set enclosing the origin, e.g., $\sphere$. Hence, we consider from now the following optimization problem, with its optimal solution that will also be $\gamma^{*}$:

\begin{equation}\label{eqn:campiOpt1}
\begin{aligned}
& \text{min}_{\gamma, P} & & \gamma \\
& \text{s.t.} 
&  & (Ax)^TP(Ax) \leq \gamma^2 x^TPx,\,\forall\ A \in \calM, \,\forall\, x \in \sphere,\\
& && P \succ 0. \\
\end{aligned}
\end{equation}



The two theorems above provide us with a \emph{converse Lyapunov result:} if there exists a CQLF, then our system is stable. If, on the contrary, there is no such stability guarantee, one may conclude a lower bound on the JSR. By combining these two results, one obtains an approximation algorithm for the JSR: the upper bound $\gamma^*$ obtained above is within an error factor of $\frac{1}{\sqrt{n}}$ of the true value. It turns out that one can still refine this technique, in order to improve the error factor, and asymptotically get rid of it. This is a well-known technique for the ``white-box'' computation of the JSR, which we summarize in the following corollary:

\begin{corollary}\label{cor:approx-products}
For any finite set of matrices such that $\rho(\cM)<\frac{1}{\sqrt[2l]{n}},$ there exists a CQLF for $$\cM^l:=\{A_{i_1},\dots, A_{i_{l}}:A_i\in\cM\},$$ that is, a $P\succ 0$ such that: $$\forall\ A\in \cM^l,\, A^TPA\preceq P. $$
\end{corollary}

\begin{proof}
It is easy to see from the definition of the JSR that $$\rho(\cM^l)=\rho(\cM)^l.$$ Thus, applying Theorem \ref{thm:john} to $\cM^l,$ one directly obtains the corollary.
\end{proof}


We can then consider from now, for any $l \in \N_{>0}$, the following optimization problem, whose solution $\gamma^{*}$ will be an upper bound on $\rho(\cM)$:

\begin{equation}\label{eqn:campiOpt2}
\begin{aligned}
& \text{min}_{\alpha, P} & & \gamma \\
& \text{s.t.} 
&  & (Ax)^TP(Ax) \leq \gamma^{2l} x^TPx,\,\forall\ A \in \calM^l, \,\forall\, x \in \sphere,\\
& && P \succ 0. \\
\end{aligned}
\end{equation}