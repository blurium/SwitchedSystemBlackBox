% !TEX root = main.tex
In this section, we show that using the Property \ref{property:convpres} and Property \ref{property:homogeneity}, by sampling finitely many points on a level set of a candidate CQLF, we can compute an upper bound on $\rho$. This section formalizes this discussion. Before preceding to the main theorem, we motivate the upcoming technical discussion by stating the following theorem to which most of this section is devoted.

\begin{theorem} \label{thm:mainTheorem0} Let $\epsilon \in (0,1]$, $\beta \in [0,1)$. Consider a uniform random sampling of $\sphere \times M$, denoted by $\omega_N$. Let $\gamma^*$ and $P$ be the optimal solution to:
\begin{equation}\label{eqn:campiOpt01}
\begin{aligned}
& \text{min} & & \gamma \\
& \text{s.t.} 
&  & (A_k x)^TP(A_k x) \leq \gamma x^TPx,\,\forall (x,k) \in \omega_N \\
& && P \succ 0. \\
\end{aligned}
\end{equation}
Then for all $Z_P$ with $\sigma_P(Z_P)\leq \epsilon$, we have with probability at least $\beta$:
\begin{equation} \label{eqn:exceptEps}(A_k x)^TP(A_k x)\leq \gamma^*x^TPx,\,\forall\, x \in E_P \setminus Z_P, \forall k \in M.\end{equation}
Moreover, we can compute $\delta(\beta, \omega_N) < \infty$  such that:
\begin{equation}\label{eqn:contraction}E_{\delta^2P} \subset  \convhull (E_{P} \setminus Z_{P}),
\end{equation}
and $\lim_{N \to \infty} \delta(\beta, \omega_N) = 1.$
\end{theorem}

Once Theorem \ref{thm:mainTheorem0} established the main result of this section follows.

\begin{theorem}[Main Theorem] \label{thm:mainTheorem} Let $\omega_N$ be a uniform sampling of $\sphere \times M$, where $N \geq \frac{n(n+1)}{2}+1$ and $\beta \in [0,1)$, we can compute $\delta(\beta, \omega_N) < \infty$ such that with probability at least $\beta$ we have $$\rho \leq \frac{\gamma^*(\omega_N)}{\delta}.$$ Moreover, $\lim_{N \to \infty} \delta(\beta, \omega_N) = 1$.
\end{theorem}

%\begin{theorem}
%Consider a black-box switching system and $N$ samples of its dynamics as in \eqref{eq:switchedSystem}. Consider the optimal solution $(\lambda^*,P)$ which minimizes $\lambda$ in \eqref{eq:lowerbound}. For any factor $1<\delta,$ one can compute the level of confidence $\beta$ such that $\rho<\delta\cdot\lambda^*.$ 
%% denote $\gamma(P,\epsilon)$ the largest $\gamma$ such that $\gamma^2y_i^TPy_i\leq x_i^TPx_i$ 
%\end{theorem}
\begin{proof}Note that, by definition of $\gamma^*$ we have:
\begin{equation*} (A_k x)^TP(A_k x) \leq \gamma^* x^TPx, \quad \forall\, (x,k)  \in \omega_N \end{equation*}
for some $P \succ 0$. Then, by Theorem \ref{thm:mainTheorem0} we have:
\begin{equation*} (A_k x)^TP(A_k x) \leq \gamma^*x^TPx, \quad \forall\, x \in E_P \setminus Z_P,\, \forall k \in M, \end{equation*}
which can be rewritten as:
\begin{equation*}\frac{\calM}{\gamma^*}(E_P \setminus Z_P) \subseteq E_P \setminus Z_P.
\end{equation*}
By Property \ref{property:convpres}, this implies:
$$\frac{\calM}{\gamma^*} \conv(E_P \setminus Z_P) \subset \convhull(E_P \setminus Z_P)$$
Then, due to Lemma \ref{thm:mainTheorem0} we also have \eqref{eqn:contraction} which means:
$(\cM/\delta \lambda^*) \convhull (E_P \setminus Z_P) \subset \convhull (E_P\setminus Z_P).$

Then, $\delta\lambda$ is un upper bound on $\rho,$ with probability at least $\beta$. 
\end{proof}
