% !TEX root = main.tex
\begin{subsection}{Proof of  $\lim_{N \to \infty}\delta(\beta, \omega_N) = 1$}
\label{app:bound}
Recall that, $\delta(\beta, \omega_N) = \alpha(m \kappa(P(\omega_N))\epsilon(\beta, \omega_N))$. We first show that $\kappa(P(\omega_N))$  is uniformly bounded in $N$. The optimization problem $\Opt(\omega_N)$ given in \eqref{eqn:campiOpt01}, with $\gamma^*(\omega_N)$ replaced by $\gamma^*(Z)(1+\frac{\eta}{2})$ is strictly feasible, and thus admits a finite optimal value $K$ for some solution $P_{\eta/2}$. Note that, $\lim_{N \to \infty} \gamma^*(\omega_N)= \gamma^*(Z)$ with probability 1. Thus, for large enough $N$, \mbox{$\gamma^*(\omega_N)(1+\eta) > \gamma^*(Z)(1+\frac{\eta}{2})$.} This also means that, for large enough $N$, $\Opt(\omega_N)$ admits $P_{\eta/2}$ as a feasible solution and thus the optimal value of $\Opt(\omega_N)$ is bounded by $K.$ In other words, \mbox{$\lambda_{\max}(P({\omega_N})) \leq K$.} Moreover, since  
$\lambda_{\max}(P(\omega_N))\geq 1$, we also have \mbox{$\det(P(\omega_N)) \geq 1$,} which means that
\begin{equation}\label{kappa}\kappa(P(\omega_N)) = \sqrt{\frac{\lambda_{\max}(P(\omega_N))^n}{\det(P(\omega_N))}} \leq \sqrt{K^n}.\end{equation}

We next show that for a fixed $\beta \in (0,1)$ $\lim_{N \to \infty} \epsilon(\beta, N) = 0.$ Note that, $\epsilon(\beta, N)$ is intrinsically defined by the following equation:
$$1-\beta = \sum_{j=0}^d {{N}\choose{j}} \epsilon^j (1-\epsilon)^{N-j}.$$
We can then upper bound the term $1-\beta$ as in:
\begin{equation}\label{eqn:beta}1-\beta \leq  (d+1)N^d (1-\epsilon)^{N-d}.\end{equation}
We prove $\lim_{N \to \infty} \epsilon(\beta, N) = 0$ by contradiction. Assume that $\lim_{N \to \infty} \epsilon(\beta, N) \not= 0$. This means that, there exists some $c > 0$ such that $\epsilon(\beta, N) > c$ infinitely often. Then, consider the subsequence $N_k$ such that $\epsilon(\beta, N_k) > c$, $\forall\, k.$ Then, by \eqref{eqn:beta} we have:
\begin{equation*}1-\beta\leq  (d+1)N_k^d (1-\epsilon)^{N_k-d}\hspace{-0.4mm} \leq \hspace{-0.7mm}(d+1)N_k^d (1-c)^{N_k-d}\, \forall\,k \in \N. 
\end{equation*}
Note that $\lim_{k \to \infty}(d+1)N_k^d (1-c)^{N_k-d} = 0.$ Therefore, there exists a $k'$ such that:
$$(d+1)N_{k'}^d (1-c)^{N_k'-d} < 1 - \beta,$$ which is a contradiction. Therefore, we must have  $\lim_{N \to \infty} \epsilon (\beta, N) = 0$.

Putting this together with \eqref{kappa}, we get:
$$\lim_{N \to \infty} m \kappa(P(\omega_N))\epsilon(\beta, \omega_N) = 0.$$ By the continuity of the function $I^{-1}$ this also implies: $\lim_{N \to \infty} \alpha(m \kappa(P(\omega_N))\epsilon(\beta, \omega_N)) = 1.$


\end{subsection}








%\begin{subsection}{Proof of  $\lim_{N \to \infty}\delta(\beta, \omega_N) = 1$ in Theorem \ref{thm:mainTheorem}} \label{app:delta}
%Recall that $\delta(\beta, \omega_N) = \alpha(m\kappa(P(\omega_N) \epsilon(\beta,N)).$
%We prove the limit of $\delta(\beta, \omega_N)$ in two parts. We first prove that $\kappa(P(\omega_N))$ is uniformly bounded in $N$. We then show that $\lim_{N \to \infty} \epsilon(\beta,N) = 0$. Then, the result of the theorem follows since $\lim_{\epsilon \to 0}\alpha(\epsilon) = 1$, which is immediate from the definition of $\alpha(\epsilon)$.
%
%\begin{lemma}\label{lem:bounded} We consider a sampling $\omega(N)$, where $N \in \N_{>0}$ and $N > d$. We assume that $\omega_N$ is sequential, i.e., $$\forall\ d \leq N_1  < N_2,\, \omega_{N_1} \subset \omega_{N_2}.$$ Let $P(\omega_N)$ be the optimal solution to the optimization problem $\Opt(\omega_N)$. Then, $\kappa(P(\omega_N))$ is uniformly bounded in $N$.
%\end{lemma}
%
%\begin{proof}We first define the following optimization problem:
%\begin{equation}\label{eqn:appendix}
%\begin{aligned}
%& \max_{P} & & \lambda_{\max}(P) \\
%& \text{s.t.} 
%&  & (A_j x)^TP(A_j x) \leq {(1 +\eta)\gamma}^2 x^TPx,\,\forall (x, j) \in \omega_N, \\
%& && \lambda_{\min}(P) = 1, \\
%\end{aligned}
%\end{equation}
%where we denote its optimal solution by $c_{\max}(\gamma , \omega_N)$.
%Recall that $\gamma^*(\omega_N)$ is the optimal solution to \eqref{eq:lowerbound}. Then we have:
%\begin{eqnarray*}c_{\max}(\gamma^*(\omega_N), \omega_N) & \leq & c_{\max}(\gamma^*(Z), \omega_N) \\ 
%& \leq & c_{\max}(\gamma^*(Z), \omega_{d+1}),
%\end{eqnarray*}
%where the first inequality follows from the fact that \mbox{$\gamma^*(Z) \geq \gamma^*(\omega_N)$.} The second inequality follows from the inclusion $\omega_{d+1} \subset\omega_N$, and the fact that as we add less constraints the value of the objective function of \eqref{eqn:appendix} increases. Let: $$\bar{\lambda}_{\max} = c_{\max}(\gamma^{*}(Z), \omega_{d+1}).$$
%Note that, $\bar{\lambda}_{\max} = c_{\max}(\gamma^{*}(Z), \omega_{d+1}) < \infty$ since we know $c_{\max}(\gamma^{*}(Z), Z) < \infty$ thanks to the additional $\eta$ factor. We then have $c_{\max}(\gamma^{*}(\omega_N), \omega_N) \leq \bar{\lambda}_{\max}.$  Note that, the optimal solution of the optimization problem \eqref{eqn:appendix} will in fact satisfy $\lambda_{\min}(P(\omega_N)) = 1$. Therefore, the optimal solution of \eqref{eqn:appendix} is an upper bound on the optimal solution of the problem \eqref{eqn:campiOpt01}. Therefore, we get $\lambda_{\max}(P(\omega_N)) \leq \bar{\lambda}.$ This immediately shows that:
%$$\kappa(P(\omega_N)) = \sqrt{\frac{\lambda_{\max}(P(\omega_N))^n}{\det(P(\omega_N))}} \leq \sqrt{\bar{\lambda}_{\max}^n},$$
%which means $\kappa(P(\omega_N))$ is uniformly bounded in $N$.
%
%% being the solution of the optimization problem \eqref{eqn:campiOpt0} with infinitely many constraints. Also note that, 
%%\begin{equation*}\lambda_{\max}(\gamma^*(\omega_N), \omega_N) \leq \lambda_{\max}(\gamma^{*}(Z), \omega_N)
%%\end{equation*}
%%%Recall that $\gamma^*(\omega_N)$ is the optimal solution to the problem \eqref{eq:lowerbound}. 
%%since we constrain the optimization more.
%%We also have \begin{equation*}\lambda_{\max}(\gamma^*(Z), \omega_N) \leq \lambda_{\max}(\gamma^{*}(Z), \omega_t))
%%\end{equation*}
%%by the same argument. We denote $$\bar{\lambda}_{\max} = \lambda_{\max}((1+\eta)\gamma^{*}(Z), \omega_{d+1}).$$
%%
%%We then have $\lambda_{\max}((1+\eta)\gamma^{*}(\omega_N), \omega_N) \leq \bar{\lambda}_{\max} $.
%
%%This gives us that $\lambda_{\max}((1+\eta)\gamma^{*}(\omega_N), \omega_N)$ obtained by solving our optimization problem [cite?] is bounded by u, for any $N \geq d+1$.
%
%
%%Since the problem \eqref{eqn:appendix} is strictly feasible for $\gamma^*(\omega_t)(1 + \eta),$ there exists a $c > 0$ such that \mbox{$\lambda_{\max}(\gamma^*(\omega_t), \omega_N) <  c$.} This implies: $$\lambda_{\max}(P({\omega_N})) = \lambda_{\max}(\gamma^*(\omega_N), \omega_N) \leq c.$$ Since $\det(P(\omega_N)) \geq 1$, this means $$\kappa(P(\omega_N)) = \sqrt{\frac{\lambda_{\max}(P(\omega_N))^n}{\det(P(\omega_N))}} \leq \sqrt{c^n},$$ which completes the proof of this lemma.
%\end{proof}
%
%Note that, even though we proved that $\kappa(P(\omega_N))$ is uniformly bounded for the special case of sequential sampling, a similar probabilistic result can be proved even when new sample points are chosen for each value of $N$.
%
%We now show that for a fixed $\beta \in [0,1)$ $\lim_{N \to \infty} \epsilon(\beta, N) = 0.$ Note that, $\epsilon(\beta, N)$ is also intrinsically defined by the following equation:
%$$1-\beta = \sum_{j=0}^d {{N}\choose{j}} \epsilon^j (1-\epsilon)^{N-j}.$$
%We can then upper bound the term $1-\beta$ as in:
%\begin{equation}\label{eqn:beta}1-\beta \leq  (d+1)N^d (1-\epsilon)^{N-d}.\end{equation}
%We prove $\lim_{N \to \infty} \epsilon(\beta, N) = 0$ by contradiction. Assume that $\lim_{N \to \infty} \epsilon(\beta, N) \not= 0$. This means that, there exists some $c > 0$ such that $\epsilon(\beta, N) > c$ infinitely often. Then, consider the subsequence $N_k$ such that $\epsilon(\beta, N_k) > c$, $\forall\, k.$ Then, by \eqref{eqn:beta} we have:
%\begin{equation*}1-\beta\leq  (d+1)N_k^d (1-\epsilon)^{N_k-d} \leq (d+1)N_k^d (1-c)^{N_k-d}\, \forall\,k \in \N. 
%\end{equation*}
%Note that $\lim_{k \to \infty}(d+1)N_k^d (1-c)^{N_k-d} = 0.$ Therefore, there exists a $k'$ such that, we have $$(d+1)N_{k'}^d (1-c)^{N_k'-d} < 1 - \beta,$$ which is a contradiction. Therefore, we must have  $\lim_{N \to \infty} \epsilon (\beta, N) = 0$. Putting together this with Lemma~\ref{lem:bounded}, we get:
%\mbox{$\lim_{N \to \infty} m \kappa(P(\omega_N))\epsilon(\beta, \omega_N) = 0.$} By the continuity of the function $I^{-1}$ this implies: $\lim_{N \to \infty} \alpha(m \kappa(P(\omega_N))\epsilon(\beta, \omega_N)) = 1.$
%\end{subsection}
