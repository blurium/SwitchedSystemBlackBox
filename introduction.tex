% !TEX root = writeUpMainTheorems.tex
Stability analysis of dynamical systems is a challenging task. The existing methods to construct stability certificates such as Lyapunov functions require the knowledge of a model of the dynamical system of interest. \cite{lyapunov} However, in industrial applications, typically there is no explicit model of the system in the form of difference or differential equations. On the other hand for a wide variety of scenarios, simulating dynamical systems with different model parameters and initial conditions is usually possible, and computationally efficient. Therefore, it is valuable to be able to analyze the stability of a given dynamical system with an unknown model, directly from data. More formally, given a dynamical system as in:
\begin{equation}\label{eq:dynamicalsystemGeneral}x_{k+1} = f(k, x_k),
\end{equation}
where, $x_k \in \R^n$, k is index of time. Let $y_k := x_{k+1}$
We ask the following question: given N input-output pairs, $(x_1, y_1)$, $(x_2, y_2)$, $\ldots$, $(x_N, y_N)$ such that $y_{k} = f(k, x_k)$, what can we say about the stability of the system \eqref{eq:dynamicalsystemGeneral}? The answer is immediate when \eqref{eq:dynamicalsystemGeneral} is a linear time-invariant system, since we can simply identify the system by $n$ linearly independent output traces. In this paper, we seek the answer to this question for switched linear system for which immediately the problem immediately becomes nontrivial. Switched linear systems are in the form:
\begin{equation}\label{eq:switchedSystem}x_{k+1} = A_{\sigma(k)}x_k,\end{equation}
where, $\sigma: \N \to \{1,2, \ldots, m\}$ is the switching sequence and $A_{\sigma(k)} \in \calM$, for all $\sigma$ and $k$. Aside from out theoretical interest, switched systems model the behavior of dynamical systems in the presence of a known or unknown varying parameter. This parameter can model internal parameters such as model uncertainties, look-up tables as well as exogenous parameters such as inputs provided by a controller in a closed-loop control system. 

%To make our reasoning clearer, we introduce the \emph{Lyapunov exponent} of the system \eqref{eq:switchedSystem}, which is a numerical quantity describing its stability.
%\begin{definition} Given a dynamical system as in \eqref{eq:dynamicalsystem} its \emph{Lyapunov exponent} is given by
%$$\rho =\inf{\{r:\,\forall x_0, \exists C\in \re^+: \quad x(0)=x_0 \Rightarrow x(t)\leq Cr^t\}}. $$
%\end{definition}
%Under certain conditions, deciding stability amounts to decide whether $\rho<1.$  In order to understand the quality of our techniques, we will actually try to prove lower and upper bounds on $\rho.$ 
Assessing the stability of nonlinear systems by leveraging simulations has been an active area of research in the recent years. Topcu et.al. \cite{topcu} and Kapinski et.al. \cite{kapinski} construct Lyapunov function candidates using the simulation traces, however to be able to formally verify the constructed Lyapunov function, they require the knowledge of the full dynamics. In \cite{lazar} and \cite{lazar2} Bobiti and Lazar address this and provide sampling based probabilistic and deterministic guarantees of a given Lyapunov function candidate. The presented method requires the knowledge of how fast the output of the system can change as the initial condition changes, and moreover the number of samples needed increases exponentially in the dimension of the state, $n$.

The stability of switched systems closely relates closely to the \emph{joint spectral radius} (JSR) of the matrices appearing in \eqref{eq:switchedSystem}. Under certain conditions deciding stability amounts to deciding whether JSR less than one or not. There has been a lot of work on developing algorithms to approximate this quantity, when the matrices appearing in \eqref{eq:switchedSystem} are known. Therefore, our work also is connected to the identification of switched systems, since once the system \eqref{eq:switchedSystem} is identified one can then apply these well-established results. However, there are two main reasons behind our quest to directly to from data to stability and bypassing the identification phase: (1) Even when $\calM$ is known, approximating the JSR is NP-hard \cite{jungers}. (2) Identifying the set $\calM$ is also NP-hard. Therefore, the existing identification techniques can identify $\calM$ up to an approximation error. As a result, how to relate this identification error to an error in deciding the stability of \eqref{eq:switchedSystem} is still nontrivial.

In this paper, we present an algorithm to approximate the JSR of a switched linear system from $N$ input-output pairs. This algorithm provides an upper bound on the JSR with a user-defined confidence level. As the number of samples increases, this bound gets tight. Moreover, we characterize with a closed form expression what the exact trade-off between the tightness of this bound and the number of samples. In order to understand the quality of our technique, the algorithm also provides a deterministic lower-bound.
%
%In this paper, we consider discrete-time switched linear systems of the form:
%\begin{equation}\label{eq:dynamicalsystem}x_{k+1} = A_{\sigma(k)}x_k,
%\end{equation}
%where, $x_k \in \R^n$, k is index of time and $\sigma: \N \to \{1,2, \ldots, m\}$ is the switching sequence. Let $y_k := x_{k+1}$. 
%We ask the following question:
%\begin{centering}
%\emph{Given N input-output-matrix pairs, $(x_1, y_1)$, $(x_2, y_2)$, $\ldots$, $(x_N, y_N)$ such that
%\begin{equation}\label{eq:triples}y_{k} = A_{\sigma(k)}x_k,
%\end{equation} for some $\sigma(k),$ what can we say about the stability of the system \eqref{eq:dynamicalsystem}?
%\end{centering}
%The only instance where answer to this question has been clearly shown is for linear time-invariant systems.
%
%Note that if \eqref{eq:dynamicalsystem} is a linear time-invariant system with only one mode, this question is easily answered by observing $n$ linearly independent data points. Switched systems can be used to model the behavior of a system of interest for different values of a parameter that varies. This parameter can represent internal parameters such as model uncertainties, as well as exogenous parameters such as inputs provided by a controller in a closed-loop control system. 


%- Stability analysis of dynamical systems is a challenging task. \\
%- Most methods in literature rely on knowing the model. Cite \\
%- How to make the case about the assumption of switched linear yet, we don't construct a model? Here cite
%something that says that this is a hard problem in general. (What is the relationship between switching linear regression)
%- Here we present stability analysis from data, this is easy for linear systems. We present something for switch linear.



%I can refer to the condition number paper?