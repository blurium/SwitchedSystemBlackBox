% !TEX root = main.tex
%The proof of the theorem is two-part. We first prove the statement of the theorem when $P=I$, i.e., when $E_P = \sphere$. (Lemma \ref{lemma:epsilon1}) Then, to apply this result on an arbitrary ellipsoid defined by $P$, via a linear transformation mapping $L \in\calL(\R^n)$, we transform the problem into a new coordinates system, such that the ellipsoid $E_P$ maps to a sphere. We then compute an upper bound on the measure of the image of the set $E'$ in this new coordinate system. (Lemma \ref{lemma:epsilon2}) We finally use the invariance of JSR under similarity transformations stated in Remark \ref{rem:scaling} to tie these two results.
\begin{subsection}{Notation and Background}
Before proceeding to the main lemmas we use to prove Theorem \ref{thm:mainTheorem1}, we first introduce the necessary preliminary definitions and related background.

Let $\dist$ be a distance on $\R^n$. The distance between a set $X \subset \R^n$ and a point $p \in \R^n$ is $\dist(X, p):=\inf_{x \in X}\dist(x,p)$. Note that the map $p \mapsto \dist(X,p)$ is continuous on $\R^n$.

\begin{definition}%[Spherical Cap]
We define the \emph{spherical cap} on $\sphere$ for a given hyperplane $c^Tx = k$ as:
\begin{equation*}\calC_{c,k} := \{x \in \sphere : c^Tx >k\}.\end{equation*}
\end{definition}

\begin{remark}\label{lemma:muMonotone}Consider the spherical caps $\calC_{c, k_1}$ and $\calC_{c, k_2}$ such that $k_1 > k_2$, then we have:
$$\sigma^{n-1}(\calC_{c,k_1}) < \sigma^{n-1}(\calC_{c,k_2}).$$
\end{remark}


\begin{definition}A \emph{supporting hyperplane} of a set $X \subset \R^n$ is a hyperplane $\{x:c^Tx = k\}$ that has the following two properties:
\begin{itemize}
\item $X \subset \{x: c^Tx \leq k\}$ or $X \subset \{x: c^Tx \geq k\}$.
\item $X \cap \{x: c^Tx = k\} \not= \emptyset$.
\end{itemize}
\end{definition}

\begin{remark}\label{suppHyperplaneRemark}\cite{boyd} Consider a convex set $X \subset \R^n$. For every $x \in \partial X$, there exists a supporting hyperplane containing $x$. Moreover, if $X$ is smooth, then this supporting hyperplane is unique.
\end{remark}

\begin{remark}\label{prop:distance}The distance between the point $x=0$ and the hyperplane $c^Tx = k$ is $\frac{|k|}{\|c\|}$.
\end{remark}

We now define the function $\Delta: \wp(\sphere) \to [0,1]$ as:
\begin{equation}\label{shrinkage}
\Delta(X) := \sup \{r: r\ball \subset \conv(\sphere \setminus X)\}.\end{equation}
Note that, $\Delta(X)$ can be rewritten as:
\begin{equation}\label{shrinkage2}
\Delta(X) =  \dist(\partial  \conv(\sphere \setminus X), 0).
\end{equation}


\begin{lemma} \label{lemma:delta2} Consider the spherical cap $\calC_{c,k}$. We have:
$$\Delta(\calC_{c,k}) = \min\left(1, \frac{|k|}{\|c\|}\right).$$
\end{lemma}

\begin{proof}
Note that: $$\conv(\sphere \setminus X)= \left\{x \in \ball : c^Tx \leq k \right\}.$$
Then the following equalities hold:
\begin{eqnarray}
\nonumber \Delta(X) &=& \dist(\partial \conv(\sphere \setminus X), 0) \\
\nonumber &=& \min(\dist(\partial \ball, 0), \dist(\partial\{x : c^Tx \leq k\}, 0)) \\
\nonumber &=& \min(\dist(\sphere, 0), \dist(\{x : c^Tx = k\}, 0)) \\
\nonumber &=& \min\left(1, \frac{|k|}{\|c\|}\right).
\end{eqnarray}
\end{proof}

\begin{corollary}\label{lemma:deltaMonotone} Consider the spherical caps $\calC_{c, k_1}$ and $\calC_{c, k_2}$ such that $k_1 \leq k_2$. Then we have: $$\Delta(\calC_{c,k_1}) \leq \Delta(\calC_{c,k_2}).$$
\end{corollary}
\end{subsection}

\begin{subsection}{Preliminary Results}
\begin{lemma}\label{lemma:constructSC}For any set $X \subset \sphere$, there exist $c$ and $k$ such that $\calC_{c,k}$ satisfies:
\begin{equation*}\calC_{c,k} \subset X,
\end{equation*}
 and
\begin{equation}\Delta(\calC_{c,k}) = \Delta(X).
\end{equation}
\end{lemma}

\begin{proof} Let $\tilde{X} := \conv(S \setminus X)$.
Since $\dist$ is continuous and the set $\partial \tilde{X}$ is compact, there exists a point $x^* \in \partial \tilde{X}$, such that:
\begin{eqnarray}\nonumber \Delta(X) = \dist(\partial X_S, 0) = 
\label{deltaSupporting} \min_{x \in \partial \tilde{X}}\dist(x, 0) = \dist(x^*, 0).\end{eqnarray} 
Next, consider the supporting hyperplane of $\tilde{X}$ at $x^*$, which we denote by $\left\{x : c^Tx = k\right\}$. Note that this supporting hyperplane is a supporting hyperplane of the ball $(\Delta(X)\ball)$ at $x^*$ since we have:
\begin{equation*} \partial (\Delta(X)\ball) \subset \partial \tilde{X} \subset \left\{x : c^Tx = k\right\}.\end{equation*}  By Remark \ref{suppHyperplaneRemark}, this implies that $\left\{x : c^Tx = k\right\}$ is in fact the unique supporting hyperplane at $x^*$.
Then we have:
\begin{equation*}\Delta(X) =  \dist(x^*, 0) = \dist(\{x : c^Tx = k\}, 0) = \min\left(1, \frac{|k|}{\|c\|}\right).
\end{equation*}
Now, consider the spherical cap $\calC_{c,k}$. Then, by Lemma \ref{lemma:delta2} we have
$\Delta(\calC_{c,k}) =  \min\left(1, \frac{|k|}{\|c\|}\right)$. Therefore, $\Delta(X) = \Delta(\calC_{c,k})$.


We next show $\calC_{c, k} \subset X$. We prove this by contradiction. Assume $x \in \calC_{c,k}$ and $x \notin X$. Note that, if $x \notin X$, then $x \in \sphere \setminus X \subset \conv(\sphere \setminus X).$ Since $x \in \calC_{c,k}$, we have $c^Tx>k$. But due to the fact that $x \in \conv(\sphere \setminus X)$, we also have $c^Tx \leq k$, which leads to a contradiction. Therefore, $\calC_{c, k} \subset X$. 
\end{proof}

\begin{lemma}\label{thm:mainSphericalCap}Let $\calX_{\epsilon} = \{X \subset \sphere: \sigma^{n-1}(X) = \epsilon\}$. Then, for any $\epsilon \in (0,1)$, the function $\Delta(X)$ attains its minimum over $X_{\epsilon}$ for some $X$ which is a spherical cap.
\end{lemma}

\begin{proof}We prove this via contradiction. Assume that there exists no spherical cap in $\calX_{\epsilon}$ such that $\Delta(X)$ attains its minimum. This means there exists an $X^* \in \calX_{\epsilon}$, where $X^*$ is not a spherical cap and $\argmin_{X \in \calX_{\epsilon}}(\Delta(X))=X^*$. By Lemma \ref{lemma:constructSC}, we can construct a spherical cap $\calC_{c,k}$ such that $\calC_{c,k} \subset X^*$ and $\calC_{c,k} = \Delta(X^*)$. Note that, we further have $\calC_{c,k} \subsetneq X^*$, since $X^*$ is assumed not to be a spherical cap. This means that, there exists a spherical cap $\sigma^{n-1}(\calC_{c,k})$ such that $\sigma^{n-1}(\calC_{c,k}) < \epsilon$. 

Then, the spherical cap $\calC_{c, \tilde{k}}$ with $\sigma^{n-1}(\calC_{c, \tilde{k}}) = \epsilon$, satisfies $\tilde{k} < k$ by Remark \ref{lemma:muMonotone}. This implies $$\Delta(\calC_{c, \tilde{k}}) < \Delta(\calC_{c, k}) = \Delta(X^*)$$ by Corollary \ref{lemma:deltaMonotone}. Therefore, $\Delta(\calC_{c, \tilde{k}}) < \Delta(X^*)$. This is a contradiction since we initially assumed that $\Delta(X)$ attains its minimum over $\calX_{\epsilon}$ at $X^*$.
\end{proof}
\end{subsection}

\begin{subsection}{Proof of Lemma \ref{lemma:eps}}
%\begin{lemma}\label{lemma:eps}Let $X_\epsilon \subset \sphere$ such that $\sigma^{n-1}(X_\epsilon) \leq \epsilon$ for some $\epsilon \in (0, 1)$, then the following holds:
%\begin{equation}\label{eqn:ballinc}\Delta(X_\epsilon) \geq \alpha(\epsilon)^2.\end{equation}
%where  \begin{equation}\label{defn:alpha}\alpha(\epsilon) = 1- I^{-1}\left(\frac{\epsilon\Gamma(\frac{n}{2})}{\pi^{n/2}}; \frac{n-1}{2}, \frac{1}{2}\right),\end{equation} and $\Gamma(x)=\int_{0}^{\infty} t^{x-1} e^{-t} \text{d}t.$ Here $I^{-1}$ is the inverse incomplete beta function, i.e.,  $I^{-1}(y; a,b)= x$ where $I(x;a,b)=y$.
%\end{lemma}

\begin{proof}Let the function $\Delta(X)$ be defined as in \eqref{shrinkage}. Then by Lemma \ref{thm:mainSphericalCap} we know that:
\begin{equation}\label{eqn:sc}\Delta(X_\epsilon) \geq \Delta(\calC_{c, k}),
\end{equation}
for some spherical cap $\calC_{c,k} \subset \sphere$, where  $\sigma^{n-1}(\calC_{c, k}) = \epsilon$. It is known (see e.g. \cite{sphericalCapRef}) that the area of such $\calC_{c, k}$, is given by the equation:
\begin{equation}\sigma^{n-1}(\calC_{c, k}) = \frac{\pi^{d/2}}{\Gamma[\frac{d}{2}]}I\left(1-\Delta(X_\epsilon)^2; \frac{d-1}{2}, \frac{1}{2}\right),
\end{equation}
where $I$ is the regularized incomplete beta function. Since, \mbox{$\sigma^{n-1}(X_\epsilon)\leq \epsilon$,} we get the following set of equations:
\begin{eqnarray}\nonumber \frac{\epsilon \Gamma[\frac{d}{2}]}{\pi^{d/2}} &\leq& I\left(1- \Delta(X_\epsilon)^2;\frac{d-1}{2}, \frac{1}{2}\right) \\
\nonumber 1- \Delta(\calC_{c, k})^2 &\leq&  I^{-1}\left(\frac{\epsilon\Gamma(\frac{d}{2})}{\pi^{d/2}}; \frac{d-1}{2}, \frac{1}{2}\right) \\
\label{eqn:last}\Delta(\calC_{c, k})^2 &\geq&  1- I^{-1}\left(\frac{\epsilon\Gamma(\frac{d}{2})}{\pi^{d/2}}; \frac{d-1}{2}, \frac{1}{2}\right)
\end{eqnarray}
The inequalities \eqref{eqn:last} and \eqref{eqn:sc} imply the inclusion given in \eqref{eqn:ballinc}, where \begin{equation}\label{eqn:alphaEpsilon}\alpha(\epsilon) = \sqrt{1- I^{-1}\left(\frac{\epsilon\Gamma(\frac{d}{2})}{\pi^{d/2}}; \frac{d-1}{2}, \frac{1}{2}\right)}.\end{equation}

%We next show $\lim_{N \to 0}\alpha(\epsilon) = 0.$
%%\begin{corollary}\label{cor:final}Let $\beta \in (0,1)$. $$\lim_{N \to \infty} \delta(\beta, \omega_N) = 1.$$
%%\end{corollary}
%
%\begin{proof} 
%We first prove that $\lim_{N \to \infty} \epsilon(\beta, N)= 0$. Note that, we can upper bound $1-\beta$ as follows:
%\begin{equation}\label{eqn:beta}1-\beta = \sum_{j=0}^d {{N}\choose{j}} \epsilon^j (1-\epsilon)^{N-j} \leq  (d+1)N^d (1-\epsilon)^{N-d}.
%\end{equation}
%We prove $\lim_{N \to \infty} \epsilon(\beta, N) = 0$ by contradiction. Assume that $\lim_{N \to \infty} \epsilon(\beta, N) \not= 0$. This means that, there exists some $\delta > 0$ such that $\epsilon(\beta, N) > \delta$ infinitely often. Then, consider the subsequence $N_k$ such that $\epsilon(\beta, N_k) > \delta$, $\forall\, k.$ Then, by \eqref{eqn:beta} we have:
%\begin{equation*}1-\beta\leq (d+1)N_k^d (1-\epsilon)^{N_k-d} \leq (d+1)N_k^d (1-\delta)^{N_k-d}\, \forall\,k \in \N. 
%\end{equation*}
%Note that $\lim_{k \to \infty}(d+1)N_k^d (1-\delta)^{N_k-d} = 0.$ Therefore, there exists a $k'$ such that, we have $$(d+1)N_{k'}^d (1-\delta)^{N_k'-d} < 1 - \beta,$$ which is a contradiction. Therefore, we must have  $\lim_{N \to \infty} \epsilon (\beta, N) = 0$.
%
%\textcolor{red}{Still to show: If $\frac{\lambda{\max}(P)}{\lambda{\min}(P)}$ is uniformly bounded in $N$, then
%since $I^{-1}$ is monotonic in its first parameter, $\delta = \sqrt{1-\alpha}$ tends to $1$ as $\epsilon \to 0$.}
%\end{proof}




\end{proof}
\end{subsection}

%put this in main text!
%We are now ready to prove the first main lemma in this section.
%\begin{lemma}\label{lemma:epsilon1}Let $\epsilon \in (0, 1)$. Consider the set of matrices and $A \in \calM$ satisfying:
%\begin{equation}\label{eqn:contraction}(A_jx)^T(A_jx) \leq \gamma^* x^Tx, \quad \forall\, x \in \sphere \setminus \sphere', \forall\,j \in, M,\end{equation}
%where $\sphere' \subset \sphere$ and $\sigma^{n-1}(\sphere') \leq \epsilon$, then we can compute
%$\alpha(\epsilon)$ such that we have:
%\begin{equation*}
%A_j \sphere \subset \frac{\gamma^*}{\delta} \sphere,\quad \forall\, j \in M.
%\end{equation*}
%\end{lemma}
%
%\begin{proof}Note that, \eqref{eqn:contraction} implies that:
%$$A_j (\sphere \setminus \sphere') \subset \gamma^* \ball.$$
%Using Property \ref{property:convpres} this also implies:
%$$A_j \convhull(\sphere \setminus \sphere') \subset \convhull(A_j(\sphere \setminus \sphere')) \subset \gamma^* \ball .$$
%By using Lemma \ref{lemma:eps} and the definition of the function $\Delta(X)$ we have:
%$$A_j (\alpha(\epsilon)^2 \ball) \subset A_j (\convhull (\sphere \setminus \sphere')) \subset \ball, \quad  \forall\, j \in M,$$
%where $\alpha(\epsilon)$ is defined as in \eqref{defn:alpha}. Therefore, we get:
%$$A_j (\alpha(\epsilon)^2 \ball) \subset \gamma^*\ball,$$
%which implies that $\rho \leq \frac{\gamma^*}{\alpha(\epsilon)}.$
%\end{proof}

%We next show that we can apply Lemma \ref{lemma:epsilon1} even when the inequality \eqref{lemma:epsilon1} is satisfied on an ellipsoid instead of a sphere, and complete the proof of the theorem.

\begin{subsection}{Proof of Theorem ??}




\end{subsection}



%
% 
%
%\begin{lemma}\label{lemma:epsilon1}Let $\epsilon \in (0, 1)$. Consider a set of matrices $A \in \calM$, and an ellipsoid $E_P$ satisfying
%\begin{equation*}(A_jx)^TP(A_jx) \leq \gamma^* x^TPx, \quad \forall\, x \in E_P \setminus E', \forall\,j \in, M,\end{equation*}
%for some $E' \subset E_P$ where $\sigma^{n-1}(E') \leq \epsilon$, then we have:
%\begin{equation*}(\bar{A}_jLx)^T(\bar{A}_jLx) \leq \gamma^* x^TPx, \quad \forall\, x \in \sphere \setminus \sphere', \forall\,j \in, M,\end{equation*}
%where $\bar{A}:=L^{-1}AL$, for some $\sphere'  \subset \sphere$ where \mbox{$\sigma^{n-1}(\sphere') \leq \frac{1}{\lambda_{\min}(L)^n} \epsilon$}
%\end{lemma}
%
%
%Given the bound $\sigma^{n-1}(\sphere')\leq \frac{1}{\lambda_{\min}(L)^n} \epsilon$, we now proceed to the second part where we compute $a$
%\begin{equation*}\bar{A} \sphere \subset \alpha(\epsilon) \sphere, \forall\, j \in M,
%\end{equation*}
%where $\bar{A}:=L^{-1}AL$.
%
%
%
%
%
%%. We denote $\epsilon:=\frac{m\epsilon}{2} \sqrt{\frac{\lambda_{\max}(P)^n}{\det(P)}}$, where the additional factor $\frac{1}{2}$ follows from the homogeneity of the dynamics which implies a symmetry of $V_\sphere$, i.e., \begin{equation*}x \in V_{\sphere} \implies -x \in V_{\sphere}.\end{equation*}
%
%\begin{corollary}\label{cor:main} Let $\beta \in (0,1)$. We have: $$\delta(\beta, \omega_N) =  \Delta(\calC_{c, k}),$$ where
%$\sigma^{n-1}(\calC_{c,k}) = \epsilon$.
%\end{corollary}
%
%
\begin{lemma}\textcolor{blue}{Consider the sequential sampling $\omega_N$.} Let $d=\frac{n(n+1)}{2}$ and $P(\omega_N)$ be the optimal solution to the optimization problem $\Opt(\omega_N)$ and let $\lambda_{\max}(P(\omega_N))$ be the optimal objective function value for this $P(\omega_N)$. Then, $\lambda_{\max}(P(\omega_N))$ is uniformly bounded in $N$.
\end{lemma}

\begin{proof}We first define the following optimization problem:
\begin{equation}\label{eqn:appendix}
\begin{aligned}
& \text{min}_{P} & & \lambda_{\max}(P) \\
& \text{s.t.} 
&  & (A_j x)^TP(A_j x) \leq {(1 +\eta)\gamma}^2 x^TPx,\,\forall (x, j) \in \omega_N, \\
& && P \succeq I, \\
\end{aligned}
\end{equation}
where we denote its optimal solution by $\lambda_{\max}(\gamma , \omega_N)$.

Note that, for all $d \in \Z$ such that $0< d \leq N$ we have $\gamma^*(\omega_d) \leq \gamma^*(\omega_N)$. Also note that,
\begin{equation*}\lambda_{\max}(\gamma^*(\omega_N), \omega_N) \leq \lambda_{\max}(\gamma^*(\omega_d), \omega_N).
\end{equation*}
But note that, there exists a $c > 0$ such that $\lambda_{\max}(\gamma^*(\omega_d), \omega_N) <  c$ since the problem \eqref{eqn:appendix} is strictly feasible for any $\gamma$ such that $\gamma \leq \gamma^*$. This implies: $\lambda_{\max}(\gamma^*(\omega_d), \omega_N) \leq c$, which completes the proof of this lemma.
\end{proof}
%
%
%%\begin{corollary}\label{cor:final}Let $\beta \in (0,1)$. $$\lim_{N \to \infty} \delta(\beta, \omega_N) = 1.$$
%%\end{corollary}
%
%\begin{proof} 
%We first prove that $\lim_{N \to \infty} \epsilon(\beta, N)= 0$. Note that, we can upper bound $1-\beta$ as follows:
%\begin{equation}\label{eqn:beta}1-\beta = \sum_{j=0}^d {{N}\choose{j}} \epsilon^j (1-\epsilon)^{N-j} \leq  (d+1)N^d (1-\epsilon)^{N-d}.
%\end{equation}
%We prove $\lim_{N \to \infty} \epsilon(\beta, N) = 0$ by contradiction. Assume that $\lim_{N \to \infty} \epsilon(\beta, N) \not= 0$. This means that, there exists some $\delta > 0$ such that $\epsilon(\beta, N) > \delta$ infinitely often. Then, consider the subsequence $N_k$ such that $\epsilon(\beta, N_k) > \delta$, $\forall\, k.$ Then, by \eqref{eqn:beta} we have:
%\begin{equation*}1-\beta\leq (d+1)N_k^d (1-\epsilon)^{N_k-d} \leq (d+1)N_k^d (1-\delta)^{N_k-d}\, \forall\,k \in \N. 
%\end{equation*}
%Note that $\lim_{k \to \infty}(d+1)N_k^d (1-\delta)^{N_k-d} = 0.$ Therefore, there exists a $k'$ such that, we have $$(d+1)N_{k'}^d (1-\delta)^{N_k'-d} < 1 - \beta,$$ which is a contradiction. Therefore, we must have  $\lim_{N \to \infty} \epsilon (\beta, N) = 0$.
%
%\textcolor{red}{Still to show: If $\frac{\lambda{\max}(P)}{\lambda{\min}(P)}$ is uniformly bounded in $N$, then
%since $I^{-1}$ is monotonic in its first parameter, $\delta = \sqrt{1-\alpha}$ tends to $1$ as $\epsilon \to 0$.}
%\end{proof}
